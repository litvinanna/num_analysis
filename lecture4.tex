\section{Лекция 4 Сплайны}

На прошлых лекциях мы рассматривали задачи приближения функции \\
$f:I \rightarrow\mathbb{R}$ полиномами т.е. элементами линейного пространства $P_n$ разымерности $n$. Чтобы распростронить другие методы приближения, прежде всего надо ввести некоторое новое конечномерное векторное пространство заменяющее $P_n$. Такой заменой может служить \textbf{ пространство} $S_{n,v}(X,I)$ \textbf{сплайнов} степени $n\geq 1$ деффекта $v\geq 1$ $(v\leq n)$ построенное по разбиению $X = (x_0=a < x_1 < x_2\ldots < x_N=b)$. 
По определению функция $s:I\rightarrow \mathbb{R}$ принадлежит пространству $S_{n,v}(X,I)$, если выполняются следующие два условия:
\begin{enumerate}
	\item На каждом интервале $(x_i,x_{i+1})$, $s(x)$ это полином степени $n$
	\item На всем интервале $I=[a,b]$ функция $s(x)$ имеет $n-v$ непрерывных производных
\end{enumerate}
Таким образом сплайн это функция "склеенная" из полиномов так, чтобы во внутренних точках $x_i,$ $ i = 1\ldots N-1$ выполняются условия "склейки":
\begin{equation}
s^{\left( k\right) }\left( x_{i}-0\right) =s^{\left( k\right) }\left( x_{i}+0\right),
\begin{aligned}k=0,1\ldots n-v\\ i=1\ldots N-1\end{aligned}
\end{equation}
Здесь через $\varphi(s_i\pm 0)$ обозначаются пределы $\varphi(x)$ при $x \rightarrow x_i$ справа и слева от точки $x_i$ производная порядка $n-v+1$ уже может иметь разрывы в точках ???.
Ясно, что $S_{n,v}(X,I)$ - линейное пространство и его размерность:
\begin{equation}
dim S_{n,v}(X,I) = v(N-1) +n+1
\end{equation}
Докажем эту формулу. У нее имеется $N$ интервалов и в каждом из них свой полином порядка $n$, который зависит от $n+1$ параметра (коэффициентов полинома). Таким образом:
\begin{equation}
\textit{число параметров} = N(n+1)
\end{equation} 
С другой стороны, в каждой из $N-1$ внутренних точек должны быть непрерывны производные поряядка $0,1\ldots n-v$ и следовательно:
\begin{equation}
\textit{число условий} = (N-1)(n-v+1)
\end{equation}
Таким образом:
\begin{equation}
\begin{aligned}dim S_{n,v}(X,I) = \textit{число параметров - число условий} =\\
 N(n+1)-(N-1)(n-v+1)=v(N-1)+n+1\end{aligned}
\end{equation}
Наиболее часто используются сплайны деффекта $1$ и
\begin{equation}
dim  S_{n,1}(X,I) = N+n
\end{equation}
В пакетах прикладных программ встречаются такие понятия как \textbf{B-сплайны} и \textbf{фундаментальные сплайны}.