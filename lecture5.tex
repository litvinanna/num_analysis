\section{Метод наименьших квадратов(1)}
Нам удоюно изменить обозначения. Будем считать, что нас интересует зависимость величины $f(t_i)$ от параметра  $t$ ( $t$ -- время, температура и т.д.). Подчеркнём, что функция $f(t_i)$ неизвестна.

ЭД -- это таблица $t_i | b_i, i = 1 \dots m$, где $b_i$ -- результат измерения и 
\begin{equation}
	b_i = f(t_i) + \varepsilon_i, i = 1 \dots m
\end{equation}

Задача состоит в том, чтобы по ЭД  $t_i | b_i$ найти функцию $\overline f(t_i)$ такую, что 
\begin{equation}
	|f(t_i) - b_i| \leq \varepsilon,  i = 1 \dots n
\end{equation}
При этом предполагается, что ошибки $\varepsilon_i$ удовлетворяют условиям
\begin{equation}
	|\varepsilon_i | \leq \varepsilon , \quad \varepsilon \ll  |b_i| .
\end{equation}
Если такая функция $\overline f(t_i)$ найдена, то считается, что
\begin{equation}
	|\overline f(t_i)| - f(t_i) | \leq \varepsilon
\end{equation}
и это всё, на что мы можем рассчитывать.

\vspace{1cm}
\textbf{Метод наименьших квадратов} (МНК) --  это некоторая стратегия, позволяющая построоить функцию $\overline f(t_i)$.

В соответствии с МНК выбирается некоторая система функций $\varphi_k (t) $ и целое число $n$. Функуция $\overline f(t_i)$ ищется в виде 
\begin{equation}
	\overline{f} (t) = \sum_{k=1}^\inf {x_k \varphi_k (t) }
\end{equation}
При этом всегда предполагает, что $n$ много меньше $m$:
\begin{equation}
	n \ll m
\end{equation}

Неизвестные величины $x_1 \dots x_n$ находятся из условия минимальност величины $Q(x_1, x_2 \dots x_n)$, то есть из условия
\begin{equation} \label{eq:5.7}
	Q(x_1, x_2 \dots x_n) = min
\end{equation}
и по определению
\begin{equation} 
	Q(x_1, x_2 \dots x_n) = \sum_{i=1}^m {(f(t_i) - b_i)^2}
\end{equation}

Обоснование этой стратегии должно рассматриваться в курсе математической статистики.

Величина $Q(x_1, x_2 \dots x_n)$ имеет вид
\begin{equation}
	Q(x_1, x_2 \dots x_n) = Q_0 + \sum_{k=1}^m{Q_k x_k } + \sum_{k, l=1}^n {Q_kl x_k x_l}
\end{equation}
и величины $Q_0, Q_k, Q_kl$ зависят от $b_1 \dots b_n$.


В соответствии с уравнением \ref{eq:5.7} величины  $x_1 \dots x_n$ определяются из уравнений
\begin{equation} \label{eq:5.10}
	 \frac{\partial Q}{\partial x_k } = 0, k = 1 \dots n
\end{equation}

Это система из $n$ линейных уравнений с $n$ неизвестными. И , если $x_1^0 \dots x_n^0$ -- решение, то надо убедиться еще, что $Q(x_1^0 \dots x_n^0) = min$.

Возникают следующие вопросы: 
\begin{enumerate}[nolistsep]
	\item Как выбрать функции $\varphi_k (t) $  и величину $n$?
	\item каков явный вид уравнений \ref{eq:5.10}?
	\item Как найти решение $x_1^0 \dots x_n^0$ этих уравнений?
	\item Является ли найденное решение \textbf{устойчивым}, то есть как измениться решение $x_1^0 \dots x_n^0$, если $b_i$ заменить на  $b_i = \varepsilon_i $?
\end{enumerate}

Эти вопросы будут рассмотрены в этой и следующей лекциях.


\begin{equation}
	A =
	\begin{pmatrix}
	a_{11} & \dots & a_{1n}\\
	a_{21} & \dots & a_{2n}\\
	\vdots & \vdots & \vdots \\
	a_{m1} & \dots & a_{mn}
	\end{pmatrix}
\end{equation}

\begin{equation}
	\overline{f} (t_i) = \sum_{k=1}^n {a_k x_k}
\end{equation}

\begin{equation}
	A (BC) = (AB) C
\end{equation}

\begin{equation}
	TA \equiv A^T, T: M_{m\times n} \rightarrow M_{n\times m}
\end{equation}

\begin{equation}
	(AB) ^ T = B^T A^T
\end{equation}

\begin{equation}
	x = \begin{pmatrix}
		x_1\\
		\vdots\\
		x_n
	\end{pmatrix}
	\in M_{n\times 1}
\end{equation}

\begin{equation}
	(x, y) = \sum_{i=1}^n x_k y_k
\end{equation}

\begin{equation}
	\|x \| = (x, x)^frac{1}{2} = (\sum_{i=1}^n x_i^2)^frac{1}{2}
\end{equation}
\begin{equation}
	(x, y) = x^T y = y^T x, x,y \in M_(n \times 1 )
\end{equation}

\begin{equation}
	(y, Ax) = (x, A^T y), \forall A \in M_{n\times m}
\end{equation}
	

\begin{equation}
	Bx = a, B = A^T A \in M_{n\times n}, a=A^T b \in M_{n\times 1}
\end{equation}

\begin{equation}
	r(x) = Ax - b
\end{equation}

\begin{equation}
	Q(x_1 \dots x_n) \equiv Q(x) = \|r(x)\| ^2 = (r(x), r(x))
\end{equation}

\begin{equation}
	\phi (x, \varepsilon ) = Q (x + \varepsilon ) - Q(x)
\end{equation}

\begin{equation}
	\phi (x, \varepsilon ) = 2(\varepsilon, A^T A x) + (A\varepsilon, A\varepsilon)
\end{equation}