\section{Метод наименьших квадратов (1)}

\label{lecture5}
Нам удобно изменить обозначения. Будем считать, что нас интересует зависимость величины $f(t_i)$ от параметра  $t$ ( $t$ -- время, температура и т.д.). Подчеркнём, что функция $f(t_i)$ неизвестна.

ЭД -- это таблица $t_i | b_i, i = 1 \dots m$, где $b_i$ -- результат измерения и 
\begin{equation}
	b_i = f(t_i) + \varepsilon_i, i = 1 \dots m
\end{equation}

Задача состоит в том, чтобы по ЭД  $t_i | b_i$ найти функцию $\overline f(t_i)$ такую, что 
\begin{equation}
	|f(t_i) - b_i| \leq \varepsilon,  i = 1 \dots n
\end{equation}
При этом предполагается, что ошибки $\varepsilon_i$ удовлетворяют условиям
\begin{equation}
	|\varepsilon_i | \leq \varepsilon , \quad \varepsilon \ll  |b_i| .
\end{equation}
Если такая функция $\overline f(t_i)$ найдена, то считается, что
\begin{equation}
	|\overline f(t_i)| - f(t_i) | \leq \varepsilon
\end{equation}
и это всё, на что мы можем рассчитывать.

\vspace{1cm}
\textbf{Метод наименьших квадратов} (МНК) --  это некоторая стратегия, позволяющая построить функцию $\overline f(t_i)$.

В соответствии с МНК выбирается некоторая система функций $\varphi_k (t) $ и целое число $n$. Функуция $\overline f(t_i)$ ищется в виде 
\begin{equation}
	\overline{f} (t) = \sum_{k=1}^\inf {x_k \varphi_k (t) }
\end{equation}
При этом всегда предполагает, что $n$ много меньше $m$:
\begin{equation}
	n \ll m
\end{equation}

Неизвестные величины $x_1 \dots x_n$ находятся из условия минимальност величины $Q(x_1, x_2 \dots x_n)$, то есть из условия
\begin{equation} \label{eq:5.7}
	Q(x_1, x_2 \dots x_n) = min
\end{equation}
и по определению
\begin{equation} 
	Q(x_1, x_2 \dots x_n) = \sum_{i=1}^m {(f(t_i) - b_i)^2}
\end{equation}

Обоснование этой стратегии должно рассматриваться в курсе математической статистики.

Величина $Q(x_1, x_2 \dots x_n)$ имеет вид
\begin{equation}
	Q(x_1, x_2 \dots x_n) = Q_0 + \sum_{k=1}^m{Q_k x_k } + \sum_{k, l=1}^n {Q_kl x_k x_l}
\end{equation}
и величины $Q_0, Q_k, Q_kl$ зависят от $b_1 \dots b_n$.


В соответствии с уравнением \ref{eq:5.7} величины  $x_1 \dots x_n$ определяются из уравнений
\begin{equation} \label{eq:5.10}
	 \frac{\partial Q}{\partial x_k } = 0, k = 1 \dots n
\end{equation}

Это система из $n$ линейных уравнений с $n$ неизвестными. И , если $x_1^0 \dots x_n^0$ -- решение, то надо убедиться ещё, что $Q(x_1^0 \dots x_n^0) = min$.

Возникают следующие вопросы: 
\begin{enumerate}[nolistsep]
	\item Как выбрать функции $\varphi_k (t) $  и величину $n$?
	\item каков явный вид уравнений \ref{eq:5.10}?
	\item Как найти решение $x_1^0 \dots x_n^0$ этих уравнений?
	\item Является ли найденное решение \textbf{устойчивым}, то есть как измениться решение $x_1^0 \dots x_n^0$, если $b_i$ заменить на  $b_i = \varepsilon_i $?
\end{enumerate}

Эти вопросы будут рассмотрены в этой и следующей лекциях.

Рассмотрим первый из этих вопросов.
Из предыдущих лекций следует, что всегда можно выбрать 
\begin{equation}
	\varphi_k(t) = t^{k-1}
\end{equation}
Но возможны и другие выборы. Например, $\varphi_k(t)$ -- система В-сплайнов, $\varphi_k(t) = e^\lambda_k t$ и~т.д.

Рассмотрим второй вопрос. Для ответа на него удобно ввести матричные обозначения. Обозначим через $M_{m \times n}$ множество матриц, имеющих $m$ строк и $n$ столбцов. Матрица $A \in M_{m \times n}$ -- это просто таблица
\begin{equation}\label{eq:5.12}
	A =
	\begin{pmatrix}
	a_{11} & \dots & a_{1n}\\
	a_{21} & \dots & a_{2n}\\
	\vdots & \vdots & \vdots \\
	a_{m1} & \dots & a_{mn}
	\end{pmatrix}
\end{equation}
$M_{m \times n}$ -- это линейное пространство размерности $mn$. То есть, если $A,B \in M_{m \times n}$, то определена матрица $С = \alpha A + \beta B$. Умножение на $\alpha \in \mathbb{R}$ и сложение проводится покомпонентно.

В МНК матрицы возникают естественно, так как
 \begin{equation}
	\overline{f} (t_i) = \sum_{k=1}^n {a_k x_k}, a_{ik} = \varphi_k(t_i) 
\end{equation}
Матрица $A = (a_{ik}) \in M_{m \times n}$ называется \textbf{матрицей плана}.

Если матрица $A$ имеет $n$ столбцов, а матрица $B$ -- $n$ строк, то определено их произведение
\begin{equation}
	C = AB, c_{ij} = \sum_{k=1}^n{a_{ik}b_{kj}}
\end{equation}
Таким образом, если $A \in M_{m \times n}, B \in M_{n \times l}$, то $C = AB \in M_{m \times l}$. Это умножение ассоциативно, то есть
\begin{equation}\label{eq:5.15}
	A (BC) = (AB) C
\end{equation}
Кроме умножения определена операция транспонирования Т (сопряжения)
\begin{equation}
	TA \equiv A^T, T: M_{m\times n} \rightarrow M_{n\times m}, (A^T)_{ij} = a_{ji},
\end{equation}
и легко видеть, что
\begin{equation}\label{eq:5.17}
	(AB) ^ T = B^T A^T
\end{equation}

Вектор $x \in \mathbb{R}^n$ также удобно рассматривать как матрицы из $M_{n \times 1}, (\mathbb{R}^n = M_{n \times 1})$
\begin{equation}
	x = \begin{pmatrix}
		x_1\\
		\vdots\\
		x_n
	\end{pmatrix}
	\in M_{n\times 1}
\end{equation}

Для векторов $x, y \in \mathbb{R}^n$ определено скалярное произведение
\begin{equation}
	(x, y) = \sum_{i=1}^n x_k y_k, 
\end{equation}
и обычная (эвклидова) норма $\|x\|$ вектора $x$  -- это
\begin{equation} 
	\|x \| = (x, x)^frac{1}{2} = (\sum_{i=1}^n x_i^2)^frac{1}{2}
\end{equation}
В матричных обозначениях
\begin{equation}
	(x, y) = x^T y = y^T x, x,y \in M_(n \times 1 )
\end{equation}
Используя это замечание и равенства \ref{eq:5.15},\ref{eq:5.17}, получим, что
\begin{equation}\label{eq:5.22}
	(y, Ax) = (x, A^T y), \forall A \in M_{n\times m}
\end{equation}
	

Рассмотрим, как найти явный вид уравнений \ref{eq:5.10}. Прежде всего заметим, что в матричных обозначениях соотношения $\overline f(t_i) = b_i$ (если \ref{eq:5.12}) имеют вид $Ax=b$. При $m\gg n$ эта система уравнений решения не имеет, так как это $m$ уравнений с $n \ll m$ неизвестными. Как проще всего из системы $Ax=b$ получить систему из $n$ уравнений? Надо просто обе части равенства $Ax=b$ умножить на $A^T$. Тогда получим
\begin{equation}\label{eq:5.23}
	Bx = a, B = A^T A \in M_{n\times n}, a=A^T b \in M_{n\times 1}
\end{equation}

Покажем, что система совпадает с системой \ref{eq:5.10}. Можно, конечно, явно вычислить $\frac{\partial Q}{\partial x_n} = 0$,но лучше поступить иначе.

Введем \textbf{вектор невязки}
\begin{equation}
	r(x) = Ax - b.
\end{equation}
Тогда
\begin{equation}
	Q(x_1 \dots x_n) \equiv Q(x) = \|r(x)\| ^2 = (r(x), r(x)).
\end{equation}
Рассмотрим величины
\begin{equation}
	\phi (x, \varepsilon ) = Q (x + \varepsilon ) - Q(x).
\end{equation}
Здесь $\varepsilon \in \mathbb{R}^n$ произвольный вектор, длина которого "мала". Легко видеть, что
\begin{equation}\label{eq:5.27}
	\phi (x, \varepsilon ) = 2(\varepsilon, A^T A x) + (A\varepsilon, A\varepsilon)
\end{equation}

При выводе этого равенства надо использовать \ref{eq:5.22}.

Из \ref{eq:5.27} следует, что величина $Q(x)$ имеет минимум при $x = x^0 = (x_1^0, \dots x_n^0)$, если $x^0$ -- это решение системы уравнений \ref{eq:5.23}.
То есть мы нашли явный вид системы уравнений для $x$. Это система \ref{eq:5.23}, и она называется системой \textbf{нормальных уравнений}.