\section{Дискретное преобразование Фурье (ДПФ)}
\label{lecture10}

В лекции \ref{lecture9} было введено преобразование Фурье $f(t) \xrightarrow{\mathscr{F}} \hat{f}(\omega)$. На практике, как правило, известно только конечное число ее отсчетов $f(t_n),\quad n = 0 \; ...\;N-1$. \textbf{Дискретное преобразование Фурье} $\hat{f}^D(\omega)$ служит заменой $\hat{f}(\omega)$.

Определим ДПФ $\hat{f}^D(\omega)$. Пусть

\begin{equation}\label{10.1}
f(t) = 
\begin{cases}
0, t < 0 \\
0, t > T.
\end{cases}
\end{equation}
Тогда
\begin{equation}\label{10.2}
\hat{f}\omega = \int_{0}^{T}e^{-i\omega t}f(t)dt.
\end{equation}
Будем предполагать, что известны величины
\begin{equation}\label{10.3}
\begin{gathered}
f_n = f(n \Delta t), \quad n = 0 ... N-1 \\
\Delta t = \frac{T}{N}.
\end{gathered}
\end{equation}

ДПФ $\hat{f}^D(\omega)$ определяется формулой
\begin{equation}\label{10.4}
\hat{f}^D(\omega) = \sum_{n = 0}^{N - 1} e^{-i \omega nt}f_n\Delta t.
\end{equation}

Это просто результат замены интеграла в уравнении \ref{10.2} интегральной суммой.

Нас интересует конечное число значений $\hat{f}^D\omega$, то есть величины 

\begin{equation}\label{10.5}
A_j = \hat{f}^D(j \Delta \omega), \quad j = 0, 1 ... N.
\end{equation}

Возникает вопрос: как выбрать значение $\Delta \omega$.
В отличие от $ \hat{f}(\omega)$, $\hat{f}^D(\omega)$ периодическая функция с периодом $\omega$

\begin{equation}\label{10.6}
\hat{f}^D(\omega + \omega) = \hat{f}^D(\omega), \quad \omega = \frac{2\pi}{\Delta t}.
\end{equation}

Поэтому естественно выбрать $\Delta \omega$ из условия
\begin{equation}\label{10.7}
\Delta \omega = \frac{\omega}{N} = \frac{2\pi}{T} \text{ и } \Delta t \Delta \omega = \frac{2\pi}{N}.
\end{equation}

В этом случае
\begin{equation}\label{10.8}
A_j = \frac{1}{N}\sum_{n = 0}^{N - 1} e^{-i \frac{2\pi n}{N}} a_n, \quad i = 0 ... N - 1,
\end{equation}

и в этой формуле
\begin{equation}\label{10.9}
a_n = f_n T, \quad n = 0 ... N-1.
\end{equation}

В общем случае (отвлекаясь от формулы \ref{10.9}) говорят, что $\{A_j\}$ -- это \textbf{ДФП последовательности} $\{a_n\}$, и 

\begin{equation}\label{10.10}
\{a_n\} \xrightarrow{\mathscr{F}_D} \{A_j\}.
\end{equation}

Основанием для такого определения служит \textbf{формула обращения}

\begin{equation}\label{10.11}
 \{A_j\}\xrightarrow{\mathscr{F}^{-1}_D} \{a_n\} ,
\end{equation}
определяемая равенством

\begin{equation}\label{10.12}
a_n = \sum_{i = 0}^{N - 1} e^{i \frac{2\pi n}{N}} j A_j, \quad i = 0 ... N - 1.
\end{equation}

Докажем эту формулу. Если в \ref{10.12} подставить выражения для $A_i$ из \ref{10.8}, то должно выполняться равенство
\begin{equation}\label{10.13}
a_n = \sum_{k = 0}^{N - 1} \frac{a_k}{N} \sum_{i = 0}^{N - 1} e^{i \frac{2\pi }{N}}(n-k) j,
\end{equation}

которое действительно выполняется, так как

\begin{equation}\label{10.14}
\sum_{i = 0}^{N - 1} e^{i \frac{2\pi }{N}}(n-k) j = N\delta_{n,k}.
\end{equation}

Это равенство -- следствие формулы для суммы геометрической прогрессии
\begin{equation}\label{10.15}
\sum_{i = 0}^{N - 1} x^i = \frac{1-x^N}{1-x}.
\end{equation}

Мы определили ДПФ $\hat{f}^D(\omega)$, но нас интересует $\hat{f}(\omega)$. Рассмотрим вопрос о связи этих двух функций. Для достаточно гладкой функции $f(t)$  $\hat{f}(\omega)$ достаточно быстро убывает при $|\omega| \rightarrow \infty $. Это следует из того, что (см. \ref{9.21})

\begin{equation}\label{10.16}
f^{(k)}(t) \equiv \frac{d^k}{dt^n}f(t) = \int_{-\infty}^{+\infty} (-i\omega)^k \hat{f}(\omega)d\omega,
\end{equation}

и, следовательно,

\begin{equation}\label{10.17}
\int_{-\infty}^{+\infty}\omega^k \hat{f}(\omega)d\omega < \infty.
\end{equation}

Таким образом, определена функция

\begin{equation}\label{10.18}
\begin{gathered}
F(\omega) = \sum_{n = -\infty}^{+\infty} \hat{f}(\omega+ n\omega), \\
F(\omega+\omega) = f(\omega),
\end{gathered}
\end{equation}

имеющая, как и $\hat{f}^D(\omega)$, период $\omega$. 
Оказывается, что эта функция совпадает с $\hat{f}^D(\omega)$, то есть

\begin{equation}\label{10.19}
\hat{f}^D(\omega) = \sum_{n = -\infty}^{+\infty} \hat{f}(\omega+ n\omega).
\end{equation}

Эта формула называется формулой Пуассона.

Так как $\hat{f}(\omega)$ достаточно быстро убывает при $|\omega| \rightarrow \infty $, то можно предположить что, если $\Delta t$ достаточно мало, то

\begin{equation}\label{10.20}
\hat{f}^D(\omega) \simeq \hat{f}(\omega) \quad |\omega| < \frac{\omega}{2}.
\end{equation}

Интервал $|\omega| < \frac{\omega}{2}$ называется \textbf{интервалом Найквиста}.

Картина приближения функции $\hat{f}(\omega)$ с помощью $\hat{f}^D(\omega)$ при достаточно малом значении $\Delta t$ представлена на рисунке Х.

Если величина $\Delta t$ не достаточно мала, то возникает \textbf{эффект наложения} при котором вклад высоких частот искажает информацию о поведении $\hat{f}(\omega)$ при низких частотах (См. Рис 11).

Этот эффект все видели в кино, когда на экране колеса автомобиля вращаются в обратную сторону.

В пакетах прикладных программ прямое и обратное ДПФ реализуются с помощью алгоритма быстрого преобразования Фурье (БПФ). Если прямо вычислять $N$ значений $A_j$ по $N$ значениям $a_n$, то требуется $\sim N^2$ операций. Оказывается, что по крайней мере при $N = 2^k$ это можно сделать на $N \mathrm{ln}N$ операций. БПФ реализует такой алгоритм (алгоритм Кули-Тьюки). Алгоритм БПФ открыл путь к практическому применению ДПФ при больших $N$.