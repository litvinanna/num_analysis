\documentclass[oneside,final,12pt]{article} %одностороння печать, чистовая версия, размер кегля, класс документа
\usepackage{ucs} 
\usepackage[utf8x]{inputenc} 
\usepackage[T2A, T1]{fontenc} 
\usepackage[english,russian]{babel} %оформление кириллицей (подписи к таблицам и т.д.)
\usepackage{ifpdf}
\usepackage{float} %для плавающих картинок и таблиц
%\usepackage{wrapfig} %для плавающих картино


\ifpdf  %% если используется pdfTEX
\usepackage{cmap}  %поиск по кириллице в готовом pdf
\usepackage[pdftex]{graphicx} %работа с графикой 
\usepackage[unicode=true]{hyperref}
\usepackage{pdfpages}
\else   %% если используется не pdfTEX
\usepackage[dvips]{graphicx}
\fi
 

\usepackage{vmargin} %размеры полос набора
\setpapersize{A4} %формат бумаги
\setmarginsrb{30mm}{25mm}{25mm}{25mm}{0pt}{0mm}{0pt}{13mm} %размеры полей: левое, верхнее, правое, нижнее, 3*колонтитулы, расстояние между нижним краем нижней строки и нижним краем номера страницы
\usepackage{indentfirst} %красная строка для первого абзаца главы или параграфа
\sloppy %борьба с залезанием строк на поля путём изменения размеров пробелов
\usepackage{amsmath} %дополнительные средства для вёрстки формул
\everymath{\displaystyle}
\usepackage{breqn} %для dmath разбить длинную формулу
%\usepackage{esvect} %vectors

%\usepackage{amscd} %диаграммы
\usepackage{amsfonts} %дополнительные шрифты для формул
\usepackage{amssymb} %дополнительные символы для формул
\usepackage[nottoc,notlot,notlof]{tocbibind}

\pagestyle{plain} %включена нумерация страниц 
\renewcommand{\thesection}{\arabic{section}} %одинарная нумерация формул


\usepackage{verbatim} %comments

\usepackage[font=small]{caption}
\usepackage{booktabs} %отступы в tabular
\usepackage{colortbl} %раскаршивание таблиц
\usepackage{xcolor} %название цветов



\usepackage{lineno} %нумерация всех строк для отладки
\usepackage{enumitem} %особенности enumerate
\usepackage{pbox} % для переносов внутри ячейки таблицы
\usepackage{array} %??
\usepackage{longtable}% перенос таблиц на страницах

\newlength{\width}
\setlength{\width}{0.97\textwidth} % у меня почему-то textwidth шире, чем текст и таблицы вылезали, поэтому я сделала меру длины поменьше
\definecolor{dark-gray}{gray}{0.4} %определение цвета
\newcommand{\graytable}[0]{\arrayrulecolor{dark-gray}} % сделать таблицу серой
\newcommand{\thinrule}[0]{\specialrule{0.3pt}{4pt}{4pt}} % тоненькая линия
\newcommand{\verythinrule}[0]{\specialrule{0.1pt}{1pt}{1pt}} %очень тоненькая линия если надо
\newcommand{\invisiblerule}[0]{\specialrule{0pt}{2pt}{2pt}} %просто создать пустого места в таблице



%\usepackage{textgreek} % \textalpha греческие буквы не в math mode
\usepackage{subcaption} % несколько картинок в одной
\usepackage{multirow} % объединение ячеек
\setcounter{tocdepth}{2} % глубина содержания

\usepackage{setspace} %for setstretch
\usepackage{cite}

\numberwithin{equation}{section} % нумерация формул внутри главы

\usepackage{amsfonts} %для букв с двойными штрихами
\usepackage{mathrsfs} % еще шрифт

\begin{document}
        \begin{titlepage}

\newcommand{\HRule}{\rule{\linewidth}{0.3mm}} % Defines a new command for the horizontal lines, change thickness here

\center

\textbf{\textsc{\Large Московский государственный университет} \textsc{\large имени }\textsc{\Large М.В.Ломоносова}}
\\[0.3cm] 
\HRule 
\\[0.3cm]
\textbf{\textsc{\large Факультет биоинженерии и биоинформатики}}
\\[4.0cm]

\begin{spacing}{1.4}
{ \LARGE \bfseries Численные методы в задачах обработки данных} \\[1.0cm]

\end{spacing}
 
 
\Large \emph{}\\
Курс лекций \\
Лектор - Попов Дмитрий Александрович
\\[4cm]

\begin{abstract}
	Курс включает обзор основных численных методов, применяемых при обработке экспериментальных данных. Основные темы курса: полиномиальная аппроксимация, интерполяция, сплайны, численное дифференцирование и интегрирование, метод наименьших квадратов, методы решения систем линейных уравнений, решение нелинейных уравнений и оптимизация, анализ Фурье.
\end{abstract}


\vfill

{\large Москва \\ 2018}


\end{titlepage}

 \newpage
        \tableofcontents \newpage
%        
\begin{figure}[h] % picture
	\centering
	\includegraphics[width = 0.1\textwidth]{pics/alpha_struct.png}
	\caption{caption}
	\label{fig:plasmid}	
\end{figure}



\begin{figure*}[h] % two pictures
	\centering
	\begin{subfigure}[t]{0.1\linewidth}
		\includegraphics[width = \textwidth]{pics/alpha_struct.png}
		\caption{caption 1}\label{fig:alpha}
	\end{subfigure}
	\begin{subfigure}[t]{0.1\linewidth}
		\includegraphics[width = \textwidth]{pics/beta_struct.png}
		\caption{caption 2}\label{fig:beta}

	\end{subfigure}
	\caption{great caption}
	\label{fig:structs}
	
\end{figure*}



\begin{table}[p]
	\small
	\caption{caption}
	\label{table:strains}
	\begin{tabular}{ p{0.25\width - \tabcolsep} p{0.65\width - 2\tabcolsep}  p{0.1\width - \tabcolsep}}
		\graytable
		\toprule
		Название штамма & Описание & Источник \\ 
		\midrule
		W303 & MAT\textbf{a} ade2-101 his3-11 trp1-1 ura3-52 can1-100 leu2-3,112, GAL, psi+ & $^1$ \\  \invisiblerule
		\midrule

		
		блабла & блабал & jdsjf \\
		\bottomrule
	\end{tabular}
	
\end{table}


 \newpage
        \section{Введение}

В этой вводной лекции рассматривается вопрос о том, на каком этапе и какие численные методы возникают при обработке экспериментальных данных (ЭД).

Начнём с конкретного примера. В 1881 году Д.И.\,Менделеев исследовал зависимость растворимости $NaNO_{2}$ в воде от температуры. Были получены следующие данные:

\begin{table}[h]
	\small
	\caption{Зависимость растворимости соли от температуры в эксперименте Менделеева.}
	\label{table:mendeleev}
	\begin{tabular}{ p{0.25\textwidth} *{10}{|c} }

		T (температура), °C & 0 & 4 & 10 & 15 & 21& 29& 36& 51&68 & x \\
		\hline
		Масса $NaNO_{2}$ в 100~мл воды, г&
		66,7&71,0&76,3&80,6&85,7&99,4&99,4&113,6&125,1& y \\

	\end{tabular}
	
\end{table}

Это типичный пример исходных экспериментальных данных -- конечная таблица $x_i | y_i, i=1\dots n$. Задача состоит в том, чтобы найти зависимость $y = f(x)$ по этим данным. В рассматриваемом примере предполагается, что $f(x)$ неизвестна и предполагается, что $y_i = f(x_i) + \varepsilon _i$, где $\varepsilon _ i$ -- ошибки.

Что касается ошибок, то мы будем предполагать, что известна величина $\varepsilon$ такая, что с нужной вероятностью $|\varepsilon _i| \leq \varepsilon$ и $\varepsilon$ много меньше наблюдаемых значений ($\varepsilon \ll y_i$). В частности, если $\varepsilon_i$ независимые, одинаково распределенные случайные величины с нулевым средним $\langle \varepsilon_i \rangle = 0$ и дисперсией $\langle \varepsilon^2 \rangle$, то $\varepsilon \sim \sqrt{\langle \varepsilon^2 \rangle }$. В более общем случае $\varepsilon$ характеризуется величиной доверительного интервала. 

Таким образом, предполагается, что из априорных соображений или путем статистической обработки мы определили величину $\varepsilon$. Это всё, что нам нужно от статистики, и вопросы статистической обработки ЭД в лекциях рассматриваться не будут.



Если представить данные Д.И.\,Менделеева графически, то возникает картина, схематически представленная на рисунке \ref{fig:1.1}. 
\begin{figure}[h]
	\centering
 \begin{tikzpicture}[scale=1.1]
 \selectcolormodel{gray}
      \begin{axis}[%
      width = 0.8\textwidth,
      height = 0.4\textwidth,
        xmin=-0.5,
        xmax=75,
        ymax=130,
        ymin=60,
        axis y line*=left,
        axis x line*=bottom,
         axis lines = left,
       xtick = {0,4, 10, 15, 21, 29, 36, 51, 68},
        xticklabels = {$x_1$,$x_2$,$x_3$, $x_4$, $x_5$, $x_6$, $x_7$, $x_8$, $x_9$},
        ytick={99.4},
        yticklabels = {$y_6$},
        ylabel = {$y$}, ylabel style={rotate=-90},
        xlabel = {$x$},
        every axis x label/.style={
    		at={(ticklabel* cs:1.0)},
   			anchor=west,
			},
		every axis y label/.style={
    		at={(ticklabel* cs:1.0)},
    		anchor=south,
			},
      ]
        \addplot+[color=black, fill=black, only marks][error bars/.cd,y dir=both,y explicit, error bar style={line width=2pt}] coordinates {%
          (0,66.7) +- (0,5)
          (4,71.0) +- (0,5)
          (10,76.3) +- (0,5)
          (15,80.6) +- (0,5)
          (21,85.7) +- (0,5)
          (29,99.4) +- (0,5)
          (36,99.4) +- (0,5)
          (51,113.6) +- (0,5)
          (68,125.1) +- (0,5)

        };
        \addplot[dashed] coordinates {(-1, 99.4) (29, 99.4 )};
        \addplot[dashed] coordinates {(29, 99.4) (29, 60)};
        \addplot[dashed] coordinates {(-0.4, 69) (70, 130)}; 
        \draw [decorate, decoration={brace,amplitude=5pt,raise=2pt,mirror}] 
        (69,119.5) -- (69,130.5);
        \node[] at (72,125) {$\varepsilon$};
             \end{axis}
    \end{tikzpicture}
    \caption{Экспериментальные данные, представленные в графическом виде.}
    \label{fig:1.1}
\end{figure}

Эта картина позволяет предположить, что исходная зависимость имеет вид 
\begin{equation}
	y = f(x) = ax + b.
\end{equation}
 Возникает вопрос -- как определить величины $a$ и $b$?  Правильная (с точки зрения математической статистики) стратегия состоит в применении метода наименьших квадратов (МНК). Согласно этому методу величины $a$ и $b$ ищутся  из условия 
 \begin{equation}
	Q(a, b) = \sum^n_{i=1}{(a+bx_i -y_i)^2} = \min_{a,b}
\end{equation}

Так как величина $Q(a,b)$ квадратична по $a,b$, то условия
\begin{equation}
	\frac{\partial Q}{\partial a} = 0,    \frac{\partial Q}{\partial b} = 0
\end{equation}
приводят к системе из двух линейных уравнений. Их решение $a = 67,5, b = 0,87$ нашел Д.И.\,Менделеев. Необходимо ещё проверить, что для функции $f(x) = 67,5 + 0,87x$ выполняются условия $|f(x_i) - y_i| < \varepsilon$ и убедиться в том, что величины $a,b$ мало меняются при замене $y_i$ на $y_i = \varepsilon _i$. Если это так, то задача решена и при этом никаких численных методов нам не понадобилось. Это связано с тем, что зависимость простая (всего два параметра) и число измерений невелико. В случае более сложной зависимости возникает картина, представленная на следующем рисунке \ref{fig:1.2}.


\begin{figure}[h]
	\centering
 \begin{tikzpicture}[scale=1.1]
 \selectcolormodel{gray}
      \begin{axis}[%
      width = 0.8\textwidth,
      height = 0.4\textwidth,
        xmin=-2.7,
        xmax=6.1,
        ymax=5,
        ymin=-30,
        axis y line*=left,
        axis x line*=bottom,
        axis lines = left,
%        xtick = {0},
%        xticklabels = {$x_1$},
%        ytick={99.4},
%        yticklabels = {$y_6$},
        ylabel = {$y$}, ylabel style={rotate=-90},
        xlabel = {$x$},
        every axis x label/.style={
    		at={(ticklabel* cs:1.0)},
   			anchor=west,
			},
		every axis y label/.style={
    		at={(ticklabel* cs:1.0)},
    		anchor=south,
			},
      ]
        \addplot [mark=none, dashed]{x^3 - 5*x^2 - x };
        \addplot+[color=black, fill=black, only marks][error bars/.cd,y dir=both,y explicit, error bar style={line width=2pt}] coordinates {  
        (-2, -25) +- (0, 3)
        (-1.5, -16) +- (0, 3)
        (-1, -6) +- (0, 3)
        (-0.5, -1) +- (0, 3)
        (0, 1) +- (0, 3)
        (1, -5.5) +- (0, 3)
        (2, -13) +- (0, 3)
        (3, -20) +- (0, 3)
        (3.5, -22) +- (0, 3)
        (4, -19) +-(0, 3)
        (4.5, -15) +- (0, 3)
        (5, -7) +- (0, 3)


        };

        \draw [decorate, decoration={brace,amplitude=3pt,raise=2pt,mirror}] 
        (-1.4, -19) -- (-1.4, -13);
        \node[] at (-1, -16) {$\varepsilon$};
        
        \draw [decorate, decoration={brace,amplitude=3pt,raise=2pt,mirror}] 
        (5.1, -10) -- (5.1, -4);
        \node[] at (5.5, -7) {$\varepsilon$};
        
        
        
             \end{axis}
    \end{tikzpicture}
    \caption{Сложная экспериментальная зависимость, представленная в графическом виде.}
    \label{fig:1.2}
\end{figure}


В этом случае можно попытаться искать зависимость в виде 
\begin{equation} \label{eq:representation}
	f(x) = a_1 \varphi _1(x) + a_2 \varphi _2(x) + \dots + a_m \varphi _1(m),
\end{equation}
где $\varphi _k(x)$ -- известные функции, например
\begin{equation}
	\varphi _k(x) = x^{k-1},  k = 1 \dots m
\end{equation}

Согласно МНК величины $a_1 \dots a_m$ ищутся из условия

 \begin{equation}
	Q(a_1 \dots a_m) = \sum^n_{i=1}{(f(x_i) -y_i)^2} = \min_{a_1 \dots a_m}
\end{equation}

и определяются исходя из полученной системы из $m$ линейных уравнений.

Возникают следующие вопросы:
\begin{enumerate}
	\item Как выбрать функции $\varphi _k(x)$ и величины $m$;
	\item Как решать полученную систему линейных уравнений;
	\item Как определить устойчиво ли решение относительно малых изменений величин $y_i$.
\end{enumerate}


Выбирая набор функций $\varphi _k(x)$ мы должны быть уверенны, что любую зависимость можно с нужной точностью представить в виде \ref{eq:representation}. Эта задача решается в теории приближений (теории аппроксимаций) и ей посвящены 2, 3, 4 лекции. Вопросы 2,3 -- это вопросы линейной алгебры и они будут рассмотрены в 5, 6, 7, лекциях.

Выше предполагалось, что вид функции $f(x)$ неизвестен. В ряде случаев вид функции  $f(x)$ известен с точностью до конечного числа неизвестных параметров.

Рассмотрим эксперимент, в котором измеряется зависимость от времени концентрации $C(t)$ некоторого вещества, при это известно, что 
\begin{equation}
	C(t) = a_1e^{\lambda_1 t} + a_2 e^{\lambda_2 t},
\end{equation}
экспериментальные данные это таблица $x_i | y_i, i=1\dots n, x_i = t_i, y_i = C(t_i) + \varepsilon_i$. Если величины $\lambda_1, \lambda_2$ неизвестны, задача их определения из условия

 \begin{equation}
	Q(a_1,a_2,\lambda_1, \lambda_2) = \sum^n_{i=1}{(C(t_i) -y_i)^2} = \min_{a_1,a_2,\lambda_1, \lambda_2}
\end{equation}
приводит к системе нелинейных уравнений. Это задача из теории оптимизации. Здесь, по существу, идёт речь о методах решения нелинейных систем уравнений, которые будут рассмотрены в лекциях 9 и 10.

Последние две лекции будут посвящены использования преобразования Фурье в задачах обработки ЭД. Будет объяснено, что такое преобразование Фурье и его дискретные варианты (ДПФ), а также как ДПФ используется в задачах "сглаживания" ЭД и обнаружения периодических сигналов.

Чтобы ориентироваться в численных методах и использовать пакеты прикладных программ, необходимо знакомство с такими понятиями как норма функций, число обусловленности матрицы и её норма, QR и SVD разложение, неподвижная точка отображения и т.д. Все эти понятия будут введены ниже по мере их возникновения в рассматриваемых задачах.

В каждой лекции используется своя нумерация формул. При ссылках на формулы впереди указывается их номер ( 3.5 = формула 5 из лекции 3).







 \newpage
        \section{Лекция 2. Приближение функции полиномами}
В первой лекции было отмечено, что для описания зависимостей полезно иметь систему функций $\varphi _i(x)$ такую, что их значения легко вычисляются и любую непрерывную функцию $f: I \rightarrow \mathbb{R} $ заданную на интервале $ I = [a,b] $ можно с заданной точностью приблизить функциями вида 
\begin{equation}
\varPsi_n(x)=\sum_{i=0}^{n-1}{a_i\varphi_i(x)} \qquad i = 1, 2, 3, ...
\end{equation}

В качестве $\{\varphi_i(x)\}$ можно выбирать различные системы функций, но наиболее важен с практической точки зрения случай 


\begin{equation}
\varphi _i(x) = x^i \qquad i = 0, 1, 2, ...
\end{equation}

Тогда
\begin{equation}\label{eq:polynom}
\varPsi _n(x) = p_n(x) = a_0 + a_1x + a_{n-1}x^{n-1} 
\end{equation}


Полиномы $p_n(x)$ образуют n-мерное \textbf{линейное пространство $\mathscr{P}_n$}, и ниже речь идет о приближении функций  $f:I \rightarrow \mathbb{R} $ полиномами $p_n(x) \in \mathscr{P}_n $. Подчеркнем, что максимальная степень полинома из пространства  $\mathscr{P}_n$ равна $n - 1$.


Определение точности приближения требует введения понятия нормы $\parallel f \parallel$ функции $f$. 
Функции $f: I \rightarrow \mathbb{R} $ образуют линейное пространство $V[f]$, размерность которого бесконечна. Это означает, что существует функции $e_i(x) \; i = 1, 2, 3 ...$ такие, что любая функция $f \in V[f]$ может быть единственным образом представлена в виде бесконечного ряда 
\begin{equation}
f(x) = \sum_{i=0}^{\infty} {c_i e_i(x)}
\end{equation}
В этом случае говорят, что $\{e_i(x)\}$ \textbf{базис} в пространстве $V[f]$.

\textbf{Нормой} в любом векторном пространстве $V$ называется любая функция 
 
\begin{equation}
\parallel \; \parallel \; : V \rightarrow \mathbb{R}^+
\end{equation}
такая что, если $X\in V$, то

\begin{equation}
\begin{array}{l}
\parallel X \parallel \; \geq 0 \; \text{и}  \parallel X \parallel \; = 0 \Rightarrow X = 0 \\
\parallel \alpha X \parallel \; = \alpha \parallel X \parallel  \\ 
\parallel \alpha X + \beta Y\parallel \; \leq \; \parallel \alpha  X \parallel + \parallel \beta Y \parallel \text{(неравенство треугольника)}
\end{array}
\end{equation}

Если $V = \mathbb{R}^n$ (n-мерное вектроное пространство) и $X = (x_1 .. x_n)$, то обычная эвклидова норма $\parallel\;\parallel_2$ (длина вектора $X$) задается равенством
\begin{equation}
\parallel X\parallel_2 \; = \left[\sum_{i=1}^{n} x_i^2 \right]^{1/2}.
\end{equation}

Но в $\mathbb{R}^n$ существуют и другие нормы, например,

\begin{equation}
\begin{gathered}
\parallel X \parallel_1 \; = \sum_{i = 1}^{n}|x_i| \\
\parallel X \parallel_c \; = \max_{i = 1...n}|x_i|
\end{gathered}
\end{equation}

Эти формулы прямо приводят к определению нормы в пространстве функции $f: I \rightarrow \mathbb{R}$ и по определению

\begin{equation}
\begin{gathered}
\parallel f \parallel_2 \; = \left[ \int_{a}^{b} f^2(x)dx \right]^{1/2} \quad \text{($L^2$-норма)}\\
\parallel f \parallel_1 \; = \int_{a}^{b} |f(x)|dx \quad \text{($L^1$-норма)} \\
\parallel f \parallel_c \; = \max_{x \in I}|f(x)| \quad \text{(равномерная норма)}
\end{gathered}
\end{equation} 

С точки зрения приложений наибольший интерес представляет равномерная норма $\parallel \; \parallel_c	$ и точность приближения функции $f$ полиномом $p_n$ задается числом 
\begin{equation}
\parallel f - p_n \parallel_c \; = \max_{x\in [a, b]} |f(x)-p_n(x)|
\end{equation}


Теорема Вейерштрасса утверждает, что любую непрерывную функцию $f: I \rightarrow \mathbb{R}$  с любой заданной точностью $\varepsilon$ можно приблизить полиномом $p_n(x)$ $(n = n(\varepsilon))$. 
Эта теорема говорит только о существовании такого полинома $p_n$, но ничего не говорит о том, как его построить и какова завиимость $n$ от $\varepsilon$. 
Из теоремы Вейерштрасса следует, что $\{x^i\} \; (i = 0, 1, 2, ...)$ базис в пространстве $V[f]$ непрерывных функций $f$ на интервале $I$.

Гораздо интереснее следующая постановка задачи: с какой точностью заданная функция $f$ может быть приближена полиномом заданной степени. 
Оказывается, что среди полиномов степени $n - 1$ существует единственный \textbf{полином $p_n[f]$ наилучшeго приближения}. Если

\begin{equation}\label{eq:best_polynom}
\parallel f - p_n(f) \parallel_c \; = \varepsilon_n[f],
\end{equation}

то для любого полинома $p$ степени $\leq n - 1$

\begin{equation}
\parallel f - p \parallel_c \; > \varepsilon_n[f],
\end{equation}

Это тоже теорема существования, в которой ничего не говорится о том, как найти $p_n(f)$ и  $\varepsilon_n[f]$.

Построение системы полиномов $p_n(f)$ для заданной функции $f$ -- очень трудная задача, точное решение которой при всех $n$ известно только для некоторых функций $f$. 
Однако существует алгоритм посторения $p_n(f)$ для заданного n (например, алгоритм Ремеза).

Важно то, что для величины $\varepsilon_n(f)$ существуют хорошие оценки сверху и для этого не надо точно знать функцию $f$ и достаточно предположиить, что $f \in W^k(M_k, I)$. Это означает, что функция $f$ на интервале $I$ имеет $k$ непрерывных производных и $|f^{(k)}(x)|\leq M_k$. 

Таким образом,

\begin{eqnarray}
W^k[M_k, I] = \{f: I \rightarrow \quad , \text{ -- производные} \nonumber \\
f^{(i)}(x) \text{ непрерывны при $i \leq k$ и } |f^{(k)}(x)| \leq M_k \}
\end{eqnarray}


\textbf{Класс функций $W^k[M_k, I]$} удобен для характеристики точности приближений, так как для любой функции $f \in W^k[M_k, I]$ имеет место достаточно точная оценка

\begin{equation}
\varepsilon_n[f] \leq \left(\frac{b-a}{2}\right)^k \frac{A_k M_k}{n^k},\; A_k = \left(\frac{\pi k}{2}\right) ^k \frac{1}{k!}
\end{equation}

При $k \geq 2$ с помощью формулы Стирлинга получим, что 

\begin{equation}
\varepsilon_n[f] \leq \left(\frac{(b-a)4,3}{n}\right)^k M_k \quad \forall f \in W^k[M_k, I].
\end{equation}

Эта формула позволяет оценить, с какой точностью функция $f \in W^k[M_k, I]$ может быть приближена полиномами степени $\leq n-1$, но мы по-прежнему не имеем метода построения приближений.
Такой метод дает \textbf{интерполяция}.

Рассмотрим \textbf{разбиение} $X = (x_1, x_2,..., x_n)$ интервала $I = [a, b]$. Это означает просто, что заданы $n$ различных точек $x_i \in I, \; i = 1 ... n$ и $x_i \neq x_j$ и не предполагается, что $x_{i+1} > x_i$.

\textbf{Интерполяционный полином} $\pi(f|x) \in \mathscr{P}_n $ зависит от $n$-параметров -- коэффициентов $a_0, a_1 .. a_{n-1}$ \ref{eq:polynom}, которые определяются из условий

\begin{equation}\label{eq:interpolation}
\pi(f|x)(x_j) = f(x_j) \quad j = 1 .. n
\end{equation}

Это $n$ условий для определения $n$ коэффициентов  $a_0, a_1 .. a_{n-1}$ полинома

\begin{equation}
\pi(f|x)(x) = a_0 + a_1x + a_{n-1}x^{x-1}.
\end{equation}
Отсюда следует, что существует только один такой полином и этот полином можно выписать явно. 

Действительно, пусть известны полиномы $e^i(x_i) \in \mathscr{P}_n $, такие что

\begin{equation}
e^i(x_i)= \delta_{ij} =
\begin{cases}
1, i = j \\
0, i \neq j

\end{cases}
\end{equation}

Тогда 
\begin{equation}
\pi_n(f|x)(x) = \sum_{i = 1}^{n} f(x_i)e^i(x).
\end{equation}

Условие интерполяции \ref{eq:interpolation} выполнено так как 

\begin{equation}
\pi_n(f|x)(x_j) = \sum_{i = 1}^{n} f(x_i)\delta_{ij} = f(x_j).
\end{equation}

Легко догадаться, какой вид имеет полином $e^i(x)$. Он задается равенством 

\begin{equation}
\varepsilon^i(x) = \frac{(x-x_1)(x-x_2)..(x-x_{i-1})(x-x_{i+1}) ...(x-x_n)}{(x_i-x_1)(x_i-x_2)..(x_i-x_{i-1})(x_i-x_{i+1})..(x_i - x_n )}
\end{equation}

Мы построили \textbf{интерполяционный полином в форме Лагранжа}. 

Рассмотрим точность интерполяции. Она задается формулой 

\begin{equation}
\parallel f - \pi_n| e(x) \parallel_c \; \leq (1+ \lambda_n(f, x))\varepsilon_n(f)
\end{equation}

и во всяком случае 

\begin{equation}
\lambda_n(f, x) \geq \frac{e_n n}{8 \sqrt{\pi}} \quad (n \geq 2)
\end{equation}

Величина $\lambda_n(f, x)$ показывает насколько интерполяционный полином проигрывает полиному наилучшего приближения $p_n(f)$ \ref{eq:best_polynom}.


Эти величины сильно зависят от выбора разбиения $X$ и в случае \textbf{равномерного разбиения} для которого

\begin{equation}
x_i = a + \frac{b-a}{n-1}(i-1) \quad i = 1...n
\end{equation}

величина $\lambda_n(f, x)$ быстро растет с ростом $n$ и


\begin{equation}
\frac{2^{n-3}}{n^{3/2}} \leq \lambda_n(f, x) \leq 2^n
\end{equation} 


Таким образом, если  $f \in W^k[M_k, I]$, то ошибка интерполяции по равномерной сетке растет как $2^kn^{-k}$.

В 1901 году немецкий физик Рунге пытался интерполировать на интервале $[-1; 1]$ функцию

\begin{equation}
f(x) = \frac{1}{1+25x^2}
\end{equation}

Используя полином 20-ой степени он обнаружил, что при равномерном разбиении ошибка интерполяции катастрофически растет в окрестности точек $x = \pm 1$ и между точками $x = 0,9$ и $1$ она имеет порядок $10^2$.

Этот пример показывает, что использовать интерполяционные полиномы высокого порядка надо с большой осторожностью. Однако их можно использовать, если перейти к неравномерному разбиению, в котором шаг разбиения уменьшается при приближении к концу интервала.
Нужное разбиение задается формулой 

\begin{equation}\label{eq:27}
x_m = \frac{b+a}{2} + \frac{b-a}{2} \cos \frac{\pi(2m-1)}{2n} \quad m = 1...n
\end{equation}

И если $X = \{x_m\}$ и $f \in W^k[M_k, I]$ то 

\begin{equation}
\parallel \pi_n (f(x) -f) \parallel_c \; \leq \left(
9+ \frac{4}{\pi}\mathrm{ln} (n)\right) \left( \frac{b-a}{2}\right)^k \frac{M_kA_k}{n^k}
\end{equation}

Разбиение \ref{eq:27} называется Чебышевским, так как $\cos \frac{\pi(2m-1)}{2n}$ это нули полинома Чебышева $T_n(y) = \cos(n \; arccos(y))$

Эти полиномы играют большую роль в теории приближений.


%$\mathrm{ln}(x) = 10$


 \newpage
        \section{Интерполяционный полином в форме Ньютона. Численное дифференцирование и интегрирование.}
На прошлой лекции был определен интерполяционный полином в форме Лагранжа (\ref{eq:2.19}) (\ref{eq:2.21}). Этот же полином (он единственный) удобнее записывать в форме Ньютона:
\begin{dmath}
	\pi(f(x), x) = f(x_1) + f(x_1, x_2)(x-x_1) + f(x_1, x_2, x_3)(x-x_1)(x-x_2) + \dots + f(x_1, x_2, \dots, x_n)(x-x_1)\dots(x-x_{n-1})
\end{dmath}
Величины $f(x_1, x_2, \dots, x_k)$ называются разделенными разностями и определяются из рекуррентных соотношений:
\begin{dmath}
\begin{cases}
	f(x_1, x_2) = \frac{f(x_2) - f(x_1)}{x_2 - x_1} \\ 
	f(x_1, x_2, x_3) = \frac{f(x_2, x_3) - f(x_1, x_2)}{x_3 - x_1} \\ 
	f(x_1, x_2, x_3, x_4) = \frac{f(x_2, x_3, x_4) - f(x_1, x_2, x_3)}{x_4-x_1} \\
	\dots
\end{cases}
\end{dmath}
Так как формула (\ref{eq:3.1}) не зависит от нумерации точек разбиения, то для её доказательства достаточно перейти к нумерации, в которой $x_1$ заменяется на $x_j$ (???). Эта формула удобна тем, что при добавлении точки $x_{n+1}$ к формуле (\ref{eq:3.1}) надо просто прибавить $f(x_1, x_2, \dots, x_n, x_{n+1})(x-x_1)\dots(x-x_{n-1})(x-x_n)$. Кроме того, эта формула указывает на связь интерполяционного полинома с разложением в ряд Тейлора.\\
Чтобы это увидеть, рассмотрим равномерное разбиение (\ref{eq:2.24}), в котором $x_{i+1}-x_i=\delta$. Тогда 
\begin{dmath}
\begin{aligned}
	f(x_1, x_2) = \frac{f(x_1+\delta) - f(x_1)}{\delta} \simeq f^{(1)}(x_1) \\
	f(x_1, x_2, x_3) = \frac{1}{2\delta}f^{(1)}(x_2)-f^{(1)}(x_1) \simeq  \frac{1}{2}f^{(2)}(x_1) \\
	\dots
\end{aligned}
\end{dmath} \newpage
        \section{Лекция 4 Сплайны}

На прошлых лекциях мы рассматривали задачи приближения функции \\
$f:I \rightarrow\mathbb{R}$ полиномами т.е. элементами линейного пространства $P_n$ размерности $n$. Чтобы распространить другие методы приближения, прежде всего надо ввести некоторое новое конечномерное векторное пространство заменяющее $P_n$. Такой заменой может служить \textbf{ пространство} $S_{n,v}(X,I)$ \textbf{сплайнов} степени $n\geq 1$ дефекта $v\geq 1$ $(v\leq n)$ построенное по разбиению $X = (x_0=a < x_1 < x_2\ldots < x_N=b)$. 
По определению функция $s:I\rightarrow \mathbb{R}$ принадлежит пространству $S_{n,v}(X,I)$, если выполняются следующие два условия:
\begin{enumerate}
	\item На каждом интервале $(x_i,x_{i+1})$, $s(x)$ это полином степени $n$
	\item На всем интервале $I=[a,b]$ функция $s(x)$ имеет $n-v$ непрерывных производных
\end{enumerate}
Таким образом сплайн это функция "склеенная" из полиномов так, чтобы во внутренних точках $x_i,$ $ i = 1\ldots N-1$ выполняются условия "склейки":
\begin{equation}
s^{\left( k\right) }\left( x_{i}-0\right) =s^{\left( k\right) }\left( x_{i}+0\right),
\begin{aligned}k=0,1\ldots n-v\\ i=1\ldots N-1\end{aligned}
\end{equation}
Здесь через $\varphi(s_i\pm 0)$ обозначаются пределы $\varphi(x)$ при $x \rightarrow x_i$ справа и слева от точки $x_i$ производная порядка $n-v+1$ уже может иметь разрывы в точках ???.
Ясно, что $S_{n,v}(X,I)$ - линейное пространство и его размерность:
\begin{equation}
dim S_{n,v}(X,I) = v(N-1) +n+1
\end{equation}
Докажем эту формулу. У нее имеется $N$ интервалов и в каждом из них свой полином порядка $n$, который зависит от $n+1$ параметра (коэффициентов полинома). Таким образом:
\begin{equation}
\textit{число параметров} = N(n+1)
\end{equation} 
С другой стороны, в каждой из $N-1$ внутренних точек должны быть непрерывны производные порядка $0,1\ldots n-v$ и следовательно:
\begin{equation}
\textit{число условий} = (N-1)(n-v+1)
\end{equation}
Таким образом:
\begin{equation}
\begin{aligned}dim S_{n,v}(X,I) = \textit{число параметров - число условий} =\\
 N(n+1)-(N-1)(n-v+1)=v(N-1)+n+1\end{aligned}
\end{equation}
Наиболее часто используются сплайны дефекта $1$ и
\begin{equation}
dim  S_{n,1}(X,I) = N+n
\end{equation}
В пакетах прикладных программ встречаются такие понятия как \textbf{B-сплайны} и \textbf{фундаментальные сплайны}. \newpage
        \section{Метод наименьших квадратов(1)}
Нам удоюно изменить обозначения. Будем считать, что нас интересует зависимость величины $f(t_i)$ от параметра  $t$ ( $t$ -- время, температура и т.д.). Подчеркнём, что функция $f(t_i)$ неизвестна.

ЭД -- это таблица $t_i | b_i, i = 1 \dots m$, где $b_i$ -- результат измерения и 
\begin{equation}
	b_i = f(t_i) + \varepsilon_i, i = 1 \dots m
\end{equation}

Задача состоит в том, чтобы по ЭД  $t_i | b_i$ найти функцию $\overline f(t_i)$ такую, что 
\begin{equation}
	|f(t_i) - b_i| \leq \varepsilon,  i = 1 \dots n
\end{equation}
При этом предполагается, что ошибки $\varepsilon_i$ удовлетворяют условиям
\begin{equation}
	|\varepsilon_i | \leq \varepsilon , \quad \varepsilon \ll  |b_i| .
\end{equation}
Если такая функция $\overline f(t_i)$ найдена, то считается, что
\begin{equation}
	|\overline f(t_i)| - f(t_i) | \leq \varepsilon
\end{equation}
и это всё, на что мы можем рассчитывать.

\vspace{1cm}
\textbf{Метод наименьших квадратов} (МНК) --  это некоторая стратегия, позволяющая построоить функцию $\overline f(t_i)$.

В соответствии с МНК выбирается некоторая система функций $\varphi_k (t) $ и целое число $n$. Функуция $\overline f(t_i)$ ищется в виде 
\begin{equation}
	\overline{f} (t) = \sum_{k=1}^\inf {x_k \varphi_k (t) }
\end{equation}
При этом всегда предполагает, что $n$ много меньше $m$:
\begin{equation}
	n \ll m
\end{equation}

Неизвестные величины $x_1 \dots x_n$ находятся из условия минимальност величины $Q(x_1, x_2 \dots x_n)$, то есть из условия
\begin{equation} \label{eq:5.7}
	Q(x_1, x_2 \dots x_n) = min
\end{equation}
и по определению
\begin{equation} 
	Q(x_1, x_2 \dots x_n) = \sum_{i=1}^m {(f(t_i) - b_i)^2}
\end{equation}

Обоснование этой стратегии должно рассматриваться в курсе математической статистики.

Величина $Q(x_1, x_2 \dots x_n)$ имеет вид
\begin{equation}
	Q(x_1, x_2 \dots x_n) = Q_0 + \sum_{k=1}^m{Q_k x_k } + \sum_{k, l=1}^n {Q_kl x_k x_l}
\end{equation}
и величины $Q_0, Q_k, Q_kl$ зависят от $b_1 \dots b_n$.


В соответствии с уравнением \ref{eq:5.7} величины  $x_1 \dots x_n$ определяются из уравнений
\begin{equation} \label{eq:5.10}
	 \frac{\partial Q}{\partial x_k } = 0, k = 1 \dots n
\end{equation}

Это система из $n$ линейных уравнений с $n$ неизвестными. И , если $x_1^0 \dots x_n^0$ -- решение, то надо убедиться еще, что $Q(x_1^0 \dots x_n^0) = min$.

Возникают следующие вопросы: 
\begin{enumerate}[nolistsep]
	\item Как выбрать функции $\varphi_k (t) $  и величину $n$?
	\item каков явный вид уравнений \ref{eq:5.10}?
	\item Как найти решение $x_1^0 \dots x_n^0$ этих уравнений?
	\item Является ли найденное решение \textbf{устойчивым}, то есть как измениться решение $x_1^0 \dots x_n^0$, если $b_i$ заменить на  $b_i = \varepsilon_i $?
\end{enumerate}

Эти вопросы будут рассмотрены в этой и следующей лекциях.


\begin{equation}
	A =
	\begin{pmatrix}
	a_{11} & \dots & a_{1n}\\
	a_{21} & \dots & a_{2n}\\
	\vdots & \vdots & \vdots \\
	a_{m1} & \dots & a_{mn}
	\end{pmatrix}
\end{equation}

\begin{equation}
	\overline{f} (t_i) = \sum_{k=1}^n {a_k x_k}
\end{equation}

\begin{equation}
	A (BC) = (AB) C
\end{equation}

\begin{equation}
	TA \equiv A^T, T: M_{m\times n} \rightarrow M_{n\times m}
\end{equation}

\begin{equation}
	(AB) ^ T = B^T A^T
\end{equation}

\begin{equation}
	x = \begin{pmatrix}
		x_1\\
		\vdots\\
		x_n
	\end{pmatrix}
	\in M_{n\times 1}
\end{equation}

\begin{equation}
	(x, y) = \sum_{i=1}^n x_k y_k
\end{equation}

\begin{equation}
	\|x \| = (x, x)^frac{1}{2} = (\sum_{i=1}^n x_i^2)^frac{1}{2}
\end{equation}
\begin{equation}
	(x, y) = x^T y = y^T x, x,y \in M_(n \times 1 )
\end{equation}

\begin{equation}
	(y, Ax) = (x, A^T y), \forall A \in M_{n\times m}
\end{equation}
	

\begin{equation}
	Bx = a, B = A^T A \in M_{n\times n}, a=A^T b \in M_{n\times 1}
\end{equation}

\begin{equation}
	r(x) = Ax - b
\end{equation}

\begin{equation}
	Q(x_1 \dots x_n) \equiv Q(x) = \|r(x)\| ^2 = (r(x), r(x))
\end{equation}

\begin{equation}
	\phi (x, \varepsilon ) = Q (x + \varepsilon ) - Q(x)
\end{equation}

\begin{equation}
	\phi (x, \varepsilon ) = 2(\varepsilon, A^T A x) + (A\varepsilon, A\varepsilon)
\end{equation} \newpage
        \section{Метод наименьших квадратов (2)}

В лекции \ref{lecture5} было показано, что реализация МНК сводится к решению системы \textbf{нормальных} уравнений 
\begin{equation} \label{eq:6.1}
	Bx = a, где B = A^TA, a = A^Tb
\end{equation}
и $A=M_{m \times n}$  -- матрица плана. решение системы \ref{eq:6.1} называется \textbf{квазирешением} переопределенной и не имеющей решений (при $m\gg n$) системы уравнений 
\begin{equation} \label{eq:6.2}
	Ax = b
\end{equation}
Рассмотрим методы поиска квазирешений, то есть рещений системы \ref{eq:6.1}. 

В пакетах прикладных программ содержится большое количество различных методов решений систем вида \ref{eq:6.1}. Ниже будут приведены некоторые из этих методов, чаще всего используемые в контексте МНК. При этом мы ограничимся методами дающими точное значение квазирешения. В соответствии с этим, методы последовательных приближений (итерационные) рассматриваться не будут.

Предполагается, что ранг матрицы $B$ равен $n$
\begin{equation} \label{eq:6.3}
	rank B = n
\end{equation}
Следовательно
\begin{equation} \label{eq:6.4}
	det B \neq 0
\end{equation}
и следовательно решение $x$ сущетсвует и единственно.

Все описанные ниже методы решения системы \ref{eq:6.1} основаны на представлении матрицы $B$ в виде
\begin{equation} \label{eq:6.5}
	B = B_1 B_2 \text{ или } B = B_1 B_2 B_3
\end{equation}
где матрицы $B_i$ \textbf{легко обращаются}.
Так как
\begin{equation} \label{eq:6.6}
	(B_1B_2)^{-1} = B_2^{-1} B_1^{-1}, (B_1B_2B_3)^{-1} = B_3^{-1}B_2^{-1} B_1^{-1}
\end{equation}
то решение системы \ref{eq:6.1} имеет вид
\begin{equation} \label{eq:6.7}
	(B_1B_2)^{-1} = B_2^{-1} B_1^{-1}, (B_1B_2B_3)^{-1} = B_3^{-1}B_2^{-1} B_1^{-1}
\end{equation}
то решение системы \ref{eq:6.1} имеет вид
\begin{equation}
	x = B^{-1} a = B_2^{-1} B_1^{-1} a \text{ или } x = B_3^{-1}B_2^{-1} B_1^{-1}
\end{equation}

Легко обращаются следующие типы квадратных матриц $M_{n \times n}$. Это верхние и нижние треугольные матрицы и ортогональные матрицы.

\textbf{Верхнетреугольные матрицы} имеют вид

\begin{equation} \label{eq:6.8}
	L = \begin{pmatrix}
		e_{11} & e_{12} & \dots &e_{1(n-1)}&e_{1n} \\
		0 & e_{22} & \dots &e_{2(n-1)}&e_{2n} \\
		%		0 & 0 & \dots &e_{3(n-1)}&e_{3n} \\
		\vdots & \vdots & \ddots &\vdots &  \vdots\\
		0 & 0 & \dots & e_{(n-1)(n-1)} &e_{(n-1)n} \\
		0 & 0 & \dots & 0 &e_{nn} \\
	\end{pmatrix}
\end{equation}
и соответственно \textbf{нижнетреугольные} матрицы вида 

\begin{equation} \label{eq:6.9}
	L = \begin{pmatrix}
		e_{11}	&		&		&	&        \\
		&e_{22}	&   		& \text{\huge0} &\\
		&		&e_{33} &	&            \\
		&\dots & & \ddots &          \\
		& 		&   		&   & e_{nn} 
	\end{pmatrix}
\end{equation}
где заштрихованная область содердит ненулевые элементы. Уравнения $Lx = y$ легко решаются, а решение $x = L^{-1}y$ задает обратную матрицу $L^{-1}$.


Матрица $P \in M_{n \times n}$ называется \textbf{ортогональной}, если
\begin{equation} \label{eq:6.10}
	(Px, Py) = (x, y), \forall x, y \in \mathbb{R}^n = M_{n \times 1}
\end{equation}
откуда сразу следует, что
\begin{equation} \label{eq:6.11}
	P^{-1} = P^T .
\end{equation}

Произвольные матрицы $B, B'$ называются \textbf{подобными}, если существуют \textbf{обратимые} (имеющие обратную) матрицы $Q_1, Q_2$ такие что

 \newpage
        \section{Оптимимзация и методы решения систем нелинейных уравнений}

Начнем с примера. Пусть, как и в МНК, зависимость $\overline{f}(t)$ ищется из условия (см. \ref{eq:3.5})
\begin{equation} \label{eq:representation}
	Q(x_1, x_2, x_3, x_4) = \sum^m_{i=1}{(\overline{f}(t_i)-b_i)^2} = \min_{x_1 \dots x_4}
\end{equation}

Но теперь в отличие от (\ref{eq:5.5}) функция $\overline{f}(t)$ зависит от $x_3$, $x_4$ нелинейно
\begin{equation} \label{eq:representation}
	\overline{f}(t) = x_1e^{x_3 t} + x_2e^{x_4 t}
\end{equation}
Тогда условие (\ref{eq:7.1}) приводит к нелинейной системе уравнений
\begin{equation} \label{eq:representation}
	f_i(x) = 0 \quad
	f_i(x) \equiv f_i(x_1, x_2, x_3, x_4) = \frac{\partial Q}{\partial x_i}(x)
\end{equation}

\textbf{Задача оптимизации} состоит в поиске минимума величины 
\begin{equation} \label{eq:representation}
	Q(x) \equiv Q(x_1 \dots x_4), \qquad
	Q(x) > 0
\end{equation}
которая называется \textbf{ценой}. Как правило на $x$ накладываются условия вида $x \in U$, где область $U$ определяется условиями
\begin{equation} \label{eq:representation}
	a_i < x_i < b_i \quad \textrm{или} \quad ||x|| < R
\end{equation}

Это вносит дополнительные трудности, так как минимум $Q(x)$ может достигаться на границе $\partial U$  области $U \subset \mathbb{R}^n$. Этот случай надо исследовать отдельно, и мы его рассматривать не будем. Таким образом, предполагается, что минимум $Q(x)$ достигается во внутренней точке области $U$ и тогда задача оптимизации сводится к решению системы нелинейных уравнений
\begin{equation} \label{eq:representation}
	f_i(x) = \frac{\partial Q}{\partial x_i}(x) = 0 \quad x = (x_1 \dots x_4), i = 1 \dots n
\end{equation}

И обратно, задача решения системы уравнений
\begin{equation} \label{eq:representation}
	f_i(x) = 0, i = 1 \dots n
\end{equation}
сводится к задаче оптимизации. Для этого достаточно положить
\begin{equation} \label{eq:representation}
	Q(x) = \sum^n_{i=1}{f_i(x)^2}
\end{equation}

В этой лекции рассматривается задача решения системы уравнений (\ref{eq:7.7}). Это трудная задача. Все методы ее решения итерационные и имеют ограниченную область применения. 

\textbf{Общая схема итерационных} методов решения системы (\ref{eq:7.7}) (методов последовательных приближений) состоит в следующем.

Выбираются некоторое нулевое приближение $x^{(0)} = (x_1^{(0)} \dots x_4^{(0)})$ и задается алгоритм $A$ построения следующих приближений $x^{(n)}$: $x^{(n)} = A(x^{(n-1)})$, $n \geq 1$. В области $U \subset \mathbb{R}^n$ считается заданной норма $||x|| (x \in U)$ и пара $(x^{(0)}, A)$ должны быть выбраны так, что последовательность $x^{(n)}$ сходится к решению $x = x_0 = (x_{01} \dots x_{0n})$ системы (\ref{eq:7.7}), которую мы будем записывать в виде
\begin{equation} \label{eq:representation}
	f(x) = 0 \quad [f(x) \in \mathbb{R}^n \quad f(x) = (f_1(x) \dots f_n(x))]
\end{equation}

Таким образом должно выполняться условие сходимости
\begin{equation} \label{eq:representation}
	||x^{(n)} - x_0|| \to 0
\end{equation}

 \newpage
        \begin{figure}[h] % picture
	\centering
	\begin{tikzpicture}
\begin{axis}[
	width = 15cm,
	xmin=-13,   xmax=15,
	ymin=-10,   ymax=10,
	axis y line*=left,
    axis x line*=bottom,
    axis lines = left,
	xtick={12, 5.9, 1.6},
	xticklabels = {$x^{(0)}$, $x^{(1)}$,$x^{(2)}$},
	ytick={6.1, 2.2},
	yticklabels = {$y^{(0)}$, $y^{(1)}$},
	        ylabel = {$y$}, ylabel style={rotate=-90},
        xlabel = {$x$},
        every axis x label/.style={
    		at={(ticklabel* cs:1.0)},
   			anchor=west,
			},
		every axis y label/.style={
    		at={(ticklabel* cs:1.0)},
    		anchor=south,
			}
]


	\draw[rotate around={-55:(0,0)},black] (0,0) ellipse (8 and 12);
	\draw[rotate around={-55:(0,0)},black] (0,0) ellipse (4.7 and 9);
	\draw[rotate around={-55:(0,0)},black] (0,0) ellipse (3.6 and 5.7);
	\draw[rotate around={-55:(0,0)},black] (0,0) ellipse (1.9 and 2.9);
	
	\addplot [only marks,mark=*] coordinates { (0,0) };

    
	\coordinate[label=above:{$(x^{(0)}, y^{(0)})$}] (A) at (12, 6.1);
	
	\draw[fill] (A) circle (1pt);
	\draw [dashed] (A) -- (A |- 0,-10);
	\draw [dashed] (A) -- (A -| -13,0);
	
	\coordinate[label=above:{$(x^{(1)}, y^{(0)})$}] (B) at (5.9, 6.1);
	
	\draw[fill] (B) circle (1pt);
	\draw [dashed] (B) -- (B |- 0,-10);
%	\draw [dashed] (B) -- (B -| -13,0);
	
	\coordinate[label=right:{$(x^{(1)}, y^{(1)})$}] (C) at (5.9, 2.2);
	
	\draw[fill] (C) circle (1pt);
%	\draw [dashed] (C) -- (C |- 0,-10);
	\draw [dashed] (C) -- (C -| -13,0);
	
	\coordinate[label=above:{$(x^{(2)}, y^{(1)})$}] (D) at (1.6, 2.2);
	
	\draw[fill] (D) circle (1pt);
	\draw [dashed] (D) -- (D |- 0,-10);
	
	\draw [->, ultra thick] (A) -- (B);
	\draw [->, ultra thick] (B) -- (C);
	\draw [->, ultra thick] (C) -- (D);
	
\end{axis}
\end{tikzpicture}
	\caption{Координатный спуск.}
	\label{fig:coordinate}
\end{figure}



\begin{figure}[h] % picture
	\centering
	\begin{tikzpicture}
\begin{axis}[
	width = 15cm,
	xmin=-13,   xmax=13,
	ymin=-10,   ymax=10,
	axis y line*=left,
    axis x line*=bottom,
    axis lines = left,
    domain=0:30,
    xtick={7.9, 1.9, -1.1},
	xticklabels = {$x^{(0)}$, $x^{(1)}$,$x^{(2)}$},
	ytick={7, 1, 4},
	yticklabels = {$y^{(0)}$, $y^{(1)}$, $y^{(2)}$},
	        ylabel = {$y$}, ylabel style={rotate=-90},
        xlabel = {$x$},
        every axis x label/.style={
    		at={(ticklabel* cs:1.0)},
   			anchor=west,
			},
		every axis y label/.style={
    		at={(ticklabel* cs:1.0)},
    		anchor=south,
			}
]

	\coordinate (O) at (-0.7, 2.8);
	\coordinate (O1) at (-2.1, 3.1);

	\draw[rotate around={-45:(0,0)},black] (0,0) ellipse (8 and 10);
	\draw[rotate around={-45:(O)},black] (O) ellipse (3.2 and 4);
	\draw[rotate around={-45:(O1)},black] (O1) ellipse (1 and 1.25);
%	
%	\draw[fill] (0, 0) circle (1pt);
%
    
	\coordinate[label=right:{$(x^{(0)}, y^{(0)})$}] (A) at (7.9, 7);
		\draw[fill] (A) circle (1pt);
		
	\addplot[dashed] {-x + 14.9}; % perpendicular is x
	
	\coordinate[label=right:{$(x^{(1)}, y^{(1)})$}] (B) at (1.9, 1);
		\draw[fill] (B) circle (1pt);
		
  	\draw[dashed] (A) -- (-0.1, -1); 
  	
  	\coordinate[label=right:{$(x^{(2)}, y^{(2)})$}] (C) at (-1.1, 4);
		\draw[fill] (C) circle (1pt);
		
	\draw[dashed] (B) -- (-3.1, 6); 
		
  	\draw[->, ultra thick] (A) -- (B);
  	\draw[->, ultra thick] (B) -- (C);
  	
  	
  	\draw [dashed] (A) -- (A |- 0,-10);
	\draw [dashed] (A) -- (A -| -13,0);
	
	\draw [dashed] (B) -- (B |- 0,-10);
	\draw [dashed] (B) -- (B -| -13,0);
	
	\draw [dashed] (C) -- (C |- 0,-10);
	\draw [dashed] (C) -- (C -| -13,0);
  	

	
\end{axis}
\end{tikzpicture}
	\caption{Градиентный спуск.}
	\label{fig:gradient}
\end{figure}


\begin{figure}[h] % picture
	\centering
	\begin{tikzpicture}
	
	\begin{axis}[
	width = 15cm,
	xmin=-7,   xmax=7,
	ymin=-7,   ymax=7,
	axis y line*=left,
    axis x line*=bottom,
    axis lines = left,
    domain=0:30,
    grid = major,
    tick label style = {white},
		    ylabel = {$y$}, ylabel style={rotate=-90},
        xlabel = {$x$},
        every axis x label/.style={
    		at={(ticklabel* cs:1.0)},
   			anchor=west,
			},
		every axis y label/.style={
    		at={(ticklabel* cs:1.0)},
    		anchor=south,
			}
]
 \draw [ultra thick,black] (5, 5 ) to[out=-45,in=30] (4, -1)
 		to[out=-150,in=45] (2,-2) to[out=-135,in=30] (0, -5)
 		to[out=-150,in=-45] (-5, -5)  to[out=135,in=-150] (-4, 1) 
 		to[out=30,in=-135] (-2, 2)  to[out=45,in=-150] (0, 5)
 		to[out=30,in=135] (5, 5) ;
 \draw [ultra thick,black] (4, 4 ) to[out=-45,in=30] (3, -0.8)
 		to[out=-150,in=45] (1.5,-1.5) to[out=-135,in=30] (0, -4)
 		to[out=-150,in=-45] (-4, -4)  to[out=135,in=-150] (-3, 0.7) 
 		to[out=30,in=-135] (-1.5, 1.5)  to[out=45,in=-150] (0, 4)
 		to[out=30,in=135] (4, 4)  ;
 \draw [ultra thick,black] (3.3, 3.3 ) to[out=-45,in=30] (2.5, 0)
 		to[out=-150,in=0] (0, 0)  to[out=90,in=-150] (1, 3)
 		to[out=30,in=135] (3.3,3.3)  ;
  \draw [ultra thick,black] (-3.3, -3.3 ) to[out=135,in=-150] (-2, 0)
 		to[out=30,in=180] (0, 0)  to[out=-90,in=30] (-0.5, -2.7)
 		to[out=-150,in=-45] (-3.3, -3.3) ;
 \draw [ultra thick,black] (2.5, 2.5) to[out=-45,in=-45] (0.5, 0.5) to[out=135,in=135] (2.5,2.5 );
  \draw [ultra thick,black] (2, 2) to[out=-45,in=-45] (1, 1) to[out=135,in=135] (2,2 );
  \draw [ultra thick,black] (-2.5, -2.5) to[out=-45,in=-45] (-0.5, -0.5) to[out=135,in=135] (-2.5,-2.5 );
  \draw [ultra thick,black] (-2, -2) to[out=-45,in=-45] (-1, -1) to[out=135,in=135] (-2,-2 );
  
  \node (A) at (-5, 6) {$Q(x) = C_1$};
  \draw (-3.3, 6) -- (1, 5.6);
  \node (B) at (-5, 5) {$Q(x) = C_2 < C_1$};
  \draw (-3.3, 5) -- (1, 4.55);
  \node (C) at (-5, 4) {$Q(x) = C_3 < C_2$};
  \draw (-3.3, 4) -- (1, 3.0);
  \node (D) at (-5, 3) {$Q(x) = C_4 < C_3$};
  \draw (-3.3, 3) -- (1, 2.2);
  \node (E) at (-5, 2) {$Q(x) = C_5 < C_4$};
  \draw (-3.3, 2) -- (1, 1.5);
\end{axis}
		
	\end{tikzpicture}
	\caption{Сложный ландшафт.}
	\label{fig:complex_landscape}
\end{figure}





\begin{figure}[h] % picture
	\centering
	\begin{tikzpicture}
	
	\begin{axis}[
	width = 15cm,
	xmin=-5,   xmax=7,
	ymin=-2,   ymax=7,
	axis y line*=left,
    axis x line*=bottom,
    axis lines = left,
    domain=0:30,
    tick label style = {white},
    grid = major,
	    ylabel = {$y$}, ylabel style={rotate=-90},
        xlabel = {$x$}
        every axis x label/.style={
    		at={(ticklabel* cs:1.0)},
   			anchor=west,
			},
		every axis y label/.style={
    		at={(ticklabel* cs:1.0)},
    		anchor=south,
			}
]
 
 \draw [ultra thick,black] (-4, -1 ) to[out=-45,in=-150] (4, 1) to[out=30,in=-100] (6, 5)  to[out=80,in=30] (5, 6) to[out=-150,in=40] (2, 2) to[out=-140,in=10] (-4, 0) to[out=-170,in=135] (-4, -1)	 
 ;
 
 \draw [ultra thick,black] (-3, -1 ) to[out=-45,in=-150] (1, 0) to[out=30,in=-100] (5.5, 4.5)  to[out=80,in=60] (4.3, 4.5) to[out=-110,in=20] (1, 1) to[out=-160,in=5] (-3, -0.5) to[out=-175,in=135] (-3, -1)	 
 ;
		
\end{axis}
	\end{tikzpicture}
	\caption{Овраг}
	\label{fig:ovrag}
\end{figure}






 \newpage
        \section{Преобразование Фурье и его приложения}
\label{lecture9}

Чтобы подойти к определению преобразования Фурье сначала рассмотрим фукцию $f(t)$ с периодом $2\pi$.

\begin{equation}\label{9.1}
	f(t+2\pi) = f(t)
\end{equation}

Такая функция раскаладывается в ряд Фурье 

\begin{equation}\label{9.2}
f(t) = \frac{a_0}{2} + \sum_{n = 1}^{\infty}(a_n cos(nt) + b_n sin(nt)) 
\end{equation}

Для сходимости этого ряда достаточно предположить, что $f(t)$ -- непрерывная функция.
Коэффициенты Фурье $a_n$, $b_n$ задаются равенствами

\begin{equation}\label{9.3}
a_n = \frac{1}{\pi} \int_{0}^{2\pi} f(t) cos(nt) dt,\quad b_n = \frac{1}{\pi} \int_{0}^{2\pi} f(t) sin(nt) dt
\end{equation} 

Если $t$ -- время, то функцию $f(t)$ принято называть \textbf{сигналом}. 
Таким образом у нас есть взаимооднозначное соответствие $f(t) \rightarrow \{a_n, b_n\}$, и все свойства функции $f(t)$ имеют свое отражение в свойствах множества $\{a_n, b_n\}$.

В случае непериодических функций аналогом формулы \ref{9.2} является равенство 

\begin{equation}\label{9.4}
f(t) = \frac{1}{2\pi}\int_{-\infty}^{+\infty} e^{i\omega t} \hat{f} (\omega) d\omega,
\end{equation}
где функция $\hat{f}(\omega)$ называется \textbf{преобразованием Фурье} функции $f(t)$ ($ \hat{f} = \mathscr{F} f$)
о определяется равенством 

\begin{equation}\label{9.5}
\hat{f} (\omega) = \int_{-\infty}^{+\infty} e^{-i\omega t} f(t) dt,
\end{equation}
Величина $\hat{f} (\omega)$ во всяком случае существует, если 

\begin{equation}\label{9.6}
\int_{-\infty}^{+\infty} |f(t)| dt < \infty
\end{equation}

и в этом случае имеет место \textbf{формула обращения} \ref{9.4}  $(f =\mathscr{F}^{-1} \hat{f})$.
Функция $\hat{f}(\omega)$ часто называется \textbf{спектром сигнала} $f(t)$. 

Объясним происхождение формулы обращения и приведем ее эвристический вывод. Используя формулы Эйлера

\begin{equation}\label{9.7}
	\cos nt = \frac{1}{2}(e^{i n t} + e^{-i n t}), \quad 	\sin nt = \frac{1}{2i}(e^{i n t} - e^{-i n t}), 
\end{equation}
запишем формулу \ref{9.2} в виде

\begin{equation}\label{9.8}
f(t) = \sum_{n = -\infty}^{+\infty} c_n e^{int}
\end{equation}

\begin{equation} \label{9.9}
c_n = \frac{1}{2\pi} \int_{-\pi}^{+\pi} f(t) e^{-i n t} dt
\end{equation}

В формуле \ref{9.9}
\begin{equation}\label{9.10}
c_n = \frac{1}{2}(a_n - ib_n) \; n \geq 0, \quad c_{-n} =  \frac{1}{2}(a_n + ib_n). 
\end{equation}

Формула \ref{9.4} аналог формулы \ref{9.8}, а формула \ref{9.5} -- аналог \ref{9.9}.

Идея "вывода" формулы обращения \ref{9.4} состоит в том, чтобы рассматривать непериодические функции как пределы периодических при стремлении периода к бесконечности. 

Пусть функция $f(x)$ имеет период $2e$ и
\begin{equation}\label{9.11}
f(x+2e) = f(x)
\end{equation}

\begin{equation}\label{9.12}
g(t) = f\left(\frac{et}{\pi}\right) \quad x = \frac{et}{\pi}, \; t = \frac{\pi x}{e}
\end{equation}
имеет период $2\pi$ и 

\begin{equation}\label{9.13}
g(t+2\pi) = g(t)
\end{equation}

Разлагая $g(t)$ в ряд Фурье и переходя к переменной $x$, получим, что 

\begin{equation}\label{9.14}
f(x) = \sum_{n = -\infty}^{+\infty}c_n e^{i n \frac{\pi x}{e}}
\end{equation}

\begin{equation}\label{9.15}
c_n = \frac{1}{2e} \int_{-e}^{e} e^{-i n \frac{\pi y}{e}} f(y) dn
\end{equation}
	
и таким образом 
\begin{equation}\label{9.16}
f(x) = \frac{1}{2\pi}\sum_{n = -\infty}^{+\infty} \Delta\omega e^{in\Delta\omega ln x} \int_{-\infty}^{+\infty} e^{-in\Delta\omega y} f(y) dy , \Delta\omega=\frac{\pi}{e}	
\end{equation}

Так как

\begin{equation}\label{9.17}
\int_{-\infty}^{+\infty} e^{i\omega x} F(\omega)d\omega \simeq	\sum_{n = -\infty}^{+\infty} e^{in\Delta\omega x} F(n\Delta\omega)\Delta\omega y,
\end{equation}	

то при $e \rightarrow 0$ из \ref{9.6} следует, что 

\begin{equation}
f(x) = \frac{1}{2\pi}\int_{-\infty}^{+\infty} e^{i\omega x}d\omega \int_{-\infty}^{+\infty} e^{i\omega y}f(y)dn = \frac{1}{2\pi}\int_{-\infty}^{+\infty}e^{i\omega x} \hat{f}(\omega)d\omega
\end{equation}
Заменяя $х$ на $t$ получаем формулу обращения \ref{9.4}.

Рассмотрим некоторые свойства преобразования Фурье. Найдем, например, преобразование Фурье $\hat f^{(1)} (\omega)$ производной $f^{(1)}(t)$

\begin{equation}\label{9.19}
f^{(1)}(t) = \frac{d}{dt} \frac{1}{2\pi} \int_{-\infty}^{+\infty} \hat{f^{(1)}}(\omega) e^{-i\omega t}d\omega
\end{equation}
С другой стороны 
\begin{equation}\label{9.20}
f^{(1)}(t) = \frac{d}{dt} \frac{1}{2\pi} \int_{-\infty}^{+\infty} e^{-i\omega t} \hat{f}(\omega)d\omega = \frac{1}{2\pi} \int_{-\infty}^{+\infty}(-i\omega)\hat{f}(\omega) e^{-i\omega t}d\omega
\end{equation}

Таким образом
\begin{equation}\label{9.21}
\hat{f^{(1)}}(\omega) = -i\omega \hat{f}(\omega)
\end{equation}
и на языке преобразований Фурье дифференцирование -- это умножение $\hat{f}(\omega)$ на $(-i\omega)$.

Это позволяет, например, решить дифференциальное уравнение вида

\begin{equation}\label{9.22}
f^{(2)}(t) + a f^{(1)}(t) + bf(t) = g(t)
\end{equation}

Беря преобразование Фурье от обеих частей этого равенства, получим, что 
\begin{equation}\label{9.23}
\begin{gathered}
(-i\omega)^2 \hat{f}(\omega) + a(-i\omega) \hat{f}(\omega)+b\hat{f}(\omega) = g(\omega) \\
\hat{f}(\omega) = \frac{\hat{g}(\omega)}{-\omega^2-i\omega a + b} \; \text{и} \; f(t)= \mathscr{F}^{-1} (\hat{f}(\omega)).
\end{gathered}
\end{equation}

Будем говорить, что $f(t)$ имеет эффективный носитель $L(f) \sim a$, если при $|x|>\alpha a $,  $\;\alpha >> 1 \;$ $f(x)$ достаточно быстро убывает, например как $e^{-|x|}$. Оказывается, что 

\begin{equation}\label{9.24}
L(f) L(\hat{f}) \sim 1.
\end{equation}

Это соотношение называется \textbf{"соотношением неопределенностей"}. Оно показывает с какой точностью можно одновременно локализовать $f(x)$ и $\hat{f}(\omega)$. Знаменитое соотношение неопределенности $\Delta p \Delta x \sim h$ в квантовой механикеявляется следствием \ref{9.24}. Рассмотрим пример

\begin{equation}\label{9.25}
\begin{gathered}
f(t)=e^{-a^2t^2}, L(f) \sim \frac{1}{a},\\
\hat{f}(\omega)=\frac{\sqrt{n}}{a} e^{-\frac{\omega^2}{na^2}}, L(\hat{f}) \sim a,
\end{gathered}
\end{equation}

что и доказывает \ref{9.24} для рассматриваеой функции $f(t)$.

В теории преобразований Фурье большую роль играет понятие свертки $f_1*f_2 $ двух функций. 
По определению 

\begin{equation}\label{9.26}
(f_1*f_2)(t) = \int_{-\infty}^{+\infty}f_1(g)f_2(t-g)dg = (f_2*f_1)(t).
\end{equation}

Значение этой операции объясняется тем, что

\begin{equation}\label{9.27}
\hat{(f_1*f_2)}(\omega) = \hat{f_1}(\omega)\hat{f_2}(\omega)
\end{equation}

Полезность преобразования Фурье в задачах обработки экспериментальных данных  проиллюстрируем двумя примерами.
В качестве первого примера рассмотрим \textbf{задачу сглаживания}.

На рисХ представлен результат сглаживания $R$ четной функции $f(t)$.

Следующая последовательность отображений описывает алгоритм сглаживания $R$.

\begin{math}
\begin{gathered}
f(t) \xrightarrow{\mathscr{F}} \hat{f}(\omega) \xrightarrow{R} \hat{f}(\omega)R(\omega)  \xrightarrow{\mathscr{F}^{-1}} f_R(t) =(f\otimes H_R)(t) \\ 
\hat{H_R}(\omega)=R(\omega), \quad R(\omega)=R(-\omega) 
\end{gathered}
\end{math}


Сглаживание (фильтр высоких частот) задается функцией $R(\omega)$, вид которой представлен на рисункеУ.

Так как спектр шума (случайных ошибок) лежит в высоких частотах, то сглаживание используется и для \textbf{фильтрации} шума.

В качестве второго примера рассмотрим задачу обнаружения малого сигнала $\Delta f(t)$, содержащего высокочастотную компоненту, на фоне большого низкочасотного сигнала $f_0(t)$.

Пусть
\begin{equation}\label{9.28}
\begin{gathered}
f(t)=f_0(t)+\Delta f(t) \\
f_0(t)=e^{-a^2t^2}, \Delta f(t)= \varepsilon cos \omega_0 t e^{-b^2t^2} \quad \varepsilon << 1
\end{gathered}
\end{equation}

При малых $\varepsilon$ сигнал $\Delta f(t)$ "плохо видим" на фоне $f_0(t)$. Переходим к преобразованию Фурье. Используя \ref{9.25} имеем


\begin{equation}\label{9.29}
\begin{gathered}
\hat{f}_0(\omega) \sim a^{-1} e^{\frac{-\omega^2}{4a^2}} \\
\Delta\hat{f}(\omega) \sim \frac{\varepsilon}{b} \left(e^{-\frac{(\omega-\omega_0)^2}{4b^2}} + e^{-\frac{(\omega+\omega_0)^2}{4b^2}} \right)
\end{gathered}
\end{equation}

Величина $\Delta\hat{f}(\omega_0)$ имеет максимум при $\omega = \omega_0$ и 
\begin{equation}\label{9.30}
\Delta\hat{f}(\omega_0) \sim \varepsilon b^{-1}
\end{equation}

С другой стороны  

\begin{equation}\label{9.31}
\Delta\hat{f}(\omega_0) \sim a^{-1} e^{\frac{-\omega_0^2}{4a^2}}
\end{equation}
и если

\begin{equation}\label{32}
\varepsilon b^{-1} >> a^{-1} e^{\frac{-\omega_0^2}{4a^2}},
\end{equation}

то в $\omega$-области сигнал $\Delta f(t)$ "хорошо видим".

Выше рассматривались функции $f(t)$, зависящие от одной переменной.
Формулы \ref{9.4}, \ref{9.5} имеют многомерное обобщение.

Если $f(x) = f(x_1...x_n)$, то
\begin{equation}\label{33}
\hat{f}(\omega) = \int e^{-i(\omega_1 x)}f(x)dx, \quad \omega = (\omega_1 ... \omega_n)
\end{equation}

Вэтой формуле $dx = dx_1...dx_n$
\begin{equation}
\begin{gathered}
\int g(x)dx \equiv  \underbrace{\int_{-\infty}^{+\infty} ...  \int_{-\infty}^{+\infty}}_{n} g(x_1...x_n)dx_1...dx_n\\
(\omega_1 x) = \sum_{i = 1}^{n} \omega_i x_i
\end{gathered}
\end{equation}

Имеет место и аналог формулы обращения
\begin{equation}\label{35}
f(x) = \frac{1}{(2\pi)^n} \int e^{i(\omega_1 x)} \hat{f}(\omega) d\omega
\end{equation}

Двумерное преобразование Фурье используется при обработке изображений. \newpage
        \section{Дискретное преобразование Фурье (ДПФ)}
\label{lecture10}

В Лекции \ref{lecture9} было введено преобразование Фурье $f(t) \xrightarrow{\mathscr{F}} \hat{f}(\omega)$. На практике, как правило, известно только конечное число ее отсчетов $f(t_n),\quad n = 0 \; ...\;N-1$. \textbf{Дискретное преобразование Фурье} $\hat{f}^D(\omega)$ служит заменой $\hat{f}(\omega)$.

Определим ДПФ $\hat{f}^D(\omega)$. Пусть

\begin{equation}\label{10.1}
f(t) = 
\begin{cases}
0, t < 0 \\
0, t > T.
\end{cases}
\end{equation}
Тогда
\begin{equation}\label{10.2}
\hat{f}\omega = \int_{0}^{T}e^{-i\omega t}f(t)dt.
\end{equation}
Будем предполагать, что известны величины
\begin{equation}\label{10.3}
\begin{gathered}
f_n = f(n \Delta t), \quad n = 0 ... N-1 \\
\Delta t = \frac{T}{N}.
\end{gathered}
\end{equation}

ДПФ $\hat{f}^D(\omega)$ определяется формулой
\begin{equation}\label{10.4}
\hat{f}^D(\omega) = \sum_{n = 0}^{N - 1} e^{-i \omega nt}f_n\Delta t.
\end{equation}

Это просто результат замены интеграла в уравнении \ref{10.2} интегральной суммой.

Нас интересует конечное число значений $\hat{f}^D\omega$, то есть величины 

\begin{equation}\label{10.5}
A_j = \hat{f}^D(j \Delta \omega), \quad j = 0, 1 ... N.
\end{equation}

Возникает вопрос: как выбрать значение $\Delta \omega$.
В отличие от $ \hat{f}(\omega)$, $\hat{f}^D(\omega)$ периодическая функция с периодом $\omega$

\begin{equation}\label{10.6}
\hat{f}^D(\omega + \omega) = \hat{f}^D(\omega), \quad \omega = \frac{2\pi}{\Delta t}.
\end{equation}

Поэтому естественно выбрать $\Delta \omega$ из условия
\begin{equation}\label{10.7}
\Delta \omega = \frac{\omega}{N} = \frac{2\pi}{T} \text{ и } \Delta t \Delta \omega = \frac{2\pi}{N}.
\end{equation}

В этом случае
\begin{equation}\label{10.8}
A_j = \frac{1}{N}\sum_{n = 0}^{N - 1} e^{-i \frac{2\pi n}{N}} a_n, \quad i = 0 ... N - 1,
\end{equation}

и в этой формуле
\begin{equation}\label{10.9}
a_n = f_n T, \quad n = 0 ... N-1.
\end{equation}

В общем случае (отвлекаясь от формулы \ref{10.9}) говорят, что $\{A_j\}$ -- это \textbf{ДФП последовательности} $\{a_n\}$, и 

\begin{equation}\label{10.10}
\{a_n\} \xrightarrow{\mathscr{F}_D} \{A_j\}.
\end{equation}

Основанием для такого определения служит \textbf{формула обращения}

\begin{equation}\label{10.11}
 \{A_j\}\xrightarrow{\mathscr{F}^{-1}_D} \{a_n\} ,
\end{equation}
определяемая равенством

\begin{equation}\label{10.12}
a_n = \sum_{i = 0}^{N - 1} e^{i \frac{2\pi n}{N}} j A_j, \quad i = 0 ... N - 1.
\end{equation}

Докажем эту формулу. Если в \ref{10.12} подставить выражения для $A_i$ из \ref{10.8}, то должно выполняться равенство
\begin{equation}\label{10.13}
a_n = \sum_{k = 0}^{N - 1} \frac{a_k}{N} \sum_{i = 0}^{N - 1} e^{i \frac{2\pi }{N}}(n-k) j,
\end{equation}

которое действительно выполняется, так как

\begin{equation}\label{10.14}
\sum_{i = 0}^{N - 1} e^{i \frac{2\pi }{N}}(n-k) j = N\delta_{n,k}.
\end{equation}

Это равенство -- следствие формулы для суммы геометрической прогрессии
\begin{equation}\label{10.15}
\sum_{i = 0}^{N - 1} x^i = \frac{1-x^N}{1-x}.
\end{equation}

Мы определили ДПФ $\hat{f}^D(\omega)$, но нас интересует $\hat{f}(\omega)$. Рассмотрим вопрос о связи этих двух функций. Для достаточно гладкой функции $f(t)$  $\hat{f}(\omega)$ достаточно быстро убывает при $|\omega| \rightarrow \infty $. Это следует из того, что (см. \ref{9.21})

\begin{equation}\label{10.16}
f^{(k)}(t) \equiv \frac{d^k}{dt^n}f(t) = \int_{-\infty}^{+\infty} (-i\omega)^k \hat{f}(\omega)d\omega,
\end{equation}

и, следовательно,

\begin{equation}\label{10.17}
\int_{-\infty}^{+\infty}\omega^k \hat{f}(\omega)d\omega < \infty.
\end{equation}

Таким образом, определена функция

\begin{equation}\label{10.18}
\begin{gathered}
F(\omega) = \sum_{n = -\infty}^{+\infty} \hat{f}(\omega+ n\omega), \\
F(\omega+\omega) = f(\omega),
\end{gathered}
\end{equation}

имеющая, как и $\hat{f}^D(\omega)$, период $\omega$. 
Оказывается, что эта функция совпадает с $\hat{f}^D(\omega)$, то есть

\begin{equation}\label{10.19}
\hat{f}^D(\omega) = \sum_{n = -\infty}^{+\infty} \hat{f}(\omega+ n\omega).
\end{equation}

Эта формула называется формулой Пуассона.

Так как $\hat{f}(\omega)$ достаточно быстро убывает при $|\omega| \rightarrow \infty $, то можно предположить что, если $\Delta t$ достаточно мало, то

\begin{equation}\label{10.20}
\hat{f}^D(\omega) \simeq \hat{f}(\omega) \quad |\omega| < \frac{\omega}{2}.
\end{equation}

Интервал $|\omega| < \frac{\omega}{2}$ называется \textbf{интервалом Найквиста}.

Картина приближения функции $\hat{f}(\omega)$ с помощью $\hat{f}^D(\omega)$ при достаточно малом значении $\Delta t$ представлена на рисунке Х.

Если величина $\Delta t$ не достаточно мала, то возникает \textbf{эффект наложения} при котором вклад высоких частот искажает информацию о поведении $\hat{f}(\omega)$ при низких частотах (См. Рис 11).

Этот эффект все видели в кино, когда на экране колеса автомобиля вращаются в обратную сторону.

В пакетах прикладных программ прямое и обратное ДПФ реализуются с помощью алгоритма быстрого преобразования Фурье (БПФ). Если прямо вычислять $N$ значений $A_j$ по $N$ значениям $a_n$, то требуется $\sim N^2$ операций. Оказывается, что по крайней мере при $N = 2^k$ это можно сделать на $N \mathrm{ln}N$ операций. БПФ реализует такой алгоритм (алгоритм Кули-Тьюки). Алгоритм БПФ открыл путь к практическому применению ДПФ при больших $N$. \newpage
        
        \footnotesize \bibliographystyle{mybib5.bst}  
\end{document}