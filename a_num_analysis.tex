%-------------------BASICS-------------------
\documentclass[oneside,final,12pt]{article} %одностороння печать, чистовая версия, размер кегля, класс документа
\usepackage{ucs} 
\usepackage[utf8x]{inputenc} 
\usepackage[T2A, T1]{fontenc} 
\usepackage[english,russian]{babel} %оформление кириллицей (подписи к таблицам и т.д.)
\usepackage{ifpdf}

\ifpdf  %% если используется pdfTEX
\usepackage{cmap}  %поиск по кириллице в готовом pdf
\usepackage[pdftex]{graphicx} %работа с графикой 
\usepackage[unicode=true]{hyperref}
\usepackage{pdfpages}
\else   %% если используется не pdfTEX
\usepackage[dvips]{graphicx}
\fi

\setcounter{tocdepth}{2} % глубина содержания

\usepackage{cite}




%-------------------FORMAT-------------------
\usepackage{vmargin} %размеры полос набора
\setpapersize{A4} %формат бумаги
\setmarginsrb{30mm}{25mm}{25mm}{25mm}{0pt}{0mm}{0pt}{13mm} %размеры полей: левое, верхнее, правое, нижнее, 3*колонтитулы, расстояние между нижним краем нижней строки и нижним краем номера страницы
\usepackage{indentfirst} %красная строка для первого абзаца главы или параграфа
\setlength{\parindent}{0cm} % отступ красной строки
\setlength{\parskip}{2mm}
\sloppy %борьба с залезанием строк на поля путём изменения размеров пробелов
\pagestyle{plain} %включена нумерация страниц 
\renewcommand{\thesection}{\arabic{section}} %арабская нумерация глав
%\usepackage{lineno} %нумерация всех строк для отладки
%\usepackage{textgreek} % \textalpha греческие буквы не в math mode
\usepackage{titlesec}
\titleformat{\section}{\normalfont\large\bfseries}{Лекция  \thesection}{1em}{} %Лекция "ее номер"     Название
%\usepackage{verbatim} %comments
\usepackage{setspace} %for setstretch


%-------------------TABLES, FIGURES-------------------

\usepackage{float} %для плавающих картинок и таблиц
%\usepackage{wrapfig} %для плавающих картинок
\usepackage[font=small]{caption}

%\usepackage{booktabs} %отступы в tabular
%\usepackage{colortbl} %раскаршивание таблиц
%\usepackage{xcolor} %название цветов
%\usepackage{longtable}% перенос таблиц на страницах
%\definecolor{dark-gray}{gray}{0.4} %определение цвета
%\newcommand{\graytable}[0]{\arrayrulecolor{dark-gray}} % сделать таблицу серой
%\newcommand{\thinrule}[0]{\specialrule{0.3pt}{4pt}{4pt}} % тоненькая линия
%\newcommand{\verythinrule}[0]{\specialrule{0.1pt}{1pt}{1pt}} %очень тоненькая линия если надо
%\newcommand{\invisiblerule}[0]{\specialrule{0pt}{2pt}{2pt}} %просто создать пустого места в таблице

\usepackage{subcaption} % несколько картинок в одной
\usepackage{multirow} % объединение ячеек
\usepackage{enumitem} %особенности enumerate
\usepackage{pbox} % для переносов внутри ячейки таблицы
\usepackage{array} 

% рисование графиков
\usepackage{tikz}
\usepackage{pgfplots}
\pgfplotsset{compat=newest}
\usepackage{rotating}
\usetikzlibrary{calc, intersections}
\usetikzlibrary{arrows.meta}






%-------------------MATH-------------------

\usepackage{amsmath} %дополнительные средства для вёрстки формул
\everymath{\displaystyle}
\usepackage{breqn} %для dmath разбить длинную формулу
%\usepackage{esvect} %vectors
%\usepackage{amscd} %диаграммы
\usepackage{amsfonts} %дополнительные шрифты для формул
\usepackage{amssymb} %дополнительные символы для формул
\usepackage{amsfonts} %для букв с двойными штрихами
\usepackage{mathrsfs} % еще шрифт
\numberwithin{equation}{section} % нумерация формул внутри главы
\usepackage{mathtools} % больгие нули в матрице


%-------------------BIBLIOGRAPHY-------------------
%\usepackage[nottoc,notlot,notlof]{tocbibind}









\begin{document}
		\begin{titlepage}

\newcommand{\HRule}{\rule{\linewidth}{0.3mm}} % Defines a new command for the horizontal lines, change thickness here

\center

\textbf{\textsc{\Large Московский государственный университет} \textsc{\large имени }\textsc{\Large М.В.Ломоносова}}
\\[0.3cm] 
\HRule 
\\[0.3cm]
\textbf{\textsc{\large Факультет биоинженерии и биоинформатики}}
\\[4.0cm]

\begin{spacing}{1.4}
{ \LARGE \bfseries Численные методы в задачах обработки данных} \\[1.0cm]

\end{spacing}
 
 
\Large \emph{}\\
Курс лекций \\
Лектор - Попов Дмитрий Александрович
\\[4cm]

\begin{abstract}
	Курс включает обзор основных численных методов, применяемых при обработке экспериментальных данных. Основные темы курса: полиномиальная аппроксимация, интерполяция, сплайны, численное дифференцирование и интегрирование, метод наименьших квадратов, методы решения систем линейных уравнений, решение нелинейных уравнений и оптимизация, анализ Фурье.
	
	
	\footnotesize{Исходные материалы \href{https://github.com/litvinanna/num_analysis}{github.com/litvinanna/num_analysis}. }
\end{abstract}


\vfill

{\large Москва \\ \today}


\end{titlepage}


        \tableofcontents \newpage
        \section{Введение}

В этой вводной лекции рассматривается вопрос о том, на каком этапе и какие численные методы возникают при обработке экспериментальных данных (ЭД).

Начнём с конкретного примера. В 1881 году Д.И.\,Менделеев исследовал зависимость растворимости $NaNO_{2}$ в воде от температуры. Были получены следующие данные:

\begin{table}[h]
	\small
	\caption{Зависимость растворимости соли от температуры в эксперименте Менделеева.}
	\label{table:mendeleev}
	\begin{tabular}{ p{0.25\textwidth} *{10}{|c} }

		T (температура), °C & 0 & 4 & 10 & 15 & 21& 29& 36& 51&68 & x \\
		\hline
		Масса $NaNO_{2}$ в 100~мл воды, г&
		66,7&71,0&76,3&80,6&85,7&99,4&99,4&113,6&125,1& y \\

	\end{tabular}
	
\end{table}

Это типичный пример исходных экспериментальных данных -- конечная таблица $x_i | y_i, i=1\dots n$. Задача состоит в том, чтобы найти зависимость $y = f(x)$ по этим данным. В рассматриваемом примере предполагается, что $f(x)$ неизвестна и предполагается, что $y_i = f(x_i) + \varepsilon _i$, где $\varepsilon _ i$ -- ошибки.

Что касается ошибок, то мы будем предполагать, что известна величина $\varepsilon$ такая, что с нужной вероятностью $|\varepsilon _i| \leq \varepsilon$ и $\varepsilon$ много меньше наблюдаемых значений ($\varepsilon \ll y_i$). В частности, если $\varepsilon_i$ независимые, одинаково распределенные случайные величины с нулевым средним $\langle \varepsilon_i \rangle = 0$ и дисперсией $\langle \varepsilon^2 \rangle$, то $\varepsilon \sim \sqrt{\langle \varepsilon^2 \rangle }$. В более общем случае $\varepsilon$ характеризуется величиной доверительного интервала. 

Таким образом, предполагается, что из априорных соображений или путем статистической обработки мы определили величину $\varepsilon$. Это всё, что нам нужно от статистики, и вопросы статистической обработки ЭД в лекциях рассматриваться не будут.



Если представить данные Д.И.\,Менделеева графически, то возникает картина, схематически представленная на рисунке . 
Эта картина позволяет предположить, что исходная зависимость имеет вид 
\begin{equation}
	y = f(x) = ax + b.
\end{equation}
 Возникает вопрос -- как определить величины $a$ и $b$?  Правильная (с точки зрения математической статистики) стратегия состоит в применении метода наименьших квадратов (МНК). Согласно этому методу величины $a$ и $b$ ищутся  из условия 
 \begin{equation}
	Q(a, b) = \sum^n_{i=1}{(a+bx_i -y_i)^2} = \min_{a,b}
\end{equation}

Так как величина $Q(a,b)$ квадратична по $a,b$, то условия
\begin{equation}
	\frac{\partial Q}{\partial a} = 0,    \frac{\partial Q}{\partial b} = 0
\end{equation}
приводят к системе из двух линейных уравнений. Их решение $a = 67,5, b = 0,87$ нашел Д.И.\,Менделеев. Необходимо ещё проверить, что для функции $f(x) = 67,5 + 0,87x$ выполняются условия $|f(x_i) - y_i| < \varepsilon$ и убедиться в том, что величины $a,b$ мало меняются при замене $y_i$ на $y_i = \varepsilon _i$. Если это так, то задача решена и при этом никаких численных методов нам не понадобилось. Это связано с тем, что зависимость простая (всего два параметра) и число измерений невелико. В случае более сложной зависимости возникает картина, представленная на следующем рисунке .

В этом случае можно попытаться искать зависимость в виде 
\begin{equation} \label{eq:representation}
	f(x) = a_1 \varphi _1(x) + a_2 \varphi _2(x) + \dots + a_m \varphi _1(m),
\end{equation}
где $\varphi _k(x)$ --известные функции, например
\begin{equation}
	\varphi _k(x) = x^{k-1},  k = 1 \dots m
\end{equation}

Согласно МНК величины $a_1 \dots a_m$ ищутся из условия

 \begin{equation}
	Q(a_1 \dots a_m) = \sum^n_{i=1}{(f(x_i) -y_i)^2} = \min_{a_1 \dots a_m}
\end{equation}

и определяются исходя из полученной системы из $m$ линейных уравнений.

Возникают следующие вопросы:
\begin{enumerate}
	\item Как выбрать функции $\varphi _k(x)$ и величины $m$;
	\item Как решать полученную систему линейных уравнений;
	\item Как определить устойчиво ли решение относительно малых изменений величин $y_i$.
\end{enumerate}


Выбирая набор функций $\varphi _k(x)$ мы должны быть уверенны, что любую зависимость можно с нужной точностью представить в виде \ref{eq:representation}. Эта задача решается в теории приближений (теории аппроксимаций) и ей посвящены 2, 3, 4 лекции. Вопросы 2,3 -- это вопросы линейной алгебры и они будут рассмотрены в 5, 6, 7, лекциях.

Выше предполагалось, что вид функции $f(x)$ неизвестен. В ряде случаев вид функции  $f(x)$ известен с точностью до конечного числа неизвестных параметров.

Рассмотрим эксперимент, в котором измеряется зависимость от времени концентрации $C(t)$ некоторого вещества, при это известно, что 
\begin{equation}
	C(t) = a_1e^{\lambda_1 t} + a_2 e^{\lambda_2 t},
\end{equation}
экспериментальные данные это таблица $x_i | y_i, i=1\dots n, x_i = t_i, y_i = C(t_i) + \varepsilon_i$. Если величины $\lambda_1, \lambda_2$ неизвестны, задача их определения из условия

 \begin{equation}
	Q(a_1,a_2,\lambda_1, \lambda_2) = \sum^n_{i=1}{(C(t_i) -y_i)^2} = \min_{a_1,a_2,\lambda_1, \lambda_2}
\end{equation}
приводит к системе нелинейных уравнений. Это задача из теории оптимизации. Здесь, по существу, идёт речь о методах решения нелинейных систем уравнений, которые будут рассмотрены в лекциях 9 и 10.

Последние две лекции будут посвящены использования преобразования Фурье в задачах обработки ЭД. Будет объяснено, что такое преобразование Фурье и его дискретные варианты (ДПФ), а также как ДПФ используется в задачах "сглаживания" ЭД и обнаружения периодических сигналов.

Чтобы ориентироваться в численных методах и использовать пакеты прикладных программ, необходимо знакомство с такими понятиями как норма функций, число обусловленности матрицы и её норма, QR и SVD разложение, неподвижная точка отображения и т.д. Все эти понятия будут введены ниже по мере их возникновения в рассматриваемых задачах.

В каждой лекции используется своя нумерация формул. При ссылках на формулы впереди указывается их номер ( 3.5 = формула 5 из лекции 3).



        \section{Приближение функции полиномами}

\label{lecture:2}

В первой лекции было отмечено, что для описания зависимостей полезно иметь систему функций $\varphi _i(x)$ такую, что их значения легко вычисляются и любую непрерывную функцию $f: I \rightarrow \mathbb{R} $ заданную на интервале $ I = [a,b] $ можно с заданной точностью приблизить функциями вида 
\begin{equation}
\varPsi_n(x)=\sum_{i=0}^{n-1}{a_i\varphi_i(x)} \qquad i = 1, 2, 3, ...
\end{equation}

В качестве $\{\varphi_i(x)\}$ можно выбирать различные системы функций, но наиболее важен с практической точки зрения случай 


\begin{equation}
\varphi _i(x) = x^i \qquad i = 0, 1, 2, ...
\end{equation}

Тогда
\begin{equation}\label{eq:polynom}
\varPsi _n(x) = p_n(x) = a_0 + a_1x + a_{n-1}x^{n-1} 
\end{equation}


Полиномы $p_n(x)$ образуют n-мерное \textbf{линейное пространство $\mathscr{P}_n$}, и ниже речь идет о приближении функций  $f:I \rightarrow \mathbb{R} $ полиномами $p_n(x) \in \mathscr{P}_n $. Подчеркнем, что максимальная степень полинома из пространства  $\mathscr{P}_n$ равна $n - 1$.


Определение точности приближения требует введения понятия нормы $\parallel f \parallel$ функции $f$. 
Функции $f: I \rightarrow \mathbb{R} $ образуют линейное пространство $V[f]$, размерность которого бесконечна. Это означает, что существует функции $e_i(x) \; i = 1, 2, 3 ...$ такие, что любая функция $f \in V[f]$ может быть единственным образом представлена в виде бесконечного ряда 
\begin{equation}
f(x) = \sum_{i=0}^{\infty} {c_i e_i(x)}
\end{equation}
В этом случае говорят, что $\{e_i(x)\}$ \textbf{базис} в пространстве $V[f]$.

\textbf{Нормой} в любом векторном пространстве $V$ называется любая функция 
 
\begin{equation}
\parallel \; \parallel \; : V \rightarrow \mathbb{R}^+
\end{equation}
такая что, если $X\in V$, то

\begin{equation}
\begin{array}{l}
\parallel X \parallel \; \geq 0 \; \text{и}  \parallel X \parallel \; = 0 \Rightarrow X = 0 \\
\parallel \alpha X \parallel \; = \alpha \parallel X \parallel  \\ 
\parallel \alpha X + \beta Y\parallel \; \leq \; \parallel \alpha  X \parallel + \parallel \beta Y \parallel \text{(неравенство треугольника)}
\end{array}
\end{equation}

Если $V = \mathbb{R}^n$ (n-мерное вектроное пространство) и $X = (x_1 .. x_n)$, то обычная эвклидова норма $\parallel\;\parallel_2$ (длина вектора $X$) задается равенством
\begin{equation}
\parallel X\parallel_2 \; = \left[\sum_{i=1}^{n} x_i^2 \right]^{1/2}.
\end{equation}

Но в $\mathbb{R}^n$ существуют и другие нормы, например,

\begin{equation}
\begin{gathered}
\parallel X \parallel_1 \; = \sum_{i = 1}^{n}|x_i| \\
\parallel X \parallel_c \; = \max_{i = 1...n}|x_i|
\end{gathered}
\end{equation}

Эти формулы прямо приводят к определению нормы в пространстве функции $f: I \rightarrow \mathbb{R}$ и по определению

\begin{equation}
\begin{gathered}
\parallel f \parallel_2 \; = \left[ \int_{a}^{b} f^2(x)dx \right]^{1/2} \quad \text{($L^2$-норма)}\\
\parallel f \parallel_1 \; = \int_{a}^{b} |f(x)|dx \quad \text{($L^1$-норма)} \\
\parallel f \parallel_c \; = \max_{x \in I}|f(x)| \quad \text{(равномерная норма)}
\end{gathered}
\end{equation} 

С точки зрения приложений наибольший интерес представляет равномерная норма $\parallel \; \parallel_c	$ и точность приближения функции $f$ полиномом $p_n$ задается числом 
\begin{equation}
\parallel f - p_n \parallel_c \; = \max_{x\in [a, b]} |f(x)-p_n(x)|
\end{equation}


Теорема Вейерштрасса утверждает, что любую непрерывную функцию $f: I \rightarrow \mathbb{R}$  с любой заданной точностью $\varepsilon$ можно приблизить полиномом $p_n(x)$ $(n = n(\varepsilon))$. 
Эта теорема говорит только о существовании такого полинома $p_n$, но ничего не говорит о том, как его построить и какова завиимость $n$ от $\varepsilon$. 
Из теоремы Вейерштрасса следует, что $\{x^i\} \; (i = 0, 1, 2, ...)$ базис в пространстве $V[f]$ непрерывных функций $f$ на интервале $I$.

Гораздо интереснее следующая постановка задачи: с какой точностью заданная функция $f$ может быть приближена полиномом заданной степени. 
Оказывается, что среди полиномов степени $n - 1$ существует единственный \textbf{полином $p_n[f]$ наилучшeго приближения}. Если

\begin{equation}\label{eq:best_polynom}
\parallel f - p_n(f) \parallel_c \; = \varepsilon_n[f],
\end{equation}

то для любого полинома $p$ степени $\leq n - 1$

\begin{equation}
\parallel f - p \parallel_c \; > \varepsilon_n[f],
\end{equation}

Это тоже теорема существования, в которой ничего не говорится о том, как найти $p_n(f)$ и  $\varepsilon_n[f]$.

Построение системы полиномов $p_n(f)$ для заданной функции $f$ -- очень трудная задача, точное решение которой при всех $n$ известно только для некоторых функций $f$. 
Однако существует алгоритм посторения $p_n(f)$ для заданного n (например, алгоритм Ремеза).

Важно то, что для величины $\varepsilon_n(f)$ существуют хорошие оценки сверху и для этого не надо точно знать функцию $f$ и достаточно предположиить, что $f \in W^k(M_k, I)$. Это означает, что функция $f$ на интервале $I$ имеет $k$ непрерывных производных и $|f^{(k)}(x)|\leq M_k$. 

Таким образом,

\begin{eqnarray}
W^k[M_k, I] = \{f: I \rightarrow \quad , \text{ -- производные} \nonumber \\
f^{(i)}(x) \text{ непрерывны при $i \leq k$ и } |f^{(k)}(x)| \leq M_k \}
\end{eqnarray}


\textbf{Класс функций $W^k[M_k, I]$} удобен для характеристики точности приближений, так как для любой функции $f \in W^k[M_k, I]$ имеет место достаточно точная оценка

\begin{equation}
\varepsilon_n[f] \leq \left(\frac{b-a}{2}\right)^k \frac{A_k M_k}{n^k},\; A_k = \left(\frac{\pi k}{2}\right) ^k \frac{1}{k!}
\end{equation}

При $k \geq 2$ с помощью формулы Стирлинга получим, что 

\begin{equation}
\varepsilon_n[f] \leq \left(\frac{(b-a)4,3}{n}\right)^k M_k \quad \forall f \in W^k[M_k, I].
\end{equation}

Эта формула позволяет оценить, с какой точностью функция $f \in W^k[M_k, I]$ может быть приближена полиномами степени $\leq n-1$, но мы по-прежнему не имеем метода построения приближений.
Такой метод дает \textbf{интерполяция}.

Рассмотрим \textbf{разбиение} $X = (x_1, x_2,..., x_n)$ интервала $I = [a, b]$. Это означает просто, что заданы $n$ различных точек $x_i \in I, \; i = 1 ... n$ и $x_i \neq x_j$ и не предполагается, что $x_{i+1} > x_i$.

\textbf{Интерполяционный полином} $\pi(f|x) \in \mathscr{P}_n $ зависит от $n$-параметров -- коэффициентов $a_0, a_1 .. a_{n-1}$ \ref{eq:polynom}, которые определяются из условий

\begin{equation}\label{eq:interpolation}
\pi(f|x)(x_j) = f(x_j) \quad j = 1 .. n
\end{equation}

Это $n$ условий для определения $n$ коэффициентов  $a_0, a_1 .. a_{n-1}$ полинома

\begin{equation}
\pi(f|x)(x) = a_0 + a_1x + a_{n-1}x^{x-1}.
\end{equation}
Отсюда следует, что существует только один такой полином и этот полином можно выписать явно. 

Действительно, пусть известны полиномы $e^i(x_i) \in \mathscr{P}_n $, такие что

\begin{equation}
e^i(x_i)= \delta_{ij} =
\begin{cases}
1, i = j \\
0, i \neq j

\end{cases}
\end{equation}

Тогда 
\begin{equation}
\pi_n(f|x)(x) = \sum_{i = 1}^{n} f(x_i)e^i(x).
\end{equation}

Условие интерполяции \ref{eq:interpolation} выполнено так как 

\begin{equation}
\pi_n(f|x)(x_j) = \sum_{i = 1}^{n} f(x_i)\delta_{ij} = f(x_j).
\end{equation}

Легко догадаться, какой вид имеет полином $e^i(x)$. Он задается равенством 

\begin{equation}
\varepsilon^i(x) = \frac{(x-x_1)(x-x_2)..(x-x_{i-1})(x-x_{i+1}) ...(x-x_n)}{(x_i-x_1)(x_i-x_2)..(x_i-x_{i-1})(x_i-x_{i+1})..(x_i - x_n )}
\end{equation}

Мы построили \textbf{интерполяционный полином в форме Лагранжа}. 

Рассмотрим точность интерполяции. Она задается формулой 

\begin{equation}
\parallel f - \pi_n| e(x) \parallel_c \; \leq (1+ \lambda_n(f, x))\varepsilon_n(f)
\end{equation}

и во всяком случае 

\begin{equation}
\lambda_n(f, x) \geq \frac{e_n n}{8 \sqrt{\pi}} \quad (n \geq 2)
\end{equation}

Величина $\lambda_n(f, x)$ показывает насколько интерполяционный полином проигрывает полиному наилучшего приближения $p_n(f)$ \ref{eq:best_polynom}.


Эти величины сильно зависят от выбора разбиения $X$ и в случае \textbf{равномерного разбиения} для которого

\begin{equation}
x_i = a + \frac{b-a}{n-1}(i-1) \quad i = 1...n
\end{equation}

величина $\lambda_n(f, x)$ быстро растет с ростом $n$ и


\begin{equation}
\frac{2^{n-3}}{n^{3/2}} \leq \lambda_n(f, x) \leq 2^n
\end{equation} 


Таким образом, если  $f \in W^k[M_k, I]$, то ошибка интерполяции по равномерной сетке растет как $2^kn^{-k}$.

В 1901 году немецкий физик Рунге пытался интерполировать на интервале $[-1; 1]$ функцию

\begin{equation}
f(x) = \frac{1}{1+25x^2}
\end{equation}

Используя полином 20-ой степени он обнаружил, что при равномерном разбиении ошибка интерполяции катастрофически растет в окрестности точек $x = \pm 1$ и между точками $x = 0,9$ и $1$ она имеет порядок $10^2$.

Этот пример показывает, что использовать интерполяционные полиномы высокого порядка надо с большой осторожностью. Однако их можно использовать, если перейти к неравномерному разбиению, в котором шаг разбиения уменьшается при приближении к концу интервала.
Нужное разбиение задается формулой 

\begin{equation}\label{eq:27}
x_m = \frac{b+a}{2} + \frac{b-a}{2} \cos \frac{\pi(2m-1)}{2n} \quad m = 1...n
\end{equation}

И если $X = \{x_m\}$ и $f \in W^k[M_k, I]$ то 

\begin{equation}
\parallel \pi_n (f(x) -f) \parallel_c \; \leq \left(
9+ \frac{4}{\pi}\mathrm{ln} (n)\right) \left( \frac{b-a}{2}\right)^k \frac{M_kA_k}{n^k}
\end{equation}

Разбиение \ref{eq:27} называется Чебышевским, так как $\cos \frac{\pi(2m-1)}{2n}$ это нули полинома Чебышева $T_n(y) = \cos(n \; arccos(y))$

Эти полиномы играют большую роль в теории приближений.


%$\mathrm{ln}(x) = 10$


 \newpage
        \section{Интерполяционный полином в форме Ньютона \\ Численное дифференцирование и интегрирование}
На прошлой лекции был определен интерполяционный полином в форме Лагранжа (\ref{eq:2.19}, \ref{eq:2.21}). Этот же полином (он единственный) удобнее записывать в форме Ньютона:
\begin{dmath}\label{eq:3.1}
	\pi(f(x), x) = f(x_1) + f(x_1, x_2)(x-x_1) + f(x_1, x_2, x_3)(x-x_1)(x-x_2) + \dots + f(x_1, x_2, \dots, x_n)(x-x_1)\dots(x-x_{n-1})
\end{dmath}

Величины $f(x_1, x_2, \dots, x_k)$ называются разделенными разностями и определяются из рекуррентных соотношений:

\begin{equation}
\begin{cases}
	f(x_1, x_2) = \frac{f(x_2) - f(x_1)}{x_2 - x_1} \\ 
	f(x_1, x_2, x_3) = \frac{f(x_2, x_3) - f(x_1, x_2)}{x_3 - x_1} \\ 
	f(x_1, x_2, x_3, x_4) = \frac{f(x_2, x_3, x_4) - f(x_1, x_2, x_3)}{x_4-x_1} \\
	\dots
\end{cases}
\end{equation}

Так как формула (\ref{eq:3.1}) не зависит от нумерации точек разбиения, то для её доказательства достаточно перейти к нумерации, в которой $x_1$ заменяется на $x_j$ (???). Эта формула удобна тем, что при добавлении точки $x_{n+1}$ к формуле (\ref{eq:3.1}) надо просто прибавить $f(x_1, x_2, \dots, x_n, x_{n+1})(x-x_1)\dots(x-x_{n-1})(x-x_n)$. Кроме того, эта формула указывает на связь интерполяционного полинома с разложением в ряд Тейлора.

Чтобы это увидеть, рассмотрим равномерное разбиение (\ref{eq:2.24}), в котором $x_{i+1}-x_i=\delta$. Тогда 


\begin{dmath}
\begin{aligned}
	f(x_1, x_2) &= \frac{f(x_1+\delta) - f(x_1)}{\delta} \simeq f^{(1)}(x_1) \\
	f(x_1, x_2, x_3) &= \frac{1}{2\delta}f^{(1)}(x_2)-f^{(1)}(x_1) \simeq  \frac{1}{2}f^{(2)}(x_1) \\
	\dots
\end{aligned}
\end{dmath}

Имеется и следующий аналог формулы для остаточного члена в ряду Тейлора:
\begin{equation} \label{eq:3.4}
	|f(x) - \pi_k(f|X)(x)| = \frac{f^{n}(\xi)}{n!} \omega_n (x,X)
\end{equation}
В этой формуле:
\begin{equation}
\begin{aligned}
	&\xi \in [y_1, y_2]; \quad y_1=min_{i}(x-x_i); \quad y_2=max_{i}(x-x_i) ;\\ 
	&\omega_n(x, X) = (x-x_1)\dots(x-x_n)
\end{aligned}
\end{equation}
Формула \ref{eq:3.4} даёт ещё один подход к оценке точности интерполяции и показывает, какие трудности возникают при попытке использовать интеполяционный полином вне интервала интерполяции, то есть для \textbf{экстраполяции}.

\bigskip
Переходим к рассмотрению задач численного дифференцирования и интегрирования. 

Будем использовать обозначение:
\begin{equation}
	\pi_n(f|X)(x) \equiv p_n(x)\\
\end{equation}
таким образом $\pi_n(x)$ --- полином степени $n-1$.

Общая идея состоит в использовании формул вида: 
\begin{equation} \label{eq:2.8}
	f^{(1)}(x) \backsimeq {p_n}^{(1)}(x)
\end{equation}
\begin{equation}
	\int_{a}^{b}f(x) \backsimeq \int_{a}^{b}{p_n}^{(1)}(x)
\end{equation}
Рассмотрим задачу численного дифференцирования. Пусть известны величины $f(x_1), f(x_2), x_2 = x_1 + \delta$, и величина $\delta$ "достаточно мала". Задача состоит в построении приблизительной формулы для величины $f^{(1)}(x_1)$

Используем линейную интерполяцию, т.е. $p_2(x)$:
\begin{equation}
	\begin{aligned}
	p &= f(x_1) + \frac{f(x_2) - f(x_1)}{x_2-x_1}(x-x_1) \\
	{p_2}^{(1)}(x_1) &= \frac{f(x_1+\delta) - f(x_1)}{\delta}
	\end{aligned}
\end{equation}
Тогда получим следующую простейшую формулу численного дифференцирования:
\begin{dmath} \label{eq:3.10}
	f^{(1)}(x_1) \backsimeq \frac{f(x_1 + \delta) - f(x_1)}{\delta} = {p_2}^{(1)}(x_1)
\end{dmath}

Пусть $f \in W^2(M_2, I)$, оценим точность формулы \ref{eq:3.10}.

Разложим правую часть этой формулы в ряд Тейлора. Получим, что
\begin{dmath} 
	\begin{aligned}
f^{(1)}(x_1) &= \frac{1}{\delta} [f(x_1) + \frac{f^{(1)}(x_1)}{1!}\delta + O(M_2\delta^2)-f(x_1)] \\
&= f^{(1)}(x_1) + O(M_2 \delta)
	\end{aligned}
\end{dmath}
Символ \textbf{$O(M_2\delta)$} обозначает величину меньшую, чем $CM_2\delta$ ($C\backsimeq$1). Таким образом
\begin{equation}
	|f^{(1)}(x_1) - {p_2}^{(1)}(x_1)| = O(M_2\delta)
\end{equation}

Теперь предположим, что известны величины $f(x_1), f(x_2), f(x_3)$ и $x_2=x_1+\delta, x_3 = x_2 + \delta$ и вычислим  ${p_3}^{(1)}(x_2)$. Получим, что  
\begin{equation}
	{p_3}^{(1)}(x_2) = \frac{f(x_1+2\delta) - f(x_1)}{2\delta} = \frac{f(x_2+\delta) - f(x_2 - \delta)}{2\delta}
\end{equation}

Пусть $f \in W^3(M_3, I)$, оценим точность формулы
\begin{equation} \label{eq:3.14}
	f^{(1)}(x_2) \backsimeq \frac{f(x_2+\delta) - f(x_2 - \delta)}{2\delta} = {p_3}^{(1)}(x_2)
\end{equation}
Разложим в ряд Тейлора правую часть этого "'равенства"'. Получим, что
\begin{dmath} 
	\begin{aligned}
		{p_3}^{(1)}(x_2) &= \frac{1}{2\delta} [f(x_2) + {f^{(1)}(x_2)}\delta + \frac{f^{(2)}(x_2)}{2} \delta^2 \\ &- f(x_2) + f^{(1)}(x_2)\delta - \frac{f^{(2)}(x_2)}{x_2} + O(M_3\delta^3)]
	\end{aligned}
\end{dmath}
И следовательно
\begin{equation}
	f^{(1)}(x_2) =  {p_3}^{(1)}(x_2) + O(M_3\delta^2)
\end{equation}
При "'малых $\delta$"' формула \ref{eq:3.14} гораздо точнее, чем \ref{eq:3.10}, и именно ее рекомендуется использовать при обработке ЭД.

Численное дифференцирование --- неустойчивая операция, так как в ней присутствует деление на "'малую"' величину $\delta$. Рассмотрим, к чему приводит учёт ошибок в задаче численного дифференцирования.

Если известны только $y_i = f(x_i) + \varepsilon_i$, то формула \ref{eq:3.10} приобретает вид:
\begin{dmath}
	f^{(1)}(x_1) = \frac{y_2 - y_1}{\delta} \backsimeq \frac{f(x_1 + \delta) + \varepsilon_2 - f(x_1) - \varepsilon_1}{\delta} = {p_2}^{(1)}(x_1)
\end{dmath}
И при $f \in W^2(M_2, I)$, поступая так же, как и выше и учитывая, что $\varepsilon_i$ --- случайные величины, получим, что
\begin{dmath}
	\begin{cases}
		\frac{y_2 - y_1}{\delta} = f^{(1)}(x_1) + \Delta_1(\varepsilon, \delta) \\
		\Delta_1(\varepsilon, \delta) \backsimeq M_2\delta + \frac{\varepsilon}{\delta}
	\end{cases}
\end{dmath}
Величина $\Delta_1(\varepsilon, \delta)$ минимальна при
\begin{equation}
	\delta \backsimeq \sqrt{\frac{\varepsilon}{M_2}} = \delta_1; \quad \Delta_1(\varepsilon, \delta_1) = \sqrt{\varepsilon M_2}
\end{equation}
Таким образом, если при наличии ошибок мы выбираем $\delta<\delta_1$, то ошибка численного дифференцирования не уменьшится, а возрастет. В эксперименте $\delta$ --- это интервал между измерениями, и т.о. при наличии шума (случайных ошибок) не следует мерить слишком часто.

В случае формулы \ref{eq:3.14} аналогично предыдущему получим, что
\begin{dmath}
	\begin{cases}
		\frac{y_3 - y_1}{\delta} = f^{(1)}(x_2) + \Delta_2(\varepsilon, \delta) \\ 
		\Delta_2(\varepsilon, \delta) \backsimeq M_3\delta^2 + \frac{\varepsilon}{\delta} \quad (|\varepsilon_i|<\varepsilon)
	\end{cases}
\end{dmath}
Оптимальной является величина
\begin{equation}
	\delta \backsimeq {(\frac{\varepsilon}{M_2})}^{\frac{1}{3}} = \delta_2; \quad \Delta_2(\varepsilon, \delta_1) = \varepsilon^{\frac{2}{3}} \delta^{\frac{1}{3}}
\end{equation}
Так как $\varepsilon$ --- "'малая величина"' , то 
\begin{equation}
	\delta_2 >> \delta_1 \quad но \quad \Delta_2(\varepsilon, \delta_2) \backsimeq \Delta_1(\varepsilon, \delta_1)
\end{equation}

Переходим к формулам численного интегрирования. В пакетах прикладных программ эти формулы называются \textbf{квадратурными}.

В отличие от численного дифференцирования, численное интегрирование --- это устойчивая операция, и учет ошибок $\varepsilon_i$ не играет роли при оценке точности квадратурных формул.

Будем предполагать, что задано разбиение $X=(x_0=a<x_1<x_2\dots<x_n=b)$ и известны величины $f(x_i)$. Если число $n$ велико, то использование формулы \ref{eq:2.8} для построения квадратурных формул требует использования полиномов высокого порядка. Чтобы этого избежать, на практике используются \textbf{составные квадратурные формулы}. Эти формулы основаны на том, что
\begin{equation}
	\int_{a}^{b} f(x)dx = \int_{a}^{c}f(x)dx + \int_{b}^{c}f(x)dx
\end{equation}
Для получения составных квадратурных формул  отрезок $[ab]$ делится на отрезки, содержащие небольшое количество точек разбиения, и в каждом таком интервале используется интерполяционный полином небольшого порядка.

Рассмотрим, например, формулу
\begin{equation}
	\int_{a}^{b} f(x)dx = \sum_{i=0}^{n-1}\int_{x_i}^{x_{i+1}}f(x)dx
\end{equation}
Простейшая квадратурная формула, \textbf{формула трапеции}, получится, если в каждом из интегралов по отрезкам $[x_ix_{i+1}]$ $f(x)$ заменить на $p_2(x)$. В случае равномерного разбиения
\begin{equation}
x_i = a + ih; \quad h = \frac{a-b}{N} \quad (n=N) \quad i=0 \dots N
\end{equation}
формула трапеций имеет вид
\begin{dmath}
	\begin{cases}
		J = \int_{a}{b}fdx \backsimeq J^T_N \\ 
		J^T_N = [\frac{1}{2}(f(a)+f(b)) + \sum_{k=1}^{N-1}f(x_k)]\frac{b-a}{N}
	\end{cases}
\end{dmath}
и, если $f \in W^2(M_2,I)$, то
\begin{equation}
	|J-J^T_N| <= \frac{(b-a)^3}{N^2}M_2
\end{equation}

Кроме формулы трапеции на практике часто применяется \textbf{формула Симпсона}. Рассмотрим эту формулу. Предполагается, что $n=2N$, т.е.
\begin{equation}
	X = (x_0 x_1 x_2, x_2 x_3 x_4, \dots x_{2N-2}x_{2N-1}x_{2N})
\end{equation}
и интеграл $J$ записывается в виде
\begin{equation}
	J=\int_{x_0}^{x_2}f(x)dx + \int_{x_2}^{x_4}f(x)dx + \dots \int_{2N-2}^{2N}f(x)dx
\end{equation}
В каждом из интегралов в правой части этой формулы $f(x)$ заменяется на $p_3(x)$. Если разбиение равномерно
\begin{equation}
	x_k = a + \frac{h}{2}k, \quad h=\frac{b-a}{N}
\end{equation}
то в формуле Симпсона
\begin{equation}
	J \backsimeq J_N^S
\end{equation}
величина $J_N^S$ имеет вид
\begin{equation}
	J_N^S = \frac{b-a}{6N}[f(a) + f(b) + 4\sum_{k=1}^{N}f(x_{2k-1}) + 2\sum_{k=1}^{N}f(x_2k)]
\end{equation}
При $f \in W^4(M_4,I)$
\begin{equation}
	|J-J^S_N| <= \frac{M_4}{12}\frac{(b-a)^5}{N^4}
\end{equation}
При больших $N$ эта формула существенно точнее формулы трапеции.

В пакетах прикладных программ содержится много других квадратурных формул. Однако они практически не используются при обработке ЭД. Исключением являются только квадратурные формулы на основе сплайнов (см. Лекцию 4). \newpage
        \section{Сплайны}



На прошлых лекциях мы рассматривали задачи приближения функции $f:I \rightarrow\mathbb{R}$ полиномами, то есть элементами линейного пространства $P_n$ размерности $n$. Чтобы распространить другие методы приближения, прежде всего надо ввести некоторое новое конечномерное векторное пространство заменяющее $P_n$. Такой заменой может служить \textbf{ пространство} $S_{n,v}(X,I)$ \textbf{сплайнов} степени $n\geq 1$, дефекта $v\geq 1$ $(v\leq n)$, построенное по разбиению $X = (x_0=a < x_1 < x_2\ldots < x_N=b)$. 

По определению функция $s:I\rightarrow \mathbb{R}$ принадлежит пространству $S_{n,v}(X,I)$, если выполняются следующие два условия:
\begin{enumerate}
	\item На каждом интервале $(x_i,x_{i+1})$, $s(x)$ это полином степени $n$
	\item На всем интервале $I=[a,b]$ функция $s(x)$ имеет $n-v$ непрерывных производных
\end{enumerate}
Таким образом сплайн это функция "склеенная" из полиномов так, чтобы во внутренних точках $x_i,$ $ i = 1\ldots N-1$ выполняются условия "склейки":
\begin{equation}
s^{\left( k\right) }\left( x_{i}-0\right) =s^{\left( k\right) }\left( x_{i}+0\right),
\begin{aligned}k=0,1\ldots n-v\\ i=1\ldots N-1\end{aligned}
\end{equation}
Здесь через $\varphi(s_i\pm 0)$ обозначаются пределы $\varphi(x)$ при $x \rightarrow x_i$ справа и слева от точки $x_i$ производная порядка $n-v+1$ уже может иметь разрывы в точках ???.

Ясно, что $S_{n,v}(X,I)$ - линейное пространство и его размерность:
\begin{equation}
dim S_{n,v}(X,I) = v(N-1) +n+1
\end{equation}
Докажем эту формулу. У нее имеется $N$ интервалов и в каждом из них свой полином порядка $n$, который зависит от $n+1$ параметра (коэффициентов полинома). Таким образом:
\begin{equation}
\textit{число параметров} = N(n+1)
\end{equation} 
С другой стороны, в каждой из $N-1$ внутренних точек должны быть непрерывны производные порядка $0,1\ldots n-v$ и следовательно:
\begin{equation}
\textit{число условий} = (N-1)(n-v+1)
\end{equation}
Таким образом:
\begin{equation}
\begin{aligned}dim S_{n,v}(X,I) = \textit{число параметров - число условий} =\\
 N(n+1)-(N-1)(n-v+1)=v(N-1)+n+1\end{aligned}
\end{equation}
Наиболее часто используются сплайны дефекта $1$ и
\begin{equation}
dim  S_{n,1}(X,I) = N+n
\end{equation}
В пакетах прикладных программ встречаются такие понятия как \textbf{B-сплайны} и \textbf{фундаментальные сплайны}. Под B-сплайнами понимаются сплайны деффекта 1 с выбором специального базиса $B_n^j(x)$ $j=-n,-n+1,\ldots N-1$ в пространстве $S_{n,1}(X,I)$. Сплайны $B_n^j(x)$ локальны, т.е.  $B_n^j(x)\neq 0$ только если $x\in(x_i,x_i+n+1)$. При этом предпологается, что к разбиению $X$ добавлены точки $x_{-n}<x_{-n+1}<\ldots<a$ и $b<x_{N+1}<x_{N+2}<\ldots<x_{N+n}$ мы не будем давать опрделение сплайнам $B_n^j(x)$, хотя они иногда применяются для сглаживания ЭД.

Фундаментальные сплайны используются в задачах интерполяции. Эти сплайны не принадлежат пространству $S_{n,1}(X,I)$,  они принадлежат пространству $S_{n,1}(\Delta,I)$, где:
\begin{equation}
\begin{aligned}\Delta=(\varepsilon_1\ldots\varepsilon_{N-n}), \\
x_i<\varepsilon_i<x_{i+n}\end{aligned}
\end{equation}
Выбор точек $\varepsilon_1\ldots\varepsilon_{N-n}$ зависит от рассматриваемой задачи и:
\begin{equation}
dim  S_{n,1}(\Delta,I) = N+1
\end{equation}
В пространстве $S_{n,1}(\Delta,I)$ существует базис $e^i(x)$ такой , что:
\begin{equation}
e^i(x_k)=\delta_{ik},\;\; 	i=0\ldots N
\end{equation}
Знание базиса $e^i(x)$ позволяет решить задачу интерполяции функции $f(x)$. Если известны величины $f(x_k)$ это решение имеет вид:
\begin{equation}
s\left( x\right) =\sum ^{N}_{i=0}e^{i}\left( x\right) f\left( x_{i}\right)
\end{equation}
Так как из (9) следует, что:
\begin{equation}
s\left( x_k\right) =f\left( x_{k}\right),\ \  k=0\ldots N
\end{equation}

Сплайны бывают \textbf{интерполяционными} и \textbf{сглаживающими}. Сплайн называется \textbf{интерполяционными}, если выполнены условия (11).

Наибольший интерес представляют интерполяционные сплайны из $S_{3,1}(X,I)$ - \textbf{кубические сплайны деффекта 1}. Число условий (11) равно $N+1$, в то время как:
\begin{equation}
dim  S_{3,1}(X,I) = N+3
\end{equation}
Таким образом для определения интерполяционного сплайна $s(x)$ надо задать еще два условия. Наиболее часто используются условия:
\begin{equation}
s^{( 1) }(a) =\dfrac{f\left( x_{1}\right) -f\left( x_{0}\right) }{x_{1}-x_{0}},s^{\left( 1\right) }\left( b\right) =\dfrac{\left( x_{N}\right)-f( x_{N}-1)}{x_{N}-x_{N-1}}
\end{equation}
или условия:
\begin{equation}
s^{(2) }(a) =s^{(2) }(b) 
\end{equation}
Пусть такие условия заданы, тогда для определения сплайна $s(x)$ имеется(cм (4)) $3(N-1)+N+1+2=4N$ условий для поределения $4N$ коэффициентов $a^i_0,a^i_1,a^i_2,a^i_3$ определяющих полином третьего порядка в интервале $(x_i;x_{i+1}), \ (i=0,1\ldots N-1)$, т.е. надо решить систему из $4N$ линейных уравнений. 


Кроме упомянутых выше сплайнов в пакетах прикладных программ встречаются \textbf{Эрмитовы сплайны}. Это элементы пространства $S_{3,2}(X,I)$, и:
\begin{equation}
dim  S_{3,2}(X,I) = 2(N+1)
\end{equation}
Если известны величины $f(x_i),f^{(1)}(x_i)$, то существует Эрмитов сплайн $s(x)$ такой, что:
\begin{equation}
s(x_i)=f(x_i), \ s^{(1)}(x_i)=f^{(1)}(x_i)
\end{equation}

В Лекции \ref{lecture:2} были отмечены трудности, возникающие при использовании интерполяционных полиномов высокого порядка. К этим трудностям добавляется ещё и неустойчивость полинома Лагранжа относительно малых вариаций величин $f(x_i)$.
Сплайны лишены этих недотатков. Пусть $f\in W^k(M_k,I)$ и рассмотрим точность интерполяции функции $f$ с помощью кубического сплайна деффекта 1. Тогда:
\begin{equation}
\parallel s-f\parallel ?? \leq C_k h^k M_k
\end{equation}
и в этой формуле $k=max(x_{i+1}-x_i)$,
\begin{equation}
C_{2}=\dfrac{13}{48},C_{3}=\dfrac{41}{564},C_{4}=\dfrac{5}{364}
\end{equation}
и никаких трудностей типа быстрого роста $\lambda_n(f,X)$ (2.22) c ростом $n$ (2.25) - не возникает.

Английское слово "spline" означает гибкую, упругую ленту. Английские корабелы, сами того не зная, использовали сплайны. На следующем рисунке сплошной линией показана правая часть поперечного разреза парусника. 

\begin{figure}[H] % picture
	\centering
	\begin{tikzpicture}
	\begin{axis}[
	width = 10cm,
	height = 7cm,
	xmin=0,   xmax=10,
	ymin=0,   ymax=7,
	axis y line*=left,
	axis x line*=bottom,
	axis lines = left,
	domain=0:30,
	xtick={0, 0.5, 2, 8},
	xticklabels = {$a$,$x_1$,$x_2$, $x_n = b$},
	ytick={1, 2, 6},
	yticklabels = {$f_1$, $f_2$, $f_n $},
	ylabel = {$y$}, ylabel style={rotate=-90},
	xlabel = {$x$},
	every axis x label/.style={
		at={(ticklabel* cs:1.0)},
		anchor=west,
	},
	every axis y label/.style={
		at={(ticklabel* cs:1.0)},
		anchor=south,
	}
	]
	
\coordinate (A) at (0,0);
\coordinate (B) at (0.5, 1);
\coordinate (C) at (2, 2);
\coordinate (D) at (4, 2.5);
\coordinate (D1) at (6.5, 3.5);
\coordinate (E) at (7.7, 4.5);
\coordinate (F) at (8, 6);

\draw[fill] (A) circle (2pt);
\draw[fill] (B) circle (2pt);
\draw[fill] (C) circle (2pt);
\draw[fill] (D) circle (2pt);
\draw[fill] (D1) circle (2pt);
\draw[fill] (E) circle (2pt);
\draw[fill] (F) circle (2pt);

\draw[thick,black] plot [smooth, tension=0.7] coordinates {(A) (B) (C) (D) (D1) (E) (F)};

\draw [dashed] (B) -- (B -| 0, 0);
\draw [dashed] (B) -- (B |- 0, 0);
\draw [dashed] (C) -- (C -| 0, 0);
\draw [dashed] (C) -- (C |- 0, 0);
\draw [dashed] (D) -- (D -| 0, 0);
\draw [dashed] (D) -- (D |- 0, 0);
\draw [dashed] (E) -- (E -| 0, 0);
\draw [dashed] (E) -- (E |- 0, 0);

\draw [dashed] (F) -- (F -| 0, 0);
\draw [dashed] (F) -- (F |- 0, 0);

\end{axis}
\end{tikzpicture}
	\caption{Градиентный спуск.}
	\label{fig:spline}
\end{figure}










Исходными данными служили точки $\ast$. Требовалось провести через эти точки гладкую кривую. Упругая лента укладывалась так, чтобы она проходила через точки $\ast$. Это и рассматривалось как чертеж поперечного сечения будущего корабля. 

Оказывается, что полученная кривая $n=f(x)$ -- это $f(x)=s(x)$, где $s(x)$ интерполяционный сплайн из $S_{3,1}(X,I)$ с дополнительными условиями (14).


С математической точки зрения это означает, что $f(x)=s(x)$ -- это решение следующей задачи: найти функцию $f(x)$, которая дает минимум величины:
\begin{equation}
J[f]=\int_a^b[f^{(2)}(x)]^2dx
\end{equation}
при условии $f(x_i)=f_i$ ($f_i$ - заданы).
Вместо $J[f]$ можно рассмотреть величину:
\begin{equation}
J[ f| S_{0},S_1\ldots S_{N}] = J \left[ f\right] + \sum ^{N}_{i=0}\rho^{-1}_{i}\left[ f\left( x_{i}\right) -f_{i}\right] ^{2}
\end{equation}
где $\rho_i$ - заданы. Минимум этой велчины - это снова $f(x)=s(x), \ s\in S_{3,1}(X,I)$, 
но теперь уже условия $s(x_i)=f_i$ не выполняются и решение называется сглаживающим сплайном. Величины $s_i$ выбираются из условия $s_i\approx \varepsilon$, где $\varepsilon$ - величина ошибки, с которой известны величины $f(x_i)$. \newpage
        \section{Метод наименьших квадратов (1)}

\label{lecture5}
Нам удобно изменить обозначения. Будем считать, что нас интересует зависимость величины $f(t_i)$ от параметра  $t$ ( $t$ -- время, температура и т.д.). Подчеркнём, что функция $f(t_i)$ неизвестна.

ЭД -- это таблица $t_i | b_i, i = 1 \dots m$, где $b_i$ -- результат измерения и 
\begin{equation}
	b_i = f(t_i) + \varepsilon_i, \quad i = 1 \dots m
\end{equation}

Задача состоит в том, чтобы по ЭД  $t_i | b_i$ найти функцию $\overline f(t_i)$ такую, что 
\begin{equation}
	|f(t_i) - b_i| \leq \varepsilon, \quad i = 1 \dots n
\end{equation}
При этом предполагается, что ошибки $\varepsilon_i$ удовлетворяют условиям
\begin{equation}
	|\varepsilon_i | \leq \varepsilon , \quad \varepsilon \ll  |b_i| .
\end{equation}
Если такая функция $\overline f(t_i)$ найдена, то считается, что
\begin{equation}
	|\overline f(t_i) - f(t_i) | \leq \varepsilon
\end{equation}
и это всё, на что мы можем рассчитывать.


\textbf{Метод наименьших квадратов} (МНК) --  это некоторая стратегия, позволяющая построить функцию $\overline f(t_i)$.

В соответствии с МНК выбирается некоторая система функций $\varphi_k (t) $ и целое число $n$. Функуция $\overline f(t_i)$ ищется в виде 
\begin{equation}
	\overline{f} (t) = \sum_{k=1}^\inf {x_k \varphi_k (t) }
\end{equation}
При этом всегда предполагает, что $n$ много меньше $m$:
\begin{equation}
	n \ll m
\end{equation}

Неизвестные величины $x_1 \dots x_n$ находятся из условия минимальности величины $Q(x_1, x_2 \dots x_n)$, то есть из условия
\begin{equation} \label{eq:5.7}
	Q(x_1, x_2 \dots x_n) = min
\end{equation}
и по определению
\begin{equation} 
	Q(x_1, x_2 \dots x_n) = \sum_{i=1}^m {(f(t_i) - b_i)^2}
\end{equation}

Обоснование этой стратегии должно рассматриваться в курсе математической статистики.

Величина $Q(x_1, x_2 \dots x_n)$ имеет вид
\begin{equation}
	Q(x_1, x_2 \dots x_n) = Q_0 + \sum_{k=1}^m{Q_k x_k } + \sum_{k, l=1}^n {Q_kl x_k x_l}
\end{equation}
и величины $Q_0, Q_k, Q_kl$ зависят от $b_1 \dots b_n$.


В соответствии с уравнением \ref{eq:5.7} величины  $x_1 \dots x_n$ определяются из уравнений
\begin{equation} \label{eq:5.10}
	 \frac{\partial Q}{\partial x_k } = 0, \quad k = 1 \dots n
\end{equation}

Это система из $n$ линейных уравнений с $n$ неизвестными. И, если $x_1^0 \dots x_n^0$ -- решение, то надо убедиться ещё, что $Q(x_1^0 \dots x_n^0) = min$.

Возникают следующие вопросы: 
\begin{enumerate}[nolistsep]
	\item Как выбрать функции $\varphi_k (t) $  и величину $n$?
	\item Каков явный вид уравнений \ref{eq:5.10}?
	\item Как найти решение $x_1^0 \dots x_n^0$ этих уравнений?
	\item Является ли найденное решение \textbf{устойчивым}, то есть как измениться решение $x_1^0 \dots x_n^0$, если $b_i$ заменить на  $b_i = \varepsilon_i $?
\end{enumerate}

Эти вопросы будут рассмотрены в этой и следующей лекциях.

Рассмотрим первый из этих вопросов.
Из предыдущих лекций следует, что всегда можно выбрать 
\begin{equation}
	\varphi_k(t) = t^{k-1}
\end{equation}
Но возможны и другие выборы. Например, $\varphi_k(t)$ -- система В-сплайнов, $\varphi_k(t) = e^\lambda_k t$ и так далее.

Рассмотрим второй вопрос. Для ответа на него удобно ввести матричные обозначения. Обозначим через $M_{m \times n}$ множество матриц, имеющих $m$ строк и $n$ столбцов. Матрица $A \in M_{m \times n}$ -- это просто таблица
\begin{equation}\label{eq:5.12}
	A =
	\begin{pmatrix}
	a_{11} & \dots & a_{1n}\\
	a_{21} & \dots & a_{2n}\\
	\vdots & \vdots & \vdots \\
	a_{m1} & \dots & a_{mn}
	\end{pmatrix}
\end{equation}
$M_{m \times n}$ -- это линейное пространство размерности $mn$. То есть, если $A,B \in M_{m \times n}$, то определена матрица $С = \alpha A + \beta B$. Умножение на $\alpha \in \mathbb{R}$ и сложение проводится покомпонентно.

В МНК матрицы возникают естественно, так как
 \begin{equation}
	\overline{f} (t_i) = \sum_{k=1}^n {a_k x_k}, a_{ik} = \varphi_k(t_i) 
\end{equation}
Матрица $A = (a_{ik}) \in M_{m \times n}$ называется \textbf{матрицей плана}.

Если матрица $A$ имеет $n$ столбцов, а матрица $B$ -- $n$ строк, то определено их произведение
\begin{equation}
	C = AB, c_{ij} = \sum_{k=1}^n{a_{ik}b_{kj}}
\end{equation}
Таким образом, если $A \in M_{m \times n}, B \in M_{n \times l}$, то $C = AB \in M_{m \times l}$. Это умножение ассоциативно, то есть
\begin{equation}\label{eq:5.15}
	A (BC) = (AB) C
\end{equation}
Кроме умножения определена операция транспонирования Т (сопряжения)
\begin{equation}
	TA \equiv A^T, T: M_{m\times n} \rightarrow M_{n\times m}, (A^T)_{ij} = a_{ji},
\end{equation}
и легко видеть, что
\begin{equation}\label{eq:5.17}
	(AB) ^ T = B^T A^T
\end{equation}

Вектор $x \in \mathbb{R}^n$ также удобно рассматривать как матрицы из $M_{n \times 1}, (\mathbb{R}^n = M_{n \times 1})$
\begin{equation}
	x = \begin{pmatrix}
		x_1\\
		\vdots\\
		x_n
	\end{pmatrix}
	\in M_{n\times 1}
\end{equation}

Для векторов $x, y \in \mathbb{R}^n$ определено скалярное произведение
\begin{equation}
	(x, y) = \sum_{i=1}^n x_k y_k, 
\end{equation}
и обычная (эвклидова) норма $\|x\|$ вектора $x$  -- это
\begin{equation} 
	\|x \| = (x, x)^{\frac{1}{2} }= (\sum_{i=1}^n x_i^2)^{\frac{1}{2}}
\end{equation}
В матричных обозначениях
\begin{equation}
	(x, y) = x^T y = y^T x,\quad  x,y \in M_{n \times 1}
\end{equation}
Используя это замечание и равенства \ref{eq:5.15},\ref{eq:5.17}, получим, что
\begin{equation}\label{eq:5.22}
	(y, Ax) = (x, A^T y), \quad \forall A \in M_{n\times m}
\end{equation}
	

Рассмотрим, как найти явный вид уравнений \ref{eq:5.10}. Прежде всего заметим, что в матричных обозначениях соотношения $\overline f(t_i) = b_i$ (если \ref{eq:5.12}) имеют вид $Ax=b$. При $m\gg n$ эта система уравнений решения не имеет, так как это $m$ уравнений с $n \ll m$ неизвестными. 

Как проще всего из системы $Ax=b$ получить систему из $n$ уравнений? Надо просто обе части равенства $Ax=b$ умножить на $A^T$. Тогда получим
\begin{equation}\label{eq:5.23}
	Bx = a, \quad B = A^T A \in M_{n\times n}, \quad a=A^T b \in M_{n\times 1}
\end{equation}

Покажем, что система совпадает с системой \ref{eq:5.10}. Можно, конечно, явно вычислить $\frac{\partial Q}{\partial x_n} = 0$, но лучше поступить иначе.

Введем \textbf{вектор невязки}
\begin{equation}
	r(x) = Ax - b.
\end{equation}
Тогда
\begin{equation}
	Q(x_1 \dots x_n) \equiv Q(x) = \|r(x)\| ^2 = (r(x), r(x)).
\end{equation}
Рассмотрим величины
\begin{equation}
	\phi (x, \varepsilon ) = Q (x + \varepsilon ) - Q(x).
\end{equation}
Здесь $\varepsilon \in \mathbb{R}^n$ произвольный вектор, длина которого "мала". Легко видеть, что
\begin{equation}\label{eq:5.27}
	\phi (x, \varepsilon ) = 2(\varepsilon, A^T A x) + (A\varepsilon, A\varepsilon)
\end{equation}

При выводе этого равенства надо использовать \ref{eq:5.22}.

Из \ref{eq:5.27} следует, что величина $Q(x)$ имеет минимум при $x = x^0 = (x_1^0, \dots x_n^0)$, если $x^0$ -- это решение системы уравнений \ref{eq:5.23}.
То есть мы нашли явный вид системы уравнений для $x$. Это система \ref{eq:5.23}, и она называется системой \textbf{нормальных уравнений}. \newpage
        \section{Метод наименьших квадратов (2)}


В лекции \ref{lecture5} было показано, что реализация МНК сводится к решению системы \textbf{нормальных} уравнений 
\begin{equation} \label{eq:6.1}
	Bx = a, где B = A^TA, a = A^Tb
\end{equation}
и $A=M_{m \times n}$  -- матрица плана. решение системы \ref{eq:6.1} называется \textbf{квазирешением} переопределенной и не имеющей решений (при $m\gg n$) системы уравнений 
\begin{equation}
	Ax = b
\end{equation}
Рассмотрим методы поиска квазирешений, то есть рещений системы \ref{eq:6.1}. 

В пакетах прикладных программ содержится большое количество различных методов решений систем вида \ref{eq:6.1}. Ниже будут приведены некоторые из этих методов, чаще всего используемые в контексте МНК. При этом мы ограничимся методами дающими точное значение квазирешения. В соответствии с этим, методы последовательных приближений (итерационные) рассматриваться не будут.

Предполагается, что ранг матрицы $B$ равен $n$
\begin{equation}
	rank B = n
\end{equation} \newpage
        \section{Оптимимзация и методы решения систем нелинейных уравнений}
\label{lecture7}

Начнем с примера. Пусть, как и в МНК, зависимость $\overline{f}(t)$ ищется из условия (см. \ref{eq:3.5})
\begin{equation} \label{eq:7.1}
	Q(x_1, x_2, x_3, x_4) = \sum^m_{i=1}{(\overline{f}(t_i)-b_i)^2} = \min_{x_1 \dots x_4}.
\end{equation}

Но теперь в отличие от (\ref{eq:5.5}) функция $\overline{f}(t)$ зависит от $x_3$, $x_4$ нелинейно
\begin{equation} \label{eq:7.2}
	\overline{f}(t) = x_1e^{x_3 t} + x_2e^{x_4 t}.
\end{equation}
Тогда условие (\ref{eq:7.1}) приводит к нелинейной системе уравнений
\begin{equation} \label{eq:7.3}
	f_i(x) = 0 \quad
	f_i(x) \equiv f_i(x_1, x_2, x_3, x_4) = \frac{\partial Q}{\partial x_i}(x).
\end{equation}

\textbf{Задача оптимизации} состоит в поиске минимума величины 
\begin{equation} \label{eq:7.4}
	Q(x) \equiv Q(x_1 \dots x_4), Q(x) > 0,
\end{equation}
которая называется \textbf{ценой}. Как правило на $x$ накладываются условия вида $x \in U$, где область $U$ определяется условиями
\begin{equation} \label{eq:7.5}
	a_i < x_i < b_i \textrm{или} \|x\| < R.
\end{equation}

Это вносит дополнительные трудности, так как минимум $Q(x)$ может достигаться на границе $\partial U$  области $U \subset \mathbb{R}^n$. Этот случай надо исследовать отдельно, и мы его рассматривать не будем. Таким образом, предполагается, что минимум $Q(x)$ достигается во внутренней точке области $U$ и тогда задача оптимизации сводится к решению системы нелинейных уравнений
\begin{equation} \label{eq:7.6}
	f_i(x) = \frac{\partial Q}{\partial x_i}(x) = 0 \quad x = (x_1 \dots x_4), i = 1 \dots n.
\end{equation}

И обратно, задача решения системы уравнений
\begin{equation} \label{eq:7.7}
	f_i(x) = 0, i = 1 \dots n.
\end{equation}
сводится к задаче оптимизации. Для этого достаточно положить
\begin{equation} \label{eq:7.8}
	Q(x) = \sum^n_{i=1}{f_i(x)^2}.
\end{equation}

В этой лекции рассматривается задача решения системы уравнений (\ref{eq:7.7}). Это трудная задача. Все методы ее решения итерационные и имеют ограниченную область применения. 

\textbf{Общая схема итерационных} методов решения системы (\ref{eq:7.7}) (методов последовательных приближений) состоит в следующем.

Выбираются некоторое нулевое приближение $x^{(0)} = (x_1^{(0)} \dots x_4^{(0)})$ и задается алгоритм $A$ построения следующих приближений $x^{(n)}$: $x^{(n)} = A(x^{(n-1)})$, $n \geq 1$. В области $U \subset \mathbb{R}^n$ считается заданной норма $\|x\| (x \in U)$ и пара $(x^{(0)}, A)$ должны быть выбраны так, что последовательность $x^{(n)}$ сходится к решению $x = x_0 = (x_{01} \dots x_{0n})$ системы (\ref{eq:7.7}), которую мы будем записывать в виде
\begin{equation} \label{eq:7.9}
	f(x) = 0, [f(x) \in \mathbb{R}^n \quad f(x) = (f_1(x) \dots f_n(x))].
\end{equation}

Таким образом должно выполняться условие сходимости
\begin{equation} \label{eq:7.10}
	\|x^{(n)} - x_0\| \xrightarrow[n \to \infty]{} 0.
\end{equation}

Ниже рассматриваются два метода решения системы $f(x) = 0$ - \textbf{метод неподвижной точки} и \textbf{метод редукции к одномерному случаю}.

Рассмотрим метод неподвижной точки.

У нас задано отображение
\begin{equation} \label{eq:7.11}
	f: U \subset \mathbb{R}^n \to \mathbb{R}^n, (f(x) = (f_1(x) \dots f_n(x))).
\end{equation}

Запишем условие $f(x) = 0$ в виде
\begin{equation} \label{eq:7.12}
	g(x) = x \textrm{, где } g(x) = x - f(x).
\end{equation}
Так как равенство $g(x_0) = x_0$ влечет $f(x_0) = 0$, то решение системы (\ref{eq:7.9}) - это неподвижная точка отображения $g: U \to \mathbb{R}^n$.

Отображение $g: U \to \mathbb{R}^n$ называется \textbf{сжимающим} в шаре $D_a(R)$ с цетром $a$ и радиусом $R$ ($\|x - a\| < R, x \in D_a(R)$).

$D_a(R) \subset U$, если для любых двух точек $x, y \in D_a(R)$ выполняется условие сжатия
\begin{equation} \label{eq:7.13}
	\|g(x) - g(y)\| < q\|x - y\|, q < 1.
\end{equation}
Если это условие выполняется, то алгоритм построения последовательных приближений $x^{(n)}$ состоит в следующем. Выбирается любая точка $x^{(0)} \in D_a(R)$ (например $x^{(0)} = a$) и $x^{{n}}$ строится по формуле
\begin{equation} \label{eq:7.14}
	x^{(n)} = g(x^{(n-1)}), n \geq 1.
\end{equation}
Так как
\begin{equation} \label{eq:7.15}
	\|x^{(n+1)} - x^{(n)}\| = \|g(x^{(n)}) - g(x^{(n-1)})\| \leq q\|x^{(n)} - x^{(n-1)}\|,
\end{equation}
то последовательность $x^{(n)}$ сходится к решению $x_0 (f(x_0) = 0)$ и 
\begin{equation} \label{eq:7.16}
	\|x^{(n)} - x_0\| \leq \frac{q^nR}{(1-q)}.
\end{equation}


Основная трудность при применении метода неподвижной точки состоит в определении области сжатия $D_a(R)$. Система (\ref{eq:7.9}) может иметь не одно решение и рассматриваемый алгоритм сходится только если нулевое приближение $x^{(0)}$ выбрано достаточно близко к одному из решений $x_0$.

Переходим к рассмотрению \textbf{метода редукции к одномерному случаю}. Будем предполагать, что мы умеем решать одно нелинейное уравнение с одним неизвестным.

Рассмотрим систему (\ref{eq:7.7}). Эта система уравнений вида
\begin{equation} \label{eq:7.17}
	\begin{cases} 
		f_1(x_1, x_2 \dots x_n) = 0 \\
		f_2(x_1, x_2 \dots x_n) = 0 \\
		\dots \\
		f_n(x_1, x_2 \dots x_n) = 0 \\
	\end{cases}.
\end{equation}
Пусть выбарно нулевое приближение $x^{(0)} = (x_1^{(0)} \dots x_n^{(0)})$. Ищем первое приближение для величины $x_1$ из уравнения
\begin{equation} \label{eq:7.18}
	f_1(x_1, x_2^{(0)} \dots x_n^{(0)}) = 0.
\end{equation}
Заметим, что нумерация уравнений и переменных $x_i$ произвольна и $f_1$ выбирается из условия простоты решения уравнения (\ref{eq:7.18}). Это же относится и к выбору $f_2$ и так далее.

Пусть $x_1^{(1)}$ решение этого уравнения. Если оно не одно, то выбирается одно из решений. Первое приближение для $x_2$ определяется как решение уравнения
\begin{equation} \label{eq:7.19}
	f_2(x_1^{(1)}, x_2, x_3^{(0)} \dots x_n^{(0)}) = 0,
\end{equation} 
а $x_3^{(1)}$ как решение уравнения
\begin{equation} \label{eq:7.20}
	f_3(x_1^{(1)}, x_2^{(1)}, x_3, x_4^{(0)} \dots x_n^{(0)}) = 0 \textrm{ и так далее}.
\end{equation} 
В результате получаем первое приближение
\begin{equation} \label{eq:7.21}
	x^{(1)} = A(x^{(0)}).
\end{equation} 
Приближение $x^{(n)}$ получается тем же процессом
\begin{equation} \label{eq:7.22}
	x^{(n)} = A(x^{(n-1)}), n \geq 1.
\end{equation} 
Таким образом, на каждом шаге необходимо решить только одно уравнение
\begin{equation} \label{eq:7.23}
	f(x) = 0, x \in [a, b].
\end{equation} 

Рассмотрим методы решения такого уравнения. Метод \textbf{деления отрезка} $[a, b]$ пополам позволяет локализовать решение (корень уравнения (\ref{eq:7.23})). Если величины $\varphi(a)$ и $\varphi(b)$ имеют различные знаки, то в $[a, b]$ содержится нечетное число корней. Если знаки $\varphi(a)$ и $\varphi(b)$ совпадают, то в интервале $[a, b]$ либо нет корней, либо их число четно. Делим отрезок $[a, b]$ пополам. Получаем два отрезка $[a, \frac{a+b}{2}], [\frac{a+b}{2}, b]$ длины $\frac{b-a}{2}$. В каждом из них повторяем ту же процедуру, то есть сравниваем знаки $\varphi(a)$ и $\varphi(\frac{a+b}{2})$ и знаки $\varphi(\frac{a+b}{2}), \varphi(b)$.

Дальше, каждый из интервалов $[a, \frac{a+b}{2}]$ и $[\frac{a+b}{2}, b]$ делится пополам и этот процесс продолжается. В результате получается локализация корней с точностью порядка $M_1 \frac{b-a}{2^n}$ (для $\varphi \in M_1[a, b]$). 

Чтобы достичь заданной точности этот процесс можно ускорить с помощью \textbf{метода Ньютона}.

Пусть известно, что в интервале $[a', b']$ есть один корень уравнения $\varphi(x) = 0$ и извечтно, что в этом интервале $(\varphi^{(1)}(x) \neq 0)$ ($\varphi(x)$ монотонна). Исходную локализацию корней можно построить просто рассмотрев график функции $y = \varphi(x)$. Выберем некоторую точку $x^{(0)} \in [a', b']$. Так как в рассматриваемом интервале 
\begin{equation} \label{eq:7.24}
	f(x) \approx f(x^{(0)}) + f^{(1)}(x^{(0)})(x-x^{(0)}),
\end{equation} 
то определяем $x^{(1)}$ из уравнения
\begin{equation} \label{eq:7.25}
	x^{(1)} = x^{(0)} - \frac{f(x^{(0)})}{f^{(1)}(x_0)}.
\end{equation}
Приближение $x^{(n)}$ в методе Ньютона определяется из уравнения
\begin{equation} \label{eq:7.26}
	x^{(n)} = x^{(n-1)} - \frac{f(x^{(n-1)})}{f^{(1)}(x^{(n-1)})}, n \geq 1.
\end{equation} 

Метод Ньютона сходится очень быстро. Если $x_0$ корень уравнения $\varphi(x) = 0$, то
\begin{equation} \label{eq:7.27}
	|x^{(n)} - x_0| \lesssim |x^{(0)} - x_0|^{2^n}
\end{equation} 

Заметим, что всегда имеется опасность при локализации корней пропустить два близко расположенных коррня или кратный корень $x_0$, для которого $\varphi^{(1)}(x_0) = 0$.

Метод Ньютона имеет многомерное обобщение. Из формулы Тейлора следует, что 
\begin{equation} \label{eq:7.28}
	f_i(x_1 + \eta_1 \dots x_n + \eta_n) = f_i(x) + \sum^n_{j=1}{\frac{\partial f_i}{\partial x_j}(x) \eta_j + \mathcal{O}(\|\eta\|^2)}.
\end{equation} 
Введем матрицу
\begin{equation} \label{eq:7.29}
	A^{(x)} \gets M_{n \times n} \quad (A^{(x)}_{ij} = \frac{\partial f_i}{\partial x_j}(x))
\end{equation} 
и предположим, что в рассматриваемой области $detA(x) \neq 0$. Равенство (\ref{eq:7.28}) в матричных обозначениях имеет вид
\begin{equation} \label{eq:7.30}
	f(x + \eta) = f(x) + A(x)\eta + \mathcal(O)(\|\eta\|^2).
\end{equation} 
По аналогии с одномерным случаем в многомерном методе Ньютона выбирается нулевое приближение $x^{(0)} = (x^{(0)}_1 \dots x^{(0)}_n)$ и приближение $x^{(n)}, n \geq 1$ определяется равенством
\begin{equation} \label{eq:7.31}
	x^{(n)} = x^{(n-1)} - A^{-1}(x^{(n)})f(x^{(n-1)}), n \geq 1.
\end{equation} 
Здесь $A^{-1}$ матрица обратная к $A$. \newpage
        \section{Методы оптимизации}

Напомним, что рассматривается задача поиска минимума величины $Q(x) \textrm{, где } x=(x_1 \dots x_n) \textrm{, при этом } Q(x) \geq 0 \textrm{ в области } U \in \mathbb{R}^n$. Предполагается, что этот минимум лежит внутри области $U$. В действительности мы ищем \textbf{локальные минимумы} $Q(x)$, которых может быть несколько. Абсолютный минимум находится путем сравнения локальных.

Подчеркнем, что все описанные ниже алгоритмы поиска локальных минимумов можно использовать и для решения системы нелинейных уравнений вида $f(x) = 0$ (\ref{eq:7.9}).

Ниже будут описаны алгоритмы \textbf{покоординатного} и \textbf{градиентного} спусков. Эти алгоритмы содержатся в любом пакете прикладных программ. Кроме того, будет кратко описан \textbf{метод оврагов}, предложенный И. М. Гельфандом и его сотрудниками.

Начнем с \textbf{покоординатного спуска}. В сущности это описанный выше метод редукции к одномерному случаю. Для реализации покоординаного спуска снова выбирается нулевое приближение $x^{(0)} = (x^{(0)}_1 \dots x^{(0)}_n)$. Уравнения $x_2 = x_2^{(0)} \dots x_n^{(0)} = x_n^{(0)}$ задают прямую в $\mathbb{R}^n$. На этой прямой ищется минимум величины $Q(x)$ по переменной $x_1$. Это дает величины $x^{(1)}_1$. Величина $x^{(0)}_2$ определяется как минимум $Q(x)$ на прямой $x_1 = x_1^{(1)}, x_3 = x_3^{(0)} \dots x_n = x_n^{(0)}$ и так далее. В методе координатного спуска, так же как и при решении системы уравнений $f(x) = 0$, возникает проблема выбора приближения $x^{(0)}$. 

Чтобы проиллюстрировать возникающие трудности, рассмотрим двумерный случай. Пусть $x = x_1 \textrm{ и } y=x_2$. Чтобы локализовать минимумы величны $Q(x, y)$, полезно нарисовать карту, на которой горизонтали это линии уровней (кривые, на которых $Q(x, y) = c_i$). На рисунке \ref{fig:map} представлена карта с двумя минимумами в точках $(x_0, y_0), (\tilde x_0, \tilde y_0)$.


%%%figure_map_not_done

%%%figure_map_not_done


Для сходимости алгоритма точку $(x^{(0)}, y^{(0)})$ необходимо выбрать так, чтобы кривые $Q(x, y) = c_i \textrm{ при } c_i < Q(x^{(0)}, y^{(0)})$ имели вид "эллипсов".

При таком выборе точки $(x^{(0)}, y^{(0)})$  схема действия алгоритма покоординатного спуска представлена на рисунке \ref{fig:coordinate_descent}.


%%%%figure_coordinate_descent_done
\begin{figure}[h] % picture
	\centering
	\begin{tikzpicture}
\begin{axis}[
	width = 15cm,
	xmin=-13,   xmax=15,
	ymin=-10,   ymax=10,
	axis y line*=left,
    axis x line*=bottom,
    axis lines = left,
	xtick={12, 5.9, 1.6},
	xticklabels = {$x^{(0)}$, $x^{(1)}$,$x^{(2)}$},
	ytick={6.1, 2.2},
	yticklabels = {$y^{(0)}$, $y^{(1)}$},
	        ylabel = {$y$}, ylabel style={rotate=-90},
        xlabel = {$x$},
        every axis x label/.style={
    		at={(ticklabel* cs:1.0)},
   			anchor=west,
			},
		every axis y label/.style={
    		at={(ticklabel* cs:1.0)},
    		anchor=south,
			}
]


	\draw[rotate around={-55:(0,0)},black] (0,0) ellipse (8 and 12);
	\draw[rotate around={-55:(0,0)},black] (0,0) ellipse (4.7 and 9);
	\draw[rotate around={-55:(0,0)},black] (0,0) ellipse (3.6 and 5.7);
	\draw[rotate around={-55:(0,0)},black] (0,0) ellipse (1.9 and 2.9);
	
	\addplot [only marks,mark=*] coordinates { (0,0) };

    
	\coordinate[label=above:{$(x^{(0)}, y^{(0)})$}] (A) at (12, 6.1);
	
	\draw[fill] (A) circle (1pt);
	\draw [dashed] (A) -- (A |- 0,-10);
	\draw [dashed] (A) -- (A -| -13,0);
	
	\coordinate[label=above:{$(x^{(1)}, y^{(0)})$}] (B) at (5.9, 6.1);
	
	\draw[fill] (B) circle (1pt);
	\draw [dashed] (B) -- (B |- 0,-10);
%	\draw [dashed] (B) -- (B -| -13,0);
	
	\coordinate[label=right:{$(x^{(1)}, y^{(1)})$}] (C) at (5.9, 2.2);
	
	\draw[fill] (C) circle (1pt);
%	\draw [dashed] (C) -- (C |- 0,-10);
	\draw [dashed] (C) -- (C -| -13,0);
	
	\coordinate[label=above:{$(x^{(2)}, y^{(1)})$}] (D) at (1.6, 2.2);
	
	\draw[fill] (D) circle (1pt);
	\draw [dashed] (D) -- (D |- 0,-10);
	
	\draw [->, ultra thick] (A) -- (B);
	\draw [->, ultra thick] (B) -- (C);
	\draw [->, ultra thick] (C) -- (D);
	
\end{axis}
\end{tikzpicture}

	\caption{Координатный спуск.}
	\label{fig:coordinate_descent}

\end{figure}
%%%%figure_coordinate_descent_done


При поиске $x^{(1)}$ мы движемся по прямой $y = y^{(0)}$ до тех пор, пока эта прямая не коснется некоторой линии уровня и так далее.

Алгоритм покоординатного спуска быстро сходится, если "эллипс" $Q(x, y) = c_i \textrm{ при } c_i < Q(x^{(0)}, y^{(0)})$ близки к кругам. Этот алгоритм сходится медленно, если "эллипсы" сильно вытянуты вдоль оси не параллельной осям координат. В этом случае полезно перейти к координатам в которых оси "эллипсов" параллельны осям координат.

Однако лучше использовать \textbf{градиентный спуск}. В этом методе прямая, вдоль которой ищется минимум $Q(x, y)$, выбирается независимо от выбора осей координат.

Для формулировки алгоритма градиентного спуска надо ввести понятие \textbf{градиента функции $Q(x)$}, где $x = (x_1 \dots x_n)$. Это вектор $\nabla Q(x)$
\begin{equation} \label{eq:8.1}
	\nabla Q(x) = (\frac{\partial Q}{x_i}(x) \dots \frac{\partial Q}{x_i}(x))
\end{equation}
Поясним геометрический смысл вектора $\nabla Q(x)$. Единичный вектор $\xi = (\xi_1 \dots \xi_n), \|\xi\| = 1$ задает некоторое направление в $\mathbb{R}^n$. Введем величину $\nabla_\xi Q$ - производную $Q(x)$ по направлению $\xi$
\begin{equation} \label{eq:8.2}
	\nabla_\xi Q(x) = \lim_{\varepsilon \to 0} \frac{Q(x+\varepsilon\xi) - Q(x)}{\varepsilon}
\end{equation}
Так как
\begin{equation} \label{eq:8.3}
	Q(x+\varepsilon\xi) = Q(x) + \varepsilon\sum^n_{i=1}{\xi_i \frac{\partial Q}{\partial x_i}(x) + \mathcal{O}(\varepsilon^2)} = Q(x) + \varepsilon(\nabla Q(x), \xi) + \mathcal{O}(\varepsilon^2),
\end{equation}
то отсюда следует, что
\begin{equation} \label{eq:8.4}
	\nabla_\xi Q(x) = (\nabla Q(x), \xi) 
\end{equation}
Вектор $\nabla Q$ задает направление $\xi_0$
\begin{equation} \label{eq:8.5}
	\nabla Q = \|\nabla Q(x)\|\xi_0 
\end{equation}
и следовательно
\begin{equation} \label{eq:8.6}
	\nabla_\xi Q(x) = \|\nabla Q(x)\|(\xi, \xi_0) 
\end{equation}
Таким образом, $\xi_0$ - это направление, вдоль которого величина $Q(x)$ меняется наиболее быстро, и вектор $\nabla Q(x)$ направлен по нормали к поверхности уровня $Q(x) = C$.

Алгоритм градиентного спуска действует следующим образом. Пусть задано нулевое приближение $x^{(0)} \in \mathbb{R}^n$. Определим функцию $\varphi_0(t) (t \in \mathbb{R}^1)$ равенством
\begin{equation} \label{eq:8.7}
	\varphi_0(t) = Q(x^{(0)} - t\nabla Q(x^{(0)})) 
\end{equation}
и найдем $t_0$ из условия $\varphi_0(t) = min$. Тогда
\begin{align}
	\begin{split} \label{eq:8.8}
		x^{(1)} = x^{(0)} - t_0\nabla Q(x^{(0)}) \\
		x^{(2)} = x^{(1)} - t_1\nabla Q(x^{(1)}),
	\end{split}
\end{align}
где $t_1$ точки минимума функции
\begin{equation} \label{eq:8.9}
	\varphi_1(t) = Q(x^{(1)} - t\nabla Q(x^{(1)})) \textrm{ и так далее}
\end{equation}

Схема действия алгоритма градиентного спуска в двумерном случае $(x_1 = x, x_2 = y)$ показана на рисунке \ref{fig:2d_gradient}.


%%%%figure_2d_gradient_done
\begin{figure}[h] % picture
	\centering
	\begin{tikzpicture}
	\begin{axis}[
	width = 15cm,
	xmin=-13,   xmax=13,
	ymin=-10,   ymax=10,
	axis y line*=left,
	axis x line*=bottom,
	axis lines = left,
	domain=0:30,
	xtick={7.9, 1.9, -1.1},
	xticklabels = {$x^{(0)}$, $x^{(1)}$,$x^{(2)}$},
	ytick={7, 1, 4},
	yticklabels = {$y^{(0)}$, $y^{(1)}$, $y^{(2)}$},
	ylabel = {$y$}, ylabel style={rotate=-90},
	xlabel = {$x$},
	every axis x label/.style={
		at={(ticklabel* cs:1.0)},
		anchor=west,
	},
	every axis y label/.style={
		at={(ticklabel* cs:1.0)},
		anchor=south,
	}
	]
	
	\coordinate (O) at (-0.7, 2.8);
	\coordinate (O1) at (-2.1, 3.1);
	
	\draw[rotate around={-45:(0,0)},black] (0,0) ellipse (8 and 10);
	\draw[rotate around={-45:(O)},black] (O) ellipse (3.2 and 4);
	\draw[rotate around={-45:(O1)},black] (O1) ellipse (1 and 1.25);
	%	
	%	\draw[fill] (0, 0) circle (1pt);
	%
	
	\coordinate[label=right:{$(x^{(0)}, y^{(0)})$}] (A) at (7.9, 7);
	\draw[fill] (A) circle (1pt);
	
	\addplot[dashed] {-x + 14.9}; % perpendicular is x
	
	\coordinate[label=right:{$(x^{(1)}, y^{(1)})$}] (B) at (1.9, 1);
	\draw[fill] (B) circle (1pt);
	
	\draw[dashed] (A) -- (-0.1, -1); 
	
	\coordinate[label=right:{$(x^{(2)}, y^{(2)})$}] (C) at (-1.1, 4);
	\draw[fill] (C) circle (1pt);
	
	\draw[dashed] (B) -- (-3.1, 6); 

		
  	\draw[->, ultra thick] (A) -- (B);
  	\draw[->, ultra thick] (B) -- (C);
  	
  	
  	\draw [dashed] (A) -- (A |- 0,-10);

	\draw [dashed] (A) -- (A -| -13,0);
	
	\draw [dashed] (B) -- (B |- 0,-10);
	\draw [dashed] (B) -- (B -| -13,0);
	
	\draw [dashed] (C) -- (C |- 0,-10);
	\draw [dashed] (C) -- (C -| -13,0);
	

\end{axis}
\end{tikzpicture}
	\caption{Градиентный спуск.}
	\label{fig:2d_gradient}
\end{figure}
%%%%figure_2d_gradient_done


Как правило, алгоритм градиентного спуска работает существенно быстрее алгоритма покоординатного спуска. Однако, если карта уровней $Q(x, y) = c_i$ выглядит как показано на рисунке \ref{fig:map_weird}, то оба описанных выше алгоритма сходятся очень медленно.



\begin{figure}[h] % picture
	\centering
	\begin{tikzpicture}
	
	\begin{axis}[
	xmin=-5,   xmax=7,
	ymin=-2,   ymax=7,
	axis y line*=left,
    axis x line*=bottom,
    axis lines = left,
    domain=0:30,
    tick label style = {white},
    grid = major,
	    ylabel = {$y$}, ylabel style={rotate=-90},
        xlabel = {$x$},
        every axis x label/.style={
    		at={(ticklabel* cs:1.0)},
   			anchor=west,
			},
		every axis y label/.style={
    		at={(ticklabel* cs:1.0)},
    		anchor=south,
			}
]
 
 \draw [ultra thick,black] (-4, -1 ) to[out=-45,in=-150] (4, 1) to[out=30,in=-100] (6, 5)  to[out=80,in=30] (5, 6) to[out=-150,in=40] (2, 2) to[out=-140,in=10] (-4, 0) to[out=-170,in=135] (-4, -1)	 
 ;
 
 \draw [ultra thick,black] (-3, -1 ) to[out=-45,in=-150] (1, 0) to[out=30,in=-100] (5.5, 4.5)  to[out=80,in=60] (4.3, 4.5) to[out=-110,in=20] (1, 1) to[out=-160,in=5] (-3, -0.5) to[out=-175,in=135] (-3, -1)	 
 ;
		
\end{axis}
	\end{tikzpicture}
	\caption{Овраг}
	\label{fig:map_wierd}
\end{figure}




И. М. Гельфанд с сотрудниками предложили \textbf{метод оврагов}, который и в этом случае быстро находит дно оврага - точку $(x_0, y_0)$. Идея состоит в том, что исходной является не одна точка $(x^{(0)}, y^{(0)})$, а две $(x^{(0)}_1, y^{(0)}_1), (x^{(0)}_2, y^{(0)}_2)$. Схема этого метода представлена на рисунке \ref{fig:ravine_method}.






\begin{figure}[h] % picture
	\centering
	\begin{tikzpicture}
	
	\begin{axis}[
	width = 16cm,
	height = 6cm,
	xmin=-7,   xmax=7,
	ymin=-1,   ymax=6,
	axis line style={draw=none},
    domain=0:30,
    tick style={draw=none},
    xtick = \empty,
    ytick = \empty
]
\coordinate[label=above:{$(x_1^{(0)}, y_1^{(0)} )$}] (A) at (-6, 3);
\draw[fill] (A) circle (2pt);

\coordinate[label=above:{$(x_2^{(0)}, y_2^{(0)} )$}] (B) at (-4, 4);
\draw[fill] (B) circle (2pt);


\coordinate (A1) at (-5, 1);
\coordinate (B1) at (-3, 1);
\draw[->, ultra thick] (A) -- (A1) node[midway,sloped,above] {$\nabla-$спуск};
\draw[->, ultra thick](B) -- (B1) node[midway,sloped,above] {$\nabla-$спуск};

\draw[dashed] ($(A1)!-5em!(B1)$) -- ($(A1)!20em!(B1)$);

\coordinate[label=below:{минимум на прямой, $(x^{(1)}, y^{(1)} )$} ] (C) at (-1, 1);
\draw[fill] (C) circle (2pt);
\draw[] (C) circle (5pt);

\coordinate (C1) at (1,3);
\draw[->, ultra thick] (C) -- (C1) node[midway,sloped,below] {$\nabla-$спуск} ;

\draw[dashed] ($(B1)!-5em!(C1)$) -- ($(B1)!20em!(C1)$);

\coordinate (D) at (3, 4);

\node[rotate=20] (F) at (2.6, 4.7) {минимум на прямой};

\draw[fill] (D) circle (2pt);
\draw[] (D) circle (5pt);

\coordinate (D1) at (5, 3);
\draw[->, ultra thick] (D) -- (D1) node[midway,sloped,above] {$\nabla-$спуск} ;
\draw[dashed] ($(C1)!-5em!(D1)$) -- ($(C1)!20em!(D1)$);

		
\end{axis}
	\end{tikzpicture}
	\caption{Метод оврагов.}
	\label{fig:ravine_method}
\end{figure}






\begin{figure}[h] % picture
	\centering
	\begin{tikzpicture}
	
	\begin{axis}[
	width = 15cm,
	xmin=-7,   xmax=7,
	ymin=-7,   ymax=7,
	axis y line*=left,
    axis x line*=bottom,
    axis lines = left,
    domain=0:30,
    grid = major,
    tick label style = {white},
		    ylabel = {$y$}, ylabel style={rotate=-90},
        xlabel = {$x$},
        every axis x label/.style={
    		at={(ticklabel* cs:1.0)},
   			anchor=west,
			},
		every axis y label/.style={
    		at={(ticklabel* cs:1.0)},
    		anchor=south,
			}
]
 \draw [ultra thick,black] (5, 5 ) to[out=-45,in=30] (4, -1)
 		to[out=-150,in=45] (2,-2) to[out=-135,in=30] (0, -5)
 		to[out=-150,in=-45] (-5, -5)  to[out=135,in=-150] (-4, 1) 
 		to[out=30,in=-135] (-2, 2)  to[out=45,in=-150] (0, 5)
 		to[out=30,in=135] (5, 5) ;
 \draw [ultra thick,black] (4, 4 ) to[out=-45,in=30] (3, -0.8)
 		to[out=-150,in=45] (1.5,-1.5) to[out=-135,in=30] (0, -4)
 		to[out=-150,in=-45] (-4, -4)  to[out=135,in=-150] (-3, 0.7) 
 		to[out=30,in=-135] (-1.5, 1.5)  to[out=45,in=-150] (0, 4)
 		to[out=30,in=135] (4, 4)  ;
 \draw [ultra thick,black] (3.3, 3.3 ) to[out=-45,in=30] (2.5, 0)
 		to[out=-150,in=0] (0, 0)  to[out=90,in=-150] (1, 3)
 		to[out=30,in=135] (3.3,3.3)  ;
  \draw [ultra thick,black] (-3.3, -3.3 ) to[out=135,in=-150] (-2, 0)
 		to[out=30,in=180] (0, 0)  to[out=-90,in=30] (-0.5, -2.7)
 		to[out=-150,in=-45] (-3.3, -3.3) ;
 \draw [ultra thick,black] (2.5, 2.5) to[out=-45,in=-45] (0.5, 0.5) to[out=135,in=135] (2.5,2.5 );
  \draw [ultra thick,black] (2, 2) to[out=-45,in=-45] (1, 1) to[out=135,in=135] (2,2 );
  \draw [ultra thick,black] (-2.5, -2.5) to[out=-45,in=-45] (-0.5, -0.5) to[out=135,in=135] (-2.5,-2.5 );
  \draw [ultra thick,black] (-2, -2) to[out=-45,in=-45] (-1, -1) to[out=135,in=135] (-2,-2 );
  
  \node (A) at (-5, 6) {$Q(x) = C_1$};
  \coordinate (A1) at (-4, 5.95);
  \draw[] ($(A)!4em!(A1)$) -- ($(A)!14.4em!(A1)$);

  \node (B) at (-5, 5) {$Q(x) = C_2 < C_1$};
  \coordinate (B1) at (-4, 4.95);
  \draw[] ($(B)!4em!(B1)$) -- ($(B)!14.8em!(B1)$);
  
  \node (C) at (-5, 4) {$Q(x) = C_3 < C_2$};
  \coordinate (C1) at (-4, 3.95);
  \draw[] ($(C)!4em!(C1)$) -- ($(C)!17em!(C1)$);
   
  \node (D) at (-5, 3) {$Q(x) = C_4 < C_3$};
  \coordinate (D1) at (-4, 2.95);
  \draw[] ($(D)!4em!(D1)$) -- ($(D)!15.5em!(D1)$);
  
  \node (E) at (-5, 2) {$Q(x) = C_5 < C_4$};
  \coordinate (E1) at (-4, 1.95);
  \draw[] ($(E)!4em!(E1)$) -- ($(E)!14.3em!(E1)$);
  

\end{axis}
		
	\end{tikzpicture}
	\caption{Сложный ландшафт.}
	\label{fig:complex_landscape}
\end{figure}












 \newpage
        \section{Лекция 9. Преобразование Фурье и его приложения}
Чтобы подойти к определению преобразования Фурье сначала рассмотрим фукцию $f(t)$ с периодом $2\pi$.

\begin{equation}\label{9.1}
	f(t+2\pi) = f(t)
\end{equation}

Такая функция раскаладывается в ряд Фурье 

\begin{equation}\label{9.2}
f(t) = \frac{a_0}{2} + \sum_{n = 1}^{\infty}(a_n cos(nt) + b_n sin(nt)) 
\end{equation}

Для сходимости этого ряда достаточно предположить, что $f(t)$ -- непрерывная функция.
Коэффициенты Фурье $a_n$, $b_n$ задаются равенствами

\begin{equation}\label{9.3}
a_n = \frac{1}{\pi} \int_{0}^{2\pi} f(t) cos(nt) dt,\quad b_n = \frac{1}{\pi} \int_{0}^{2\pi} f(t) sin(nt) dt
\end{equation} 

Если $t$ -- время, то функцию $f(t)$ принято называть \textbf{сигналом}. 
Таким образом у нас есть взаимооднозначное соответствие $f(t) \rightarrow \{a_n, b_n\}$, и все свойства функции $f(t)$ имеют свое отражение в свойствах множества $\{a_n, b_n\}$.

В случае непериодических функций аналогом формулы \ref{9.2} является равенство 

\begin{equation}\label{9.4}
f(t) = \frac{1}{2\pi}\int_{-\infty}^{+\infty} e^{i\omega t} \hat{f} (\omega) d\omega,
\end{equation}
где функция $\hat{f}(\omega)$ называется \textbf{преобразованием Фурье} функции $f(t)$ ($ \hat{f} = \mathscr{F} f$)
о определяется равенством 

\begin{equation}\label{9.5}
\hat{f} (\omega) = \int_{-\infty}^{+\infty} e^{-i\omega t} f(t) dt,
\end{equation}
Величина $\hat{f} (\omega)$ во всяком случае существует, если 

\begin{equation}\label{9.6}
\int_{-\infty}^{+\infty} |f(t)| dt < \infty
\end{equation}

и в этом случае имеет место \textbf{формула обращения} \ref{9.4}  $(f =\mathscr{F}^{-1} \hat{f})$.
Функция $\hat{f}(\omega)$ часто называется \textbf{спектром сигнала} $f(t)$. 

Объясним происхождение формулы обращения и приведем ее эвристический вывод. Используя формулы Эйлера

\begin{equation}\label{9.7}
	\cos nt = \frac{1}{2}(e^{i n t} + e^{-i n t}), \quad 	\sin nt = \frac{1}{2i}(e^{i n t} - e^{-i n t}), 
\end{equation}
запишем формулу \ref{9.2} в виде

\begin{equation}\label{9.8}
f(t) = \sum_{n = -\infty}^{+\infty} c_n e^{int}
\end{equation}

\begin{equation} \label{9.9}
c_n = \frac{1}{2\pi} \int_{-\pi}^{+\pi} f(t) e^{-i n t} dt
\end{equation}

В формуле \ref{9.9}
\begin{equation}\label{9.10}
c_n = \frac{1}{2}(a_n - ib_n) \; n \geq 0, \quad c_{-n} =  \frac{1}{2}(a_n + ib_n). 
\end{equation}

Формула \ref{9.4} аналог формулы \ref{9.8}, а формула \ref{9.5} -- аналог \ref{9.9}.

Идея "вывода" формулы обращения \ref{9.4} состоит в том, чтобы рассматривать непериодические функции как пределы периодических при стремлении периода к бесконечности. 

Пусть функция $f(x)$ имеет период $2e$ и
\begin{equation}\label{9.11}
f(x+2e) = f(x)
\end{equation}

\begin{equation}\label{9.12}
g(t) = f\left(\frac{et}{\pi}\right) \quad x = \frac{et}{\pi}, \; t = \frac{\pi x}{e}
\end{equation}
имеет период $2\pi$ и 

\begin{equation}\label{9.13}
g(t+2\pi) = g(t)
\end{equation}

Разлагая $g(t)$ в ряд Фурье и переходя к переменной $x$, получим, что 

\begin{equation}\label{9.14}
f(x) = \sum_{n = -\infty}^{+\infty}c_n e^{i n \frac{\pi x}{e}}
\end{equation}

\begin{equation}\label{9.15}
c_n = \frac{1}{2e} \int_{-e}^{e} e^{-i n \frac{\pi y}{e}} f(y) dn
\end{equation}
	
и таким образом 
\begin{equation}\label{9.16}
f(x) = \frac{1}{2\pi}\sum_{n = -\infty}^{+\infty} \Delta\omega e^{in\Delta\omega ln x} \int_{-\infty}^{+\infty} e^{-in\Delta\omega y} f(y) dy , \Delta\omega=\frac{\pi}{e}	
\end{equation}

Так как

\begin{equation}\label{9.17}
\int_{-\infty}^{+\infty} e^{i\omega x} F(\omega)d\omega \simeq	\sum_{n = -\infty}^{+\infty} e^{in\Delta\omega x} F(n\Delta\omega)\Delta\omega y,
\end{equation}	

то при $e \rightarrow 0$ из \ref{9.6} следует, что 

\begin{equation}
f(x) = \frac{1}{2\pi}\int_{-\infty}^{+\infty} e^{i\omega x}d\omega \int_{-\infty}^{+\infty} e^{i\omega y}f(y)dn = \frac{1}{2\pi}\int_{-\infty}^{+\infty}e^{i\omega x} \hat{f}(\omega)d\omega
\end{equation}
Заменяя $х$ на $t$ получаем формулу обращения \ref{9.4}.

Рассмотрим некоторые свойства преобразования Фурье. Найдем, например, преобразование Фурье $\hat f^{(1)} (\omega)$ производной $f^{(1)}(t)$

\begin{equation}\label{9.19}
f^{(1)}(t) = \frac{d}{dt} \frac{1}{2\pi} \int_{-\infty}^{+\infty} \hat{f^{(1)}}(\omega) e^{-i\omega t}d\omega
\end{equation}
С другой стороны 
\begin{equation}\label{9.20}
f^{(1)}(t) = \frac{d}{dt} \frac{1}{2\pi} \int_{-\infty}^{+\infty} e^{-i\omega t} \hat{f}(\omega)d\omega = \frac{1}{2\pi} \int_{-\infty}^{+\infty}(-i\omega)\hat{f}(\omega) e^{-i\omega t}d\omega
\end{equation}

Таким образом
\begin{equation}\label{9.21}
\hat{f^{(1)}}(\omega) = -i\omega \hat{f}(\omega)
\end{equation}
и на языке преобразований Фурье дифференцирование -- это умножение $\hat{f}(\omega)$ на $(-i\omega)$.

Это позволяет, например, решить дифференциальное уравнение вида

\begin{equation}\label{9.22}
f^{(2)}(t) + a f^{(1)}(t) + bf(t) = g(t)
\end{equation}

Беря преобразование Фурье от обеих частей этого равенства, получим, что 
\begin{equation}\label{9.23}
\begin{gathered}
(-i\omega)^2 \hat{f}(\omega) + a(-i\omega) \hat{f}(\omega)+b\hat{f}(\omega) = g(\omega) \\
\hat{f}(\omega) = \frac{\hat{g}(\omega)}{-\omega^2-i\omega a + b} \; \text{и} \; f(t)= \mathscr{F}^{-1} (\hat{f}(\omega)).
\end{gathered}
\end{equation}

Будем говорить, что $f(t)$ имеет эффективный носитель $L(f) \sim a$, если при $|x|>\alpha a $,  $\;\alpha >> 1 \;$ $f(x)$ достаточно быстро убывает, например как $e^{-|x|}$. Оказывается, что 

\begin{equation}\label{9.24}
L(f) L(\hat{f}) \sim 1.
\end{equation}

Это соотношение называется \textbf{"соотношением неопределенностей"}. Оно показывает с какой точностью можно одновременно локализовать $f(x)$ и $\hat{f}(\omega)$. Знаменитое соотношение неопределенности $\Delta p \Delta x \sim h$ в квантовой механикеявляется следствием \ref{9.24}. Рассмотрим пример

\begin{equation}\label{9.25}
\begin{gathered}
f(t)=e^{-a^2t^2}, L(f) \sim \frac{1}{a},\\
\hat{f}(\omega)=\frac{\sqrt{n}}{a} e^{-\frac{\omega^2}{na^2}}, L(\hat{f}) \sim a,
\end{gathered}
\end{equation}

что и доказывает \ref{9.24} для рассматриваеой функции $f(t)$.

В теории преобразований Фурье большую роль играет понятие свертки $f_1*f_2 $ двух функций. 
По определению 

\begin{equation}\label{9.26}
(f_1*f_2)(t) = \int_{-\infty}^{+\infty}f_1(g)f_2(t-g)dg = (f_2*f_1)(t).
\end{equation}

Значение этой операции объясняется тем, что

\begin{equation}\label{9.27}
\hat{(f_1*f_2)}(\omega) = \hat{f_1}(\omega)\hat{f_2}(\omega)
\end{equation}

Полезность преобразования Фурье в задачах обработки экспериментальных данных  проиллюстрируем двумя примерами.
В качестве первого примера рассмотрим \textbf{задачу сглаживания}.

На рисХ представлен результат сглаживания $R$ четной функции $f(t)$.

Следующая последовательность отображений описывает алгоритм сглаживания $R$.

\begin{math}
\begin{gathered}
f(t) \xrightarrow{\mathscr{F}} \hat{f}(\omega) \xrightarrow{R} \hat{f}(\omega)R(\omega)  \xrightarrow{\mathscr{F}^{-1}} f_R(t) =(f\otimes H_R)(t) \\ 
\hat{H_R}(\omega)=R(\omega), \quad R(\omega)=R(-\omega) 
\end{gathered}
\end{math}


Сглаживание (фильтр высоких частот) задается функцией $R(\omega)$, вид которой представлен на рисункеУ.

Так как спектр шума (случайных ошибок) лежит в высоких частотах, то сглаживание используется и для \textbf{фильтрации} шума.

В качестве второго примера рассмотрим задачу обнаружения малого сигнала $\Delta f(t)$, содержащего высокочастотную компоненту, на фоне большого низкочасотного сигнала $f_0(t)$.

Пусть
\begin{equation}\label{9.28}
\begin{gathered}
f(t)=f_0(t)+\Delta f(t) \\
f_0(t)=e^{-a^2t^2}, \Delta f(t)= \varepsilon cos \omega_0 t e^{-b^2t^2} \quad \varepsilon << 1
\end{gathered}
\end{equation}

При малых $\varepsilon$ сигнал $\Delta f(t)$ "плохо видим" на фоне $f_0(t)$. Переходим к преобразованию Фурье. Используя \ref{9.25} имеем


\begin{equation}\label{9.29}
\begin{gathered}
\hat{f}_0(\omega) \sim a^{-1} e^{\frac{-\omega^2}{4a^2}} \\
\Delta\hat{f}(\omega) \sim \frac{\varepsilon}{b} \left(e^{-\frac{(\omega-\omega_0)^2}{4b^2}} + e^{-\frac{(\omega+\omega_0)^2}{4b^2}} \right)
\end{gathered}
\end{equation}

Величина $\Delta\hat{f}(\omega_0)$ имеет максимум при $\omega = \omega_0$ и 
\begin{equation}\label{9.30}
\Delta\hat{f}(\omega_0) \sim \varepsilon b^{-1}
\end{equation}

С другой стороны  

\begin{equation}\label{9.31}
\Delta\hat{f}(\omega_0) \sim a^{-1} e^{\frac{-\omega_0^2}{4a^2}}
\end{equation}
и если

\begin{equation}\label{32}
\varepsilon b^{-1} >> a^{-1} e^{\frac{-\omega_0^2}{4a^2}},
\end{equation}

то в $\omega$-области сигнал $\Delta f(t)$ "хорошо видим".

Выше рассматривались функции $f(t)$, зависящие от одной переменной.
Формулы \ref{9.4}, \ref{9.5} имеют многомерное обобщение.

Если $f(x) = f(x_1...x_n)$, то
\begin{equation}\label{33}
\hat{f}(\omega) = \int e^{-i(\omega_1 x)}f(x)dx, \quad \omega = (\omega_1 ... \omega_n)
\end{equation}

Вэтой формуле $dx = dx_1...dx_n$
\begin{equation}
\begin{gathered}
\int g(x)dx \equiv  \underbrace{\int_{-\infty}^{+\infty} ...  \int_{-\infty}^{+\infty}}_{n} g(x_1...x_n)dx_1...dx_n\\
(\omega_1 x) = \sum_{i = 1}^{n} \omega_i x_i
\end{gathered}
\end{equation}

Имеет место и аналог формулы обращения
\begin{equation}\label{35}
f(x) = \frac{1}{(2\pi)^n} \int e^{i(\omega_1 x)} \hat{f}(\omega) d\omega
\end{equation}

Двумерное преобразование Фурье используется при обработке изображений. \newpage
        \section{Дискретное преобразование Фурье (ДПФ)}
\label{lecture10}

В Лекции \ref{lecture9} было введено преобразование Фурье $f(t) \xrightarrow{\mathscr{F}} \hat{f}(\omega)$. На практике, как правило, известно только конечное число ее отсчетов $f(t_n),\quad n = 0 \; ...\;N-1$. \textbf{Дискретное преобразование Фурье} $\hat{f}^D(\omega)$ служит заменой $\hat{f}(\omega)$.

Определим ДПФ $\hat{f}^D(\omega)$. Пусть

\begin{equation}\label{10.1}
f(t) = 
\begin{cases}
0, t < 0 \\
0, t > T.
\end{cases}
\end{equation}
Тогда
\begin{equation}\label{10.2}
\hat{f}\omega = \int_{0}^{T}e^{-i\omega t}f(t)dt.
\end{equation}
Будем предполагать, что известны величины
\begin{equation}\label{10.3}
\begin{gathered}
f_n = f(n \Delta t), \quad n = 0 ... N-1 \\
\Delta t = \frac{T}{N}.
\end{gathered}
\end{equation}

ДПФ $\hat{f}^D(\omega)$ определяется формулой
\begin{equation}\label{10.4}
\hat{f}^D(\omega) = \sum_{n = 0}^{N - 1} e^{-i \omega nt}f_n\Delta t.
\end{equation}

Это просто результат замены интеграла в уравнении \ref{10.2} интегральной суммой.

Нас интересует конечное число значений $\hat{f}^D\omega$, то есть величины 

\begin{equation}\label{10.5}
A_j = \hat{f}^D(j \Delta \omega), \quad j = 0, 1 ... N.
\end{equation}

Возникает вопрос: как выбрать значение $\Delta \omega$.
В отличие от $ \hat{f}(\omega)$, $\hat{f}^D(\omega)$ периодическая функция с периодом $\omega$

\begin{equation}\label{10.6}
\hat{f}^D(\omega + \omega) = \hat{f}^D(\omega), \quad \omega = \frac{2\pi}{\Delta t}.
\end{equation}

Поэтому естественно выбрать $\Delta \omega$ из условия
\begin{equation}\label{10.7}
\Delta \omega = \frac{\omega}{N} = \frac{2\pi}{T} \text{ и } \Delta t \Delta \omega = \frac{2\pi}{N}.
\end{equation}

В этом случае
\begin{equation}\label{10.8}
A_j = \frac{1}{N}\sum_{n = 0}^{N - 1} e^{-i \frac{2\pi n}{N}} a_n, \quad i = 0 ... N - 1,
\end{equation}

и в этой формуле
\begin{equation}\label{10.9}
a_n = f_n T, \quad n = 0 ... N-1.
\end{equation}

В общем случае (отвлекаясь от формулы \ref{10.9}) говорят, что $\{A_j\}$ -- это \textbf{ДФП последовательности} $\{a_n\}$, и 

\begin{equation}\label{10.10}
\{a_n\} \xrightarrow{\mathscr{F}_D} \{A_j\}.
\end{equation}

Основанием для такого определения служит \textbf{формула обращения}

\begin{equation}\label{10.11}
 \{A_j\}\xrightarrow{\mathscr{F}^{-1}_D} \{a_n\} ,
\end{equation}
определяемая равенством

\begin{equation}\label{10.12}
a_n = \sum_{i = 0}^{N - 1} e^{i \frac{2\pi n}{N}} j A_j, \quad i = 0 ... N - 1.
\end{equation}

Докажем эту формулу. Если в \ref{10.12} подставить выражения для $A_i$ из \ref{10.8}, то должно выполняться равенство
\begin{equation}\label{10.13}
a_n = \sum_{k = 0}^{N - 1} \frac{a_k}{N} \sum_{i = 0}^{N - 1} e^{i \frac{2\pi }{N}}(n-k) j,
\end{equation}

которое действительно выполняется, так как

\begin{equation}\label{10.14}
\sum_{i = 0}^{N - 1} e^{i \frac{2\pi }{N}}(n-k) j = N\delta_{n,k}.
\end{equation}

Это равенство -- следствие формулы для суммы геометрической прогрессии
\begin{equation}\label{10.15}
\sum_{i = 0}^{N - 1} x^i = \frac{1-x^N}{1-x}.
\end{equation}

Мы определили ДПФ $\hat{f}^D(\omega)$, но нас интересует $\hat{f}(\omega)$. Рассмотрим вопрос о связи этих двух функций. Для достаточно гладкой функции $f(t)$  $\hat{f}(\omega)$ достаточно быстро убывает при $|\omega| \rightarrow \infty $. Это следует из того, что (см. \ref{9.21})

\begin{equation}\label{10.16}
f^{(k)}(t) \equiv \frac{d^k}{dt^n}f(t) = \int_{-\infty}^{+\infty} (-i\omega)^k \hat{f}(\omega)d\omega,
\end{equation}

и, следовательно,

\begin{equation}\label{10.17}
\int_{-\infty}^{+\infty}\omega^k \hat{f}(\omega)d\omega < \infty.
\end{equation}

Таким образом, определена функция

\begin{equation}\label{10.18}
\begin{gathered}
F(\omega) = \sum_{n = -\infty}^{+\infty} \hat{f}(\omega+ n\omega), \\
F(\omega+\omega) = f(\omega),
\end{gathered}
\end{equation}

имеющая, как и $\hat{f}^D(\omega)$, период $\omega$. 
Оказывается, что эта функция совпадает с $\hat{f}^D(\omega)$, то есть

\begin{equation}\label{10.19}
\hat{f}^D(\omega) = \sum_{n = -\infty}^{+\infty} \hat{f}(\omega+ n\omega).
\end{equation}

Эта формула называется формулой Пуассона.

Так как $\hat{f}(\omega)$ достаточно быстро убывает при $|\omega| \rightarrow \infty $, то можно предположить что, если $\Delta t$ достаточно мало, то

\begin{equation}\label{10.20}
\hat{f}^D(\omega) \simeq \hat{f}(\omega) \quad |\omega| < \frac{\omega}{2}.
\end{equation}

Интервал $|\omega| < \frac{\omega}{2}$ называется \textbf{интервалом Найквиста}.

Картина приближения функции $\hat{f}(\omega)$ с помощью $\hat{f}^D(\omega)$ при достаточно малом значении $\Delta t$ представлена на рисунке Х.

Если величина $\Delta t$ не достаточно мала, то возникает \textbf{эффект наложения} при котором вклад высоких частот искажает информацию о поведении $\hat{f}(\omega)$ при низких частотах (См. Рис 11).

Этот эффект все видели в кино, когда на экране колеса автомобиля вращаются в обратную сторону.

В пакетах прикладных программ прямое и обратное ДПФ реализуются с помощью алгоритма быстрого преобразования Фурье (БПФ). Если прямо вычислять $N$ значений $A_j$ по $N$ значениям $a_n$, то требуется $\sim N^2$ операций. Оказывается, что по крайней мере при $N = 2^k$ это можно сделать на $N \mathrm{ln}N$ операций. БПФ реализует такой алгоритм (алгоритм Кули-Тьюки). Алгоритм БПФ открыл путь к практическому применению ДПФ при больших $N$. \newpage
       
       %\footnotesize \bibliographystyle{mybib5.bst}  
\end{document}