\documentclass[oneside,final,12pt]{article} %одностороння печать, чистовая версия, размер кегля, класс документа
\usepackage{ucs} 
\usepackage[utf8x]{inputenc} 
\usepackage[T2A, T1]{fontenc} 
\usepackage[english,russian]{babel} %оформление кириллицей (подписи к таблицам и т.д.)
\usepackage{ifpdf}
\usepackage{float} %для плавающих картинок и таблиц
%\usepackage{wrapfig} %для плавающих картино


\ifpdf  %% если используется pdfTEX
\usepackage{cmap}  %поиск по кириллице в готовом pdf
\usepackage[pdftex]{graphicx} %работа с графикой 
\usepackage[unicode=true]{hyperref}
\usepackage{pdfpages}
\else   %% если используется не pdfTEX
\usepackage[dvips]{graphicx}
\fi
 

\usepackage{vmargin} %размеры полос набора
\setpapersize{A4} %формат бумаги
\setmarginsrb{30mm}{25mm}{25mm}{25mm}{0pt}{0mm}{0pt}{13mm} %размеры полей: левое, верхнее, правое, нижнее, 3*колонтитулы, расстояние между нижним краем нижней строки и нижним краем номера страницы
\usepackage{indentfirst} %красная строка для первого абзаца главы или параграфа
\sloppy %борьба с залезанием строк на поля путём изменения размеров пробелов
\usepackage{amsmath} %дополнительные средства для вёрстки формул
\everymath{\displaystyle}
\usepackage{breqn} %для dmath разбить длинную формулу
%\usepackage{esvect} %vectors

%\usepackage{amscd} %диаграммы
\usepackage{amsfonts} %дополнительные шрифты для формул
\usepackage{amssymb} %дополнительные символы для формул
\usepackage[nottoc,notlot,notlof]{tocbibind}

\pagestyle{plain} %включена нумерация страниц 
\renewcommand{\thesection}{\arabic{section}} %одинарная нумерация формул


\usepackage{verbatim} %comments

\usepackage[font=small]{caption}
\usepackage{booktabs} %отступы в tabular
\usepackage{colortbl} %раскаршивание таблиц
\usepackage{xcolor} %название цветов



\usepackage{lineno} %нумерация всех строк для отладки
\usepackage{enumitem} %особенности enumerate
\usepackage{pbox} % для переносов внутри ячейки таблицы
\usepackage{array} %??
\usepackage{longtable}% перенос таблиц на страницах

\newlength{\width}
\setlength{\width}{0.97\textwidth} % у меня почему-то textwidth шире, чем текст и таблицы вылезали, поэтому я сделала меру длины поменьше
\definecolor{dark-gray}{gray}{0.4} %определение цвета
\newcommand{\graytable}[0]{\arrayrulecolor{dark-gray}} % сделать таблицу серой
\newcommand{\thinrule}[0]{\specialrule{0.3pt}{4pt}{4pt}} % тоненькая линия
\newcommand{\verythinrule}[0]{\specialrule{0.1pt}{1pt}{1pt}} %очень тоненькая линия если надо
\newcommand{\invisiblerule}[0]{\specialrule{0pt}{2pt}{2pt}} %просто создать пустого места в таблице



\usepackage{textgreek} % \textalpha греческие буквы не в math mode
\usepackage{subcaption} % несколько картинок в одной
\usepackage{multirow} % объединение ячеек
\setcounter{tocdepth}{2} % глубина содержания

\usepackage{setspace} %for setstretch
\usepackage{cite}



\begin{document}
        \begin{titlepage}

\newcommand{\HRule}{\rule{\linewidth}{0.3mm}} % Defines a new command for the horizontal lines, change thickness here

\center

\textbf{\textsc{\Large Московский государственный университет} \textsc{\large имени }\textsc{\Large М.В.Ломоносова}}
\\[0.3cm] 
\HRule 
\\[0.3cm]
\textbf{\textsc{\large Факультет биоинженерии и биоинформатики}}
\\[4.0cm]

\begin{spacing}{1.4}
{ \LARGE \bfseries Численные методы в задачах обработки данных} \\[1.0cm]

\end{spacing}
 
 
\Large \emph{}\\
Курс лекций \\
Лектор - Попов Дмитрий Александрович
\\[4cm]

\begin{abstract}
	Курс включает обзор основных численных методов, применяемых при обработке экспериментальных данных. Основные темы курса: полиномиальная аппроксимация, интерполяция, сплайны, численное дифференцирование и интегрирование, метод наименьших квадратов, методы решения систем линейных уравнений, решение нелинейных уравнений и оптимизация, анализ Фурье.
	
	
	\footnotesize{Исходные материалы \href{https://github.com/litvinanna/num_analysis}{github.com/litvinanna/num_analysis}. }
\end{abstract}


\vfill

{\large Москва \\ \today}


\end{titlepage}

 \newpage
        \tableofcontents \newpage
        %
%\begin{figure}[h] % picture
%	\centering
%	\includegraphics[width = 0.1\textwidth]{}
%	\caption{caption}
%	\label{fig:plasmid}	
%\end{figure}
%
%
%
%\begin{figure*}[h] % two pictures
%	\centering
%	\begin{subfigure}[t]{0.1\linewidth}
%		\includegraphics[width = \textwidth]{}
%		\caption{caption 1}\label{fig:alpha}
%	\end{subfigure}
%	\begin{subfigure}[t]{0.1\linewidth}
%		\includegraphics[width = \textwidth]{}
%		\caption{caption 2}\label{fig:beta}
%
%	\end{subfigure}
%	\caption{great caption}
%	\label{fig:structs}
%	
%\end{figure*}
%
%
%
%\begin{table}[p]
%	\small
%	\caption{caption}
%	\label{table:strains}
%	\begin{tabular}{ p p p }
%
%		Название штамма & Описание & Источник \\ 
%
%		W303 & MAT\textbf{a} ade2-101 his3-11 trp1-1 ura3-52 can1-100 leu2-3,112, GAL, psi+ & $^1$ \\  
%
%
%		
%		блабла & блабал & jdsjf \\
%
%	\end{tabular}
%	
%\end{table}
%
%

\begin{figure*}[h] % picture
\begin{subfigure}[t]{0.4\linewidth}
	\centering
	\begin{tikzpicture}
	
	\begin{axis}[
	width = \linewidth,
	xmin=0,   xmax=8,
	ymin=-1,   ymax=5,
	axis y line*=left,
	axis x line*=bottom,
	axis lines = left,
	domain=0:30,
	grid = major,
	tick label style = {white},
	ylabel = {$y$}, ylabel style={rotate=-90},
	xlabel = {$x$},
	every axis x label/.style={
		at={(ticklabel* cs:1.0)},
		anchor=west,
	},
	every axis y label/.style={
		at={(ticklabel* cs:1.0)},
		anchor=south,
	}
	]
	
	\coordinate (A) at (0, 4);
	\coordinate (B) at (5, 4);
	\coordinate (C) at (5, 0);
	\coordinate (D) at (8, 0);
	\draw (A) -- (B) -- (C) -- (D);
	
	
	\end{axis}

	\end{tikzpicture}
	\end{subfigure}
	\begin{subfigure}[t]{0.2\linewidth}
	\begin{tikzpicture}[scale=2]
		\begin{axis}[
		width = \linewidth,
		xmin=0, xmax=1,
		ymin=0, ymax=2,
		unit vector ratio=1 1,
		axis line style={draw=none},
		tick style={draw=none},
		tick label style={color=white}
		]
		
		\draw [->] (0, 1) -- (1, 1);	
		\node[scale=0.3] (A) at (0.5, 1.5) {сглаживание};
		\end{axis}

	\end{tikzpicture}
		
	\end{subfigure}
	\begin{subfigure}[t]{0.4\linewidth}
	
	\centering
	\begin{tikzpicture}
	
	\begin{axis}[
	width = \linewidth,
	xmin=0,   xmax=8,
	ymin=-1,   ymax=5,
	axis y line*=left,
	axis x line*=bottom,
	axis lines = left,
	domain=0:30,
	grid = major,
	tick label style = {white},
	ylabel = {$y$}, ylabel style={rotate=-90},
	xlabel = {$x$},
	every axis x label/.style={
		at={(ticklabel* cs:1.0)},
		anchor=west,
	},
	every axis y label/.style={
		at={(ticklabel* cs:1.0)},
		anchor=south,
	}
	]
	
	\coordinate (A) at (0, 4);
	\coordinate (A1) at (3, 4);
	\coordinate (A2) at (3.4, 3.9);
	\coordinate (A3) at (3.6, 4.1);
	
	\coordinate (B1) at (4, 3.5);
	\coordinate (B2) at (5, 4.3);
	\coordinate (B3) at (5.3, 4.7);
	
	
	\coordinate (C1) at (5.5, -0.5);
	\coordinate (C2) at (6, 0);
	\coordinate (C3) at (6.3, 0.3 );
	\coordinate (C4) at (6.5, -0.1);
	
	\coordinate (D1) at (7, 0);
	\coordinate (D2) at (8,0);
	
	\draw plot [smooth] coordinates {(A) (A1) (A2) (A3) (B1)  (B2)  (C1) (C2) (C3) (C4) (D1) (D2) } ;
	\end{axis}

	\end{tikzpicture}
	\end{subfigure}

	
	\caption{}
	\label{fig:}
\end{figure*}
 \newpage
        \section{Введение}

В этой вводной лекции рассматривается вопрос о том, на каком этапе и какие численные методы возникают при обработке экспериментальных данных (ЭД).

Начнём с конкретного примера. В 1881 году Д.И.\,Менделеев исследовал зависимость растворимости $NaNO_{2}$ в воде от температуры. Были получены следующие данные:

\begin{table}[h]
	\small
	\caption{Зависимость растворимости соли от температуры в эксперименте Менделеева.}
	\label{table:mendeleev}
	\begin{tabular}{ p{0.25\textwidth} *{10}{|c} }

		T (температура), °C & 0 & 4 & 10 & 15 & 21& 29& 36& 51&68 & x \\
		\hline
		Масса $NaNO_{2}$ в 100~мл воды, г&
		66,7&71,0&76,3&80,6&85,7&99,4&99,4&113,6&125,1& y \\

	\end{tabular}
	
\end{table}

Это типичный пример исходных экспериментальных данных -- конечная таблица $x_i | y_i, i=1\dots n$. Задача состоит в том, чтобы найти зависимость $y = f(x)$ по этим данным. В рассматриваемом примере предполагается, что $f(x)$ неизвестна и предполагается, что $y_i = f(x_i) + \varepsilon _i$, где $\varepsilon _ i$ -- ошибки.

Что касается ошибок, то мы будем предполагать, что известна величина $\varepsilon$ такая, что с нужной вероятностью $|\varepsilon _i| \leq \varepsilon$ и $\varepsilon$ много меньше наблюдаемых значений ($\varepsilon \ll y_i$). В частности, если $\varepsilon_i$ независимые, одинаково распределенные случайные величины с нулевым средним $\langle \varepsilon_i \rangle = 0$ и дисперсией $\langle \varepsilon^2 \rangle$, то $\varepsilon \sim \sqrt{\langle \varepsilon^2 \rangle }$. В более общем случае $\varepsilon$ характеризуется величиной доверительного интервала. 

Таким образом, предполагается, что из априорных соображений или путем статистической обработки мы определили величину $\varepsilon$. Это всё, что нам нужно от статистики, и вопросы статистической обработки ЭД в лекциях рассматриваться не будут.



Если представить данные Д.И.\,Менделеева графически, то возникает картина, схематически представленная на рисунке . 
Эта картина позволяет предположить, что исходная зависимость имеет вид 
\begin{equation}
	y = f(x) = ax + b.
\end{equation}
 Возникает вопрос -- как определить величины $a$ и $b$?  Правильная (с точки зрения математической статистики) стратегия состоит в применении метода наименьших квадратов (МНК). Согласно этому методу величины $a$ и $b$ ищутся  из условия 
 \begin{equation}
	Q(a, b) = \sum^n_{i=1}{(a+bx_i -y_i)^2} = \min_{a,b}
\end{equation}

Так как величина $Q(a,b)$ квадратична по $a,b$, то условия
\begin{equation}
	\frac{\partial Q}{\partial a} = 0,    \frac{\partial Q}{\partial b} = 0
\end{equation}
приводят к системе из двух линейных уравнений. Их решение $a = 67,5, b = 0,87$ нашел Д.И.\,Менделеев. Необходимо ещё проверить, что для функции $f(x) = 67,5 + 0,87x$ выполняются условия $|f(x_i) - y_i| < \varepsilon$ и убедиться в том, что величины $a,b$ мало меняются при замене $y_i$ на $y_i = \varepsilon _i$. Если это так, то задача решена и при этом никаких численных методов нам не понадобилось. Это связано с тем, что зависимость простая (всего два параметра) и число измерений невелико. В случае более сложной зависимости возникает картина, представленная на следующем рисунке .

В этом случае можно попытаться искать зависимость в виде 
\begin{equation} \label{eq:representation}
	f(x) = a_1 \varphi _1(x) + a_2 \varphi _2(x) + \dots + a_m \varphi _1(m),
\end{equation}
где $\varphi _k(x)$ --известные функции, например
\begin{equation}
	\varphi _k(x) = x^{k-1},  k = 1 \dots m
\end{equation}

Согласно МНК величины $a_1 \dots a_m$ ищутся из условия

 \begin{equation}
	Q(a_1 \dots a_m) = \sum^n_{i=1}{(f(x_i) -y_i)^2} = \min_{a_1 \dots a_m}
\end{equation}

и определяются исходя из полученной системы из $m$ линейных уравнений.

Возникают следующие вопросы:
\begin{enumerate}
	\item Как выбрать функции $\varphi _k(x)$ и величины $m$;
	\item Как решать полученную систему линейных уравнений;
	\item Как определить устойчиво ли решение относительно малых изменений величин $y_i$.
\end{enumerate}


Выбирая набор функций $\varphi _k(x)$ мы должны быть уверенны, что любую зависимость можно с нужной точностью представить в виде \ref{eq:representation}. Эта задача решается в теории приближений (теории аппроксимаций) и ей посвящены 2, 3, 4 лекции. Вопросы 2,3 -- это вопросы линейной алгебры и они будут рассмотрены в 5, 6, 7, лекциях.

Выше предполагалось, что вид функции $f(x)$ неизвестен. В ряде случаев вид функции  $f(x)$ известен с точностью до конечного числа неизвестных параметров.

Рассмотрим эксперимент, в котором измеряется зависимость от времени концентрации $C(t)$ некоторого вещества, при это известно, что 
\begin{equation}
	C(t) = a_1e^{\lambda_1 t} + a_2 e^{\lambda_2 t},
\end{equation}
экспериментальные данные это таблица $x_i | y_i, i=1\dots n, x_i = t_i, y_i = C(t_i) + \varepsilon_i$. Если величины $\lambda_1, \lambda_2$ неизвестны, задача их определения из условия

 \begin{equation}
	Q(a_1,a_2,\lambda_1, \lambda_2) = \sum^n_{i=1}{(C(t_i) -y_i)^2} = \min_{a_1,a_2,\lambda_1, \lambda_2}
\end{equation}
приводит к системе нелинейных уравнений. Это задача из теории оптимизации. Здесь, по существу, идёт речь о методах решения нелинейных систем уравнений, которые будут рассмотрены в лекциях 9 и 10.

Последние две лекции будут посвящены использования преобразования Фурье в задачах обработки ЭД. Будет объяснено, что такое преобразование Фурье и его дискретные варианты (ДПФ), а также как ДПФ используется в задачах "сглаживания" ЭД и обнаружения периодических сигналов.

Чтобы ориентироваться в численных методах и использовать пакеты прикладных программ, необходимо знакомство с такими понятиями как норма функций, число обусловленности матрицы и её норма, QR и SVD разложение, неподвижная точка отображения и т.д. Все эти понятия будут введены ниже по мере их возникновения в рассматриваемых задачах.

В каждой лекции используется своя нумерация формул. При ссылках на формулы впереди указывается их номер ( 3.5 = формула 5 из лекции 3).


 \newpage
        \section{Приближение функции полиномами}

\label{lecture:2}

В первой лекции было отмечено, что для описания зависимостей полезно иметь систему функций $\varphi _i(x)$ такую, что их значения легко вычисляются и любую непрерывную функцию $f: I \rightarrow \mathbb{R} $ заданную на интервале $ I = [a,b] $ можно с заданной точностью приблизить функциями вида 
\begin{equation}
\varPsi_n(x)=\sum_{i=0}^{n-1}{a_i\varphi_i(x)} \qquad i = 1, 2, 3, ...
\end{equation}

В качестве $\{\varphi_i(x)\}$ можно выбирать различные системы функций, но наиболее важен с практической точки зрения случай 


\begin{equation}
\varphi _i(x) = x^i \qquad i = 0, 1, 2, ...
\end{equation}

Тогда
\begin{equation}\label{eq:polynom}
\varPsi _n(x) = p_n(x) = a_0 + a_1x + a_{n-1}x^{n-1} 
\end{equation}


Полиномы $p_n(x)$ образуют n-мерное \textbf{линейное пространство $\mathscr{P}_n$}, и ниже речь идет о приближении функций  $f:I \rightarrow \mathbb{R} $ полиномами $p_n(x) \in \mathscr{P}_n $. Подчеркнем, что максимальная степень полинома из пространства  $\mathscr{P}_n$ равна $n - 1$.


Определение точности приближения требует введения понятия нормы $\parallel f \parallel$ функции $f$. 
Функции $f: I \rightarrow \mathbb{R} $ образуют линейное пространство $V[f]$, размерность которого бесконечна. Это означает, что существует функции $e_i(x) \; i = 1, 2, 3 ...$ такие, что любая функция $f \in V[f]$ может быть единственным образом представлена в виде бесконечного ряда 
\begin{equation}
f(x) = \sum_{i=0}^{\infty} {c_i e_i(x)}
\end{equation}
В этом случае говорят, что $\{e_i(x)\}$ \textbf{базис} в пространстве $V[f]$.

\textbf{Нормой} в любом векторном пространстве $V$ называется любая функция 
 
\begin{equation}
\parallel \; \parallel \; : V \rightarrow \mathbb{R}^+
\end{equation}
такая что, если $X\in V$, то

\begin{equation}
\begin{array}{l}
\parallel X \parallel \; \geq 0 \; \text{и}  \parallel X \parallel \; = 0 \Rightarrow X = 0 \\
\parallel \alpha X \parallel \; = \alpha \parallel X \parallel  \\ 
\parallel \alpha X + \beta Y\parallel \; \leq \; \parallel \alpha  X \parallel + \parallel \beta Y \parallel \text{(неравенство треугольника)}
\end{array}
\end{equation}

Если $V = \mathbb{R}^n$ (n-мерное вектроное пространство) и $X = (x_1 .. x_n)$, то обычная эвклидова норма $\parallel\;\parallel_2$ (длина вектора $X$) задается равенством
\begin{equation}
\parallel X\parallel_2 \; = \left[\sum_{i=1}^{n} x_i^2 \right]^{1/2}.
\end{equation}

Но в $\mathbb{R}^n$ существуют и другие нормы, например,

\begin{equation}
\begin{gathered}
\parallel X \parallel_1 \; = \sum_{i = 1}^{n}|x_i| \\
\parallel X \parallel_c \; = \max_{i = 1...n}|x_i|
\end{gathered}
\end{equation}

Эти формулы прямо приводят к определению нормы в пространстве функции $f: I \rightarrow \mathbb{R}$ и по определению

\begin{equation}
\begin{gathered}
\parallel f \parallel_2 \; = \left[ \int_{a}^{b} f^2(x)dx \right]^{1/2} \quad \text{($L^2$-норма)}\\
\parallel f \parallel_1 \; = \int_{a}^{b} |f(x)|dx \quad \text{($L^1$-норма)} \\
\parallel f \parallel_c \; = \max_{x \in I}|f(x)| \quad \text{(равномерная норма)}
\end{gathered}
\end{equation} 

С точки зрения приложений наибольший интерес представляет равномерная норма $\parallel \; \parallel_c	$ и точность приближения функции $f$ полиномом $p_n$ задается числом 
\begin{equation}
\parallel f - p_n \parallel_c \; = \max_{x\in [a, b]} |f(x)-p_n(x)|
\end{equation}


Теорема Вейерштрасса утверждает, что любую непрерывную функцию $f: I \rightarrow \mathbb{R}$  с любой заданной точностью $\varepsilon$ можно приблизить полиномом $p_n(x)$ $(n = n(\varepsilon))$. 
Эта теорема говорит только о существовании такого полинома $p_n$, но ничего не говорит о том, как его построить и какова завиимость $n$ от $\varepsilon$. 
Из теоремы Вейерштрасса следует, что $\{x^i\} \; (i = 0, 1, 2, ...)$ базис в пространстве $V[f]$ непрерывных функций $f$ на интервале $I$.

Гораздо интереснее следующая постановка задачи: с какой точностью заданная функция $f$ может быть приближена полиномом заданной степени. 
Оказывается, что среди полиномов степени $n - 1$ существует единственный \textbf{полином $p_n[f]$ наилучшeго приближения}. Если

\begin{equation}\label{eq:best_polynom}
\parallel f - p_n(f) \parallel_c \; = \varepsilon_n[f],
\end{equation}

то для любого полинома $p$ степени $\leq n - 1$

\begin{equation}
\parallel f - p \parallel_c \; > \varepsilon_n[f],
\end{equation}

Это тоже теорема существования, в которой ничего не говорится о том, как найти $p_n(f)$ и  $\varepsilon_n[f]$.

Построение системы полиномов $p_n(f)$ для заданной функции $f$ -- очень трудная задача, точное решение которой при всех $n$ известно только для некоторых функций $f$. 
Однако существует алгоритм посторения $p_n(f)$ для заданного n (например, алгоритм Ремеза).

Важно то, что для величины $\varepsilon_n(f)$ существуют хорошие оценки сверху и для этого не надо точно знать функцию $f$ и достаточно предположиить, что $f \in W^k(M_k, I)$. Это означает, что функция $f$ на интервале $I$ имеет $k$ непрерывных производных и $|f^{(k)}(x)|\leq M_k$. 

Таким образом,

\begin{eqnarray}
W^k[M_k, I] = \{f: I \rightarrow \quad , \text{ -- производные} \nonumber \\
f^{(i)}(x) \text{ непрерывны при $i \leq k$ и } |f^{(k)}(x)| \leq M_k \}
\end{eqnarray}


\textbf{Класс функций $W^k[M_k, I]$} удобен для характеристики точности приближений, так как для любой функции $f \in W^k[M_k, I]$ имеет место достаточно точная оценка

\begin{equation}
\varepsilon_n[f] \leq \left(\frac{b-a}{2}\right)^k \frac{A_k M_k}{n^k},\; A_k = \left(\frac{\pi k}{2}\right) ^k \frac{1}{k!}
\end{equation}

При $k \geq 2$ с помощью формулы Стирлинга получим, что 

\begin{equation}
\varepsilon_n[f] \leq \left(\frac{(b-a)4,3}{n}\right)^k M_k \quad \forall f \in W^k[M_k, I].
\end{equation}

Эта формула позволяет оценить, с какой точностью функция $f \in W^k[M_k, I]$ может быть приближена полиномами степени $\leq n-1$, но мы по-прежнему не имеем метода построения приближений.
Такой метод дает \textbf{интерполяция}.

Рассмотрим \textbf{разбиение} $X = (x_1, x_2,..., x_n)$ интервала $I = [a, b]$. Это означает просто, что заданы $n$ различных точек $x_i \in I, \; i = 1 ... n$ и $x_i \neq x_j$ и не предполагается, что $x_{i+1} > x_i$.

\textbf{Интерполяционный полином} $\pi(f|x) \in \mathscr{P}_n $ зависит от $n$-параметров -- коэффициентов $a_0, a_1 .. a_{n-1}$ \ref{eq:polynom}, которые определяются из условий

\begin{equation}\label{eq:interpolation}
\pi(f|x)(x_j) = f(x_j) \quad j = 1 .. n
\end{equation}

Это $n$ условий для определения $n$ коэффициентов  $a_0, a_1 .. a_{n-1}$ полинома

\begin{equation}
\pi(f|x)(x) = a_0 + a_1x + a_{n-1}x^{x-1}.
\end{equation}
Отсюда следует, что существует только один такой полином и этот полином можно выписать явно. 

Действительно, пусть известны полиномы $e^i(x_i) \in \mathscr{P}_n $, такие что

\begin{equation}
e^i(x_i)= \delta_{ij} =
\begin{cases}
1, i = j \\
0, i \neq j

\end{cases}
\end{equation}

Тогда 
\begin{equation}
\pi_n(f|x)(x) = \sum_{i = 1}^{n} f(x_i)e^i(x).
\end{equation}

Условие интерполяции \ref{eq:interpolation} выполнено так как 

\begin{equation}
\pi_n(f|x)(x_j) = \sum_{i = 1}^{n} f(x_i)\delta_{ij} = f(x_j).
\end{equation}

Легко догадаться, какой вид имеет полином $e^i(x)$. Он задается равенством 

\begin{equation}
\varepsilon^i(x) = \frac{(x-x_1)(x-x_2)..(x-x_{i-1})(x-x_{i+1}) ...(x-x_n)}{(x_i-x_1)(x_i-x_2)..(x_i-x_{i-1})(x_i-x_{i+1})..(x_i - x_n )}
\end{equation}

Мы построили \textbf{интерполяционный полином в форме Лагранжа}. 

Рассмотрим точность интерполяции. Она задается формулой 

\begin{equation}
\parallel f - \pi_n| e(x) \parallel_c \; \leq (1+ \lambda_n(f, x))\varepsilon_n(f)
\end{equation}

и во всяком случае 

\begin{equation}
\lambda_n(f, x) \geq \frac{e_n n}{8 \sqrt{\pi}} \quad (n \geq 2)
\end{equation}

Величина $\lambda_n(f, x)$ показывает насколько интерполяционный полином проигрывает полиному наилучшего приближения $p_n(f)$ \ref{eq:best_polynom}.


Эти величины сильно зависят от выбора разбиения $X$ и в случае \textbf{равномерного разбиения} для которого

\begin{equation}
x_i = a + \frac{b-a}{n-1}(i-1) \quad i = 1...n
\end{equation}

величина $\lambda_n(f, x)$ быстро растет с ростом $n$ и


\begin{equation}
\frac{2^{n-3}}{n^{3/2}} \leq \lambda_n(f, x) \leq 2^n
\end{equation} 


Таким образом, если  $f \in W^k[M_k, I]$, то ошибка интерполяции по равномерной сетке растет как $2^kn^{-k}$.

В 1901 году немецкий физик Рунге пытался интерполировать на интервале $[-1; 1]$ функцию

\begin{equation}
f(x) = \frac{1}{1+25x^2}
\end{equation}

Используя полином 20-ой степени он обнаружил, что при равномерном разбиении ошибка интерполяции катастрофически растет в окрестности точек $x = \pm 1$ и между точками $x = 0,9$ и $1$ она имеет порядок $10^2$.

Этот пример показывает, что использовать интерполяционные полиномы высокого порядка надо с большой осторожностью. Однако их можно использовать, если перейти к неравномерному разбиению, в котором шаг разбиения уменьшается при приближении к концу интервала.
Нужное разбиение задается формулой 

\begin{equation}\label{eq:27}
x_m = \frac{b+a}{2} + \frac{b-a}{2} \cos \frac{\pi(2m-1)}{2n} \quad m = 1...n
\end{equation}

И если $X = \{x_m\}$ и $f \in W^k[M_k, I]$ то 

\begin{equation}
\parallel \pi_n (f(x) -f) \parallel_c \; \leq \left(
9+ \frac{4}{\pi}\mathrm{ln} (n)\right) \left( \frac{b-a}{2}\right)^k \frac{M_kA_k}{n^k}
\end{equation}

Разбиение \ref{eq:27} называется Чебышевским, так как $\cos \frac{\pi(2m-1)}{2n}$ это нули полинома Чебышева $T_n(y) = \cos(n \; arccos(y))$

Эти полиномы играют большую роль в теории приближений.


%$\mathrm{ln}(x) = 10$


 \newpage
        \section{Интерполяционный полином в форме Ньютона \\ Численное дифференцирование и интегрирование}
На прошлой лекции был определен интерполяционный полином в форме Лагранжа (\ref{eq:2.19}, \ref{eq:2.21}). Этот же полином (он единственный) удобнее записывать в форме Ньютона:
\begin{dmath}\label{eq:3.1}
	\pi(f(x), x) = f(x_1) + f(x_1, x_2)(x-x_1) + f(x_1, x_2, x_3)(x-x_1)(x-x_2) + \dots + f(x_1, x_2, \dots, x_n)(x-x_1)\dots(x-x_{n-1})
\end{dmath}

Величины $f(x_1, x_2, \dots, x_k)$ называются разделенными разностями и определяются из рекуррентных соотношений:

\begin{equation}
\begin{cases}
	f(x_1, x_2) = \frac{f(x_2) - f(x_1)}{x_2 - x_1} \\ 
	f(x_1, x_2, x_3) = \frac{f(x_2, x_3) - f(x_1, x_2)}{x_3 - x_1} \\ 
	f(x_1, x_2, x_3, x_4) = \frac{f(x_2, x_3, x_4) - f(x_1, x_2, x_3)}{x_4-x_1} \\
	\dots
\end{cases}
\end{equation}

Так как формула (\ref{eq:3.1}) не зависит от нумерации точек разбиения, то для её доказательства достаточно перейти к нумерации, в которой $x_1$ заменяется на $x_j$ (???). Эта формула удобна тем, что при добавлении точки $x_{n+1}$ к формуле (\ref{eq:3.1}) надо просто прибавить $f(x_1, x_2, \dots, x_n, x_{n+1})(x-x_1)\dots(x-x_{n-1})(x-x_n)$. Кроме того, эта формула указывает на связь интерполяционного полинома с разложением в ряд Тейлора.

Чтобы это увидеть, рассмотрим равномерное разбиение (\ref{eq:2.24}), в котором $x_{i+1}-x_i=\delta$. Тогда 


\begin{dmath}
\begin{aligned}
	f(x_1, x_2) &= \frac{f(x_1+\delta) - f(x_1)}{\delta} \simeq f^{(1)}(x_1) \\
	f(x_1, x_2, x_3) &= \frac{1}{2\delta}f^{(1)}(x_2)-f^{(1)}(x_1) \simeq  \frac{1}{2}f^{(2)}(x_1) \\
	\dots
\end{aligned}
\end{dmath}

Имеется и следующий аналог формулы для остаточного члена в ряду Тейлора:
\begin{equation} \label{eq:3.4}
	|f(x) - \pi_k(f|X)(x)| = \frac{f^{n}(\xi)}{n!} \omega_n (x,X)
\end{equation}
В этой формуле:
\begin{equation}
\begin{aligned}
	&\xi \in [y_1, y_2]; \quad y_1=min_{i}(x-x_i); \quad y_2=max_{i}(x-x_i) ;\\ 
	&\omega_n(x, X) = (x-x_1)\dots(x-x_n)
\end{aligned}
\end{equation}
Формула \ref{eq:3.4} даёт ещё один подход к оценке точности интерполяции и показывает, какие трудности возникают при попытке использовать интеполяционный полином вне интервала интерполяции, то есть для \textbf{экстраполяции}.

\bigskip
Переходим к рассмотрению задач численного дифференцирования и интегрирования. 

Будем использовать обозначение:
\begin{equation}
	\pi_n(f|X)(x) \equiv p_n(x)\\
\end{equation}
таким образом $\pi_n(x)$ --- полином степени $n-1$.

Общая идея состоит в использовании формул вида: 
\begin{equation} \label{eq:2.8}
	f^{(1)}(x) \backsimeq {p_n}^{(1)}(x)
\end{equation}
\begin{equation}
	\int_{a}^{b}f(x) \backsimeq \int_{a}^{b}{p_n}^{(1)}(x)
\end{equation}
Рассмотрим задачу численного дифференцирования. Пусть известны величины $f(x_1), f(x_2), x_2 = x_1 + \delta$, и величина $\delta$ "достаточно мала". Задача состоит в построении приблизительной формулы для величины $f^{(1)}(x_1)$

Используем линейную интерполяцию, т.е. $p_2(x)$:
\begin{equation}
	\begin{aligned}
	p &= f(x_1) + \frac{f(x_2) - f(x_1)}{x_2-x_1}(x-x_1) \\
	{p_2}^{(1)}(x_1) &= \frac{f(x_1+\delta) - f(x_1)}{\delta}
	\end{aligned}
\end{equation}
Тогда получим следующую простейшую формулу численного дифференцирования:
\begin{dmath} \label{eq:3.10}
	f^{(1)}(x_1) \backsimeq \frac{f(x_1 + \delta) - f(x_1)}{\delta} = {p_2}^{(1)}(x_1)
\end{dmath}

Пусть $f \in W^2(M_2, I)$, оценим точность формулы \ref{eq:3.10}.

Разложим правую часть этой формулы в ряд Тейлора. Получим, что
\begin{dmath} 
	\begin{aligned}
f^{(1)}(x_1) &= \frac{1}{\delta} [f(x_1) + \frac{f^{(1)}(x_1)}{1!}\delta + O(M_2\delta^2)-f(x_1)] \\
&= f^{(1)}(x_1) + O(M_2 \delta)
	\end{aligned}
\end{dmath}
Символ \textbf{$O(M_2\delta)$} обозначает величину меньшую, чем $CM_2\delta$ ($C\backsimeq$1). Таким образом
\begin{equation}
	|f^{(1)}(x_1) - {p_2}^{(1)}(x_1)| = O(M_2\delta)
\end{equation}

Теперь предположим, что известны величины $f(x_1), f(x_2), f(x_3)$ и $x_2=x_1+\delta, x_3 = x_2 + \delta$ и вычислим  ${p_3}^{(1)}(x_2)$. Получим, что  
\begin{equation}
	{p_3}^{(1)}(x_2) = \frac{f(x_1+2\delta) - f(x_1)}{2\delta} = \frac{f(x_2+\delta) - f(x_2 - \delta)}{2\delta}
\end{equation}

Пусть $f \in W^3(M_3, I)$, оценим точность формулы
\begin{equation} \label{eq:3.14}
	f^{(1)}(x_2) \backsimeq \frac{f(x_2+\delta) - f(x_2 - \delta)}{2\delta} = {p_3}^{(1)}(x_2)
\end{equation}
Разложим в ряд Тейлора правую часть этого "'равенства"'. Получим, что
\begin{dmath} 
	\begin{aligned}
		{p_3}^{(1)}(x_2) &= \frac{1}{2\delta} [f(x_2) + {f^{(1)}(x_2)}\delta + \frac{f^{(2)}(x_2)}{2} \delta^2 \\ &- f(x_2) + f^{(1)}(x_2)\delta - \frac{f^{(2)}(x_2)}{x_2} + O(M_3\delta^3)]
	\end{aligned}
\end{dmath}
И следовательно
\begin{equation}
	f^{(1)}(x_2) =  {p_3}^{(1)}(x_2) + O(M_3\delta^2)
\end{equation}
При "'малых $\delta$"' формула \ref{eq:3.14} гораздо точнее, чем \ref{eq:3.10}, и именно ее рекомендуется использовать при обработке ЭД.

Численное дифференцирование --- неустойчивая операция, так как в ней присутствует деление на "'малую"' величину $\delta$. Рассмотрим, к чему приводит учёт ошибок в задаче численного дифференцирования.

Если известны только $y_i = f(x_i) + \varepsilon_i$, то формула \ref{eq:3.10} приобретает вид:
\begin{dmath}
	f^{(1)}(x_1) = \frac{y_2 - y_1}{\delta} \backsimeq \frac{f(x_1 + \delta) + \varepsilon_2 - f(x_1) - \varepsilon_1}{\delta} = {p_2}^{(1)}(x_1)
\end{dmath}
И при $f \in W^2(M_2, I)$, поступая так же, как и выше и учитывая, что $\varepsilon_i$ --- случайные величины, получим, что
\begin{dmath}
	\begin{cases}
		\frac{y_2 - y_1}{\delta} = f^{(1)}(x_1) + \Delta_1(\varepsilon, \delta) \\
		\Delta_1(\varepsilon, \delta) \backsimeq M_2\delta + \frac{\varepsilon}{\delta}
	\end{cases}
\end{dmath}
Величина $\Delta_1(\varepsilon, \delta)$ минимальна при
\begin{equation}
	\delta \backsimeq \sqrt{\frac{\varepsilon}{M_2}} = \delta_1; \quad \Delta_1(\varepsilon, \delta_1) = \sqrt{\varepsilon M_2}
\end{equation}
Таким образом, если при наличии ошибок мы выбираем $\delta<\delta_1$, то ошибка численного дифференцирования не уменьшится, а возрастет. В эксперименте $\delta$ --- это интервал между измерениями, и т.о. при наличии шума (случайных ошибок) не следует мерить слишком часто.

В случае формулы \ref{eq:3.14} аналогично предыдущему получим, что
\begin{dmath}
	\begin{cases}
		\frac{y_3 - y_1}{\delta} = f^{(1)}(x_2) + \Delta_2(\varepsilon, \delta) \\ 
		\Delta_2(\varepsilon, \delta) \backsimeq M_3\delta^2 + \frac{\varepsilon}{\delta} \quad (|\varepsilon_i|<\varepsilon)
	\end{cases}
\end{dmath}
Оптимальной является величина
\begin{equation}
	\delta \backsimeq {(\frac{\varepsilon}{M_2})}^{\frac{1}{3}} = \delta_2; \quad \Delta_2(\varepsilon, \delta_1) = \varepsilon^{\frac{2}{3}} \delta^{\frac{1}{3}}
\end{equation}
Так как $\varepsilon$ --- "'малая величина"' , то 
\begin{equation}
	\delta_2 >> \delta_1 \quad но \quad \Delta_2(\varepsilon, \delta_2) \backsimeq \Delta_1(\varepsilon, \delta_1)
\end{equation}

Переходим к формулам численного интегрирования. В пакетах прикладных программ эти формулы называются \textbf{квадратурными}.

В отличие от численного дифференцирования, численное интегрирование --- это устойчивая операция, и учет ошибок $\varepsilon_i$ не играет роли при оценке точности квадратурных формул.

Будем предполагать, что задано разбиение $X=(x_0=a<x_1<x_2\dots<x_n=b)$ и известны величины $f(x_i)$. Если число $n$ велико, то использование формулы \ref{eq:2.8} для построения квадратурных формул требует использования полиномов высокого порядка. Чтобы этого избежать, на практике используются \textbf{составные квадратурные формулы}. Эти формулы основаны на том, что
\begin{equation}
	\int_{a}^{b} f(x)dx = \int_{a}^{c}f(x)dx + \int_{b}^{c}f(x)dx
\end{equation}
Для получения составных квадратурных формул  отрезок $[ab]$ делится на отрезки, содержащие небольшое количество точек разбиения, и в каждом таком интервале используется интерполяционный полином небольшого порядка.

Рассмотрим, например, формулу
\begin{equation}
	\int_{a}^{b} f(x)dx = \sum_{i=0}^{n-1}\int_{x_i}^{x_{i+1}}f(x)dx
\end{equation}
Простейшая квадратурная формула, \textbf{формула трапеции}, получится, если в каждом из интегралов по отрезкам $[x_ix_{i+1}]$ $f(x)$ заменить на $p_2(x)$. В случае равномерного разбиения
\begin{equation}
x_i = a + ih; \quad h = \frac{a-b}{N} \quad (n=N) \quad i=0 \dots N
\end{equation}
формула трапеций имеет вид
\begin{dmath}
	\begin{cases}
		J = \int_{a}{b}fdx \backsimeq J^T_N \\ 
		J^T_N = [\frac{1}{2}(f(a)+f(b)) + \sum_{k=1}^{N-1}f(x_k)]\frac{b-a}{N}
	\end{cases}
\end{dmath}
и, если $f \in W^2(M_2,I)$, то
\begin{equation}
	|J-J^T_N| <= \frac{(b-a)^3}{N^2}M_2
\end{equation}

Кроме формулы трапеции на практике часто применяется \textbf{формула Симпсона}. Рассмотрим эту формулу. Предполагается, что $n=2N$, т.е.
\begin{equation}
	X = (x_0 x_1 x_2, x_2 x_3 x_4, \dots x_{2N-2}x_{2N-1}x_{2N})
\end{equation}
и интеграл $J$ записывается в виде
\begin{equation}
	J=\int_{x_0}^{x_2}f(x)dx + \int_{x_2}^{x_4}f(x)dx + \dots \int_{2N-2}^{2N}f(x)dx
\end{equation}
В каждом из интегралов в правой части этой формулы $f(x)$ заменяется на $p_3(x)$. Если разбиение равномерно
\begin{equation}
	x_k = a + \frac{h}{2}k, \quad h=\frac{b-a}{N}
\end{equation}
то в формуле Симпсона
\begin{equation}
	J \backsimeq J_N^S
\end{equation}
величина $J_N^S$ имеет вид
\begin{equation}
	J_N^S = \frac{b-a}{6N}[f(a) + f(b) + 4\sum_{k=1}^{N}f(x_{2k-1}) + 2\sum_{k=1}^{N}f(x_2k)]
\end{equation}
При $f \in W^4(M_4,I)$
\begin{equation}
	|J-J^S_N| <= \frac{M_4}{12}\frac{(b-a)^5}{N^4}
\end{equation}
При больших $N$ эта формула существенно точнее формулы трапеции.

В пакетах прикладных программ содержится много других квадратурных формул. Однако они практически не используются при обработке ЭД. Исключением являются только квадратурные формулы на основе сплайнов (см. Лекцию 4). \newpage
        \section{Сплайны}



На прошлых лекциях мы рассматривали задачи приближения функции $f:I \rightarrow\mathbb{R}$ полиномами, то есть элементами линейного пространства $P_n$ размерности $n$. Чтобы распространить другие методы приближения, прежде всего надо ввести некоторое новое конечномерное векторное пространство заменяющее $P_n$. Такой заменой может служить \textbf{ пространство} $S_{n,v}(X,I)$ \textbf{сплайнов} степени $n\geq 1$, дефекта $v\geq 1$ $(v\leq n)$, построенное по разбиению $X = (x_0=a < x_1 < x_2\ldots < x_N=b)$. 

По определению функция $s:I\rightarrow \mathbb{R}$ принадлежит пространству $S_{n,v}(X,I)$, если выполняются следующие два условия:
\begin{enumerate}
	\item На каждом интервале $(x_i,x_{i+1})$, $s(x)$ это полином степени $n$
	\item На всем интервале $I=[a,b]$ функция $s(x)$ имеет $n-v$ непрерывных производных
\end{enumerate}
Таким образом сплайн это функция "склеенная" из полиномов так, чтобы во внутренних точках $x_i,$ $ i = 1\ldots N-1$ выполняются условия "склейки":
\begin{equation}
s^{\left( k\right) }\left( x_{i}-0\right) =s^{\left( k\right) }\left( x_{i}+0\right),
\begin{aligned}k=0,1\ldots n-v\\ i=1\ldots N-1\end{aligned}
\end{equation}
Здесь через $\varphi(s_i\pm 0)$ обозначаются пределы $\varphi(x)$ при $x \rightarrow x_i$ справа и слева от точки $x_i$ производная порядка $n-v+1$ уже может иметь разрывы в точках ???.

Ясно, что $S_{n,v}(X,I)$ - линейное пространство и его размерность:
\begin{equation}
dim S_{n,v}(X,I) = v(N-1) +n+1
\end{equation}
Докажем эту формулу. У нее имеется $N$ интервалов и в каждом из них свой полином порядка $n$, который зависит от $n+1$ параметра (коэффициентов полинома). Таким образом:
\begin{equation}
\textit{число параметров} = N(n+1)
\end{equation} 
С другой стороны, в каждой из $N-1$ внутренних точек должны быть непрерывны производные порядка $0,1\ldots n-v$ и следовательно:
\begin{equation}
\textit{число условий} = (N-1)(n-v+1)
\end{equation}
Таким образом:
\begin{equation}
\begin{aligned}dim S_{n,v}(X,I) = \textit{число параметров - число условий} =\\
 N(n+1)-(N-1)(n-v+1)=v(N-1)+n+1\end{aligned}
\end{equation}
Наиболее часто используются сплайны дефекта $1$ и
\begin{equation}
dim  S_{n,1}(X,I) = N+n
\end{equation}
В пакетах прикладных программ встречаются такие понятия как \textbf{B-сплайны} и \textbf{фундаментальные сплайны}. Под B-сплайнами понимаются сплайны деффекта 1 с выбором специального базиса $B_n^j(x)$ $j=-n,-n+1,\ldots N-1$ в пространстве $S_{n,1}(X,I)$. Сплайны $B_n^j(x)$ локальны, т.е.  $B_n^j(x)\neq 0$ только если $x\in(x_i,x_i+n+1)$. При этом предпологается, что к разбиению $X$ добавлены точки $x_{-n}<x_{-n+1}<\ldots<a$ и $b<x_{N+1}<x_{N+2}<\ldots<x_{N+n}$ мы не будем давать опрделение сплайнам $B_n^j(x)$, хотя они иногда применяются для сглаживания ЭД.

Фундаментальные сплайны используются в задачах интерполяции. Эти сплайны не принадлежат пространству $S_{n,1}(X,I)$,  они принадлежат пространству $S_{n,1}(\Delta,I)$, где:
\begin{equation}
\begin{aligned}\Delta=(\varepsilon_1\ldots\varepsilon_{N-n}), \\
x_i<\varepsilon_i<x_{i+n}\end{aligned}
\end{equation}
Выбор точек $\varepsilon_1\ldots\varepsilon_{N-n}$ зависит от рассматриваемой задачи и:
\begin{equation}
dim  S_{n,1}(\Delta,I) = N+1
\end{equation}
В пространстве $S_{n,1}(\Delta,I)$ существует базис $e^i(x)$ такой , что:
\begin{equation}
e^i(x_k)=\delta_{ik},\;\; 	i=0\ldots N
\end{equation}
Знание базиса $e^i(x)$ позволяет решить задачу интерполяции функции $f(x)$. Если известны величины $f(x_k)$ это решение имеет вид:
\begin{equation}
s\left( x\right) =\sum ^{N}_{i=0}e^{i}\left( x\right) f\left( x_{i}\right)
\end{equation}
Так как из (9) следует, что:
\begin{equation}
s\left( x_k\right) =f\left( x_{k}\right),\ \  k=0\ldots N
\end{equation}

Сплайны бывают \textbf{интерполяционными} и \textbf{сглаживающими}. Сплайн называется \textbf{интерполяционными}, если выполнены условия (11).

Наибольший интерес представляют интерполяционные сплайны из $S_{3,1}(X,I)$ - \textbf{кубические сплайны деффекта 1}. Число условий (11) равно $N+1$, в то время как:
\begin{equation}
dim  S_{3,1}(X,I) = N+3
\end{equation}
Таким образом для определения интерполяционного сплайна $s(x)$ надо задать еще два условия. Наиболее часто используются условия:
\begin{equation}
s^{( 1) }(a) =\dfrac{f\left( x_{1}\right) -f\left( x_{0}\right) }{x_{1}-x_{0}},s^{\left( 1\right) }\left( b\right) =\dfrac{\left( x_{N}\right)-f( x_{N}-1)}{x_{N}-x_{N-1}}
\end{equation}
или условия:
\begin{equation}
s^{(2) }(a) =s^{(2) }(b) 
\end{equation}
Пусть такие условия заданы, тогда для определения сплайна $s(x)$ имеется(cм (4)) $3(N-1)+N+1+2=4N$ условий для поределения $4N$ коэффициентов $a^i_0,a^i_1,a^i_2,a^i_3$ определяющих полином третьего порядка в интервале $(x_i;x_{i+1}), \ (i=0,1\ldots N-1)$, т.е. надо решить систему из $4N$ линейных уравнений. 


Кроме упомянутых выше сплайнов в пакетах прикладных программ встречаются \textbf{Эрмитовы сплайны}. Это элементы пространства $S_{3,2}(X,I)$, и:
\begin{equation}
dim  S_{3,2}(X,I) = 2(N+1)
\end{equation}
Если известны величины $f(x_i),f^{(1)}(x_i)$, то существует Эрмитов сплайн $s(x)$ такой, что:
\begin{equation}
s(x_i)=f(x_i), \ s^{(1)}(x_i)=f^{(1)}(x_i)
\end{equation}

В Лекции \ref{lecture:2} были отмечены трудности, возникающие при использовании интерполяционных полиномов высокого порядка. К этим трудностям добавляется ещё и неустойчивость полинома Лагранжа относительно малых вариаций величин $f(x_i)$.
Сплайны лишены этих недотатков. Пусть $f\in W^k(M_k,I)$ и рассмотрим точность интерполяции функции $f$ с помощью кубического сплайна деффекта 1. Тогда:
\begin{equation}
\parallel s-f\parallel ?? \leq C_k h^k M_k
\end{equation}
и в этой формуле $k=max(x_{i+1}-x_i)$,
\begin{equation}
C_{2}=\dfrac{13}{48},C_{3}=\dfrac{41}{564},C_{4}=\dfrac{5}{364}
\end{equation}
и никаких трудностей типа быстрого роста $\lambda_n(f,X)$ (2.22) c ростом $n$ (2.25) - не возникает.

Английское слово "spline" означает гибкую, упругую ленту. Английские корабелы, сами того не зная, использовали сплайны. На следующем рисунке сплошной линией показана правая часть поперечного разреза парусника. 

\begin{figure}[H] % picture
	\centering
	\begin{tikzpicture}
	\begin{axis}[
	width = 10cm,
	height = 7cm,
	xmin=0,   xmax=10,
	ymin=0,   ymax=7,
	axis y line*=left,
	axis x line*=bottom,
	axis lines = left,
	domain=0:30,
	xtick={0, 0.5, 2, 8},
	xticklabels = {$a$,$x_1$,$x_2$, $x_n = b$},
	ytick={1, 2, 6},
	yticklabels = {$f_1$, $f_2$, $f_n $},
	ylabel = {$y$}, ylabel style={rotate=-90},
	xlabel = {$x$},
	every axis x label/.style={
		at={(ticklabel* cs:1.0)},
		anchor=west,
	},
	every axis y label/.style={
		at={(ticklabel* cs:1.0)},
		anchor=south,
	}
	]
	
\coordinate (A) at (0,0);
\coordinate (B) at (0.5, 1);
\coordinate (C) at (2, 2);
\coordinate (D) at (4, 2.5);
\coordinate (D1) at (6.5, 3.5);
\coordinate (E) at (7.7, 4.5);
\coordinate (F) at (8, 6);

\draw[fill] (A) circle (2pt);
\draw[fill] (B) circle (2pt);
\draw[fill] (C) circle (2pt);
\draw[fill] (D) circle (2pt);
\draw[fill] (D1) circle (2pt);
\draw[fill] (E) circle (2pt);
\draw[fill] (F) circle (2pt);

\draw[thick,black] plot [smooth, tension=0.7] coordinates {(A) (B) (C) (D) (D1) (E) (F)};

\draw [dashed] (B) -- (B -| 0, 0);
\draw [dashed] (B) -- (B |- 0, 0);
\draw [dashed] (C) -- (C -| 0, 0);
\draw [dashed] (C) -- (C |- 0, 0);
\draw [dashed] (D) -- (D -| 0, 0);
\draw [dashed] (D) -- (D |- 0, 0);
\draw [dashed] (E) -- (E -| 0, 0);
\draw [dashed] (E) -- (E |- 0, 0);

\draw [dashed] (F) -- (F -| 0, 0);
\draw [dashed] (F) -- (F |- 0, 0);

\end{axis}
\end{tikzpicture}
	\caption{Градиентный спуск.}
	\label{fig:spline}
\end{figure}










Исходными данными служили точки $\ast$. Требовалось провести через эти точки гладкую кривую. Упругая лента укладывалась так, чтобы она проходила через точки $\ast$. Это и рассматривалось как чертеж поперечного сечения будущего корабля. 

Оказывается, что полученная кривая $n=f(x)$ -- это $f(x)=s(x)$, где $s(x)$ интерполяционный сплайн из $S_{3,1}(X,I)$ с дополнительными условиями (14).


С математической точки зрения это означает, что $f(x)=s(x)$ -- это решение следующей задачи: найти функцию $f(x)$, которая дает минимум величины:
\begin{equation}
J[f]=\int_a^b[f^{(2)}(x)]^2dx
\end{equation}
при условии $f(x_i)=f_i$ ($f_i$ - заданы).
Вместо $J[f]$ можно рассмотреть величину:
\begin{equation}
J[ f| S_{0},S_1\ldots S_{N}] = J \left[ f\right] + \sum ^{N}_{i=0}\rho^{-1}_{i}\left[ f\left( x_{i}\right) -f_{i}\right] ^{2}
\end{equation}
где $\rho_i$ - заданы. Минимум этой велчины - это снова $f(x)=s(x), \ s\in S_{3,1}(X,I)$, 
но теперь уже условия $s(x_i)=f_i$ не выполняются и решение называется сглаживающим сплайном. Величины $s_i$ выбираются из условия $s_i\approx \varepsilon$, где $\varepsilon$ - величина ошибки, с которой известны величины $f(x_i)$. \newpage
        \footnotesize \bibliographystyle{mybib5.bst}  
        \bibliography{bib.bib}
\end{document}