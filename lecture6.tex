\section{Метод наименьших квадратов (2)}


В лекции \ref{lecture5} было показано, что реализация МНК сводится к решению системы \textbf{нормальных} уравнений 
\begin{equation} \label{eq:6.1}
	Bx = a, где B = A^TA, a = A^Tb
\end{equation}
и $A=M_{m \times n}$  -- матрица плана. решение системы \ref{eq:6.1} называется \textbf{квазирешением} переопределенной и не имеющей решений (при $m\gg n$) системы уравнений 
\begin{equation}
	Ax = b
\end{equation}
Рассмотрим методы поиска квазирешений, то есть рещений системы \ref{eq:6.1}. 

В пакетах прикладных программ содержится большое количество различных методов решений систем вида \ref{eq:6.1}. Ниже будут приведены некоторые из этих методов, чаще всего используемые в контексте МНК. При этом мы ограничимся методами дающими точное значение квазирешения. В соответствии с этим, методы последовательных приближений (итерационные) рассматриваться не будут.

Предполагается, что ранг матрицы $B$ равен $n$
\begin{equation}
	rank B = n
\end{equation}