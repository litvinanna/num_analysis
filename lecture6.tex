\section*{Лекция 6. Метод наименьших квадратов (2)}
\addcontentsline{toc}{section}{Лекция 6. Метод наименьших квадратов (2)}

В лекции \ref{lecture5} было показано, что реализация МНК сводится к решению системы \textbf{нормальных} уравнений 
\begin{equation} \label{eq:6.1}
	Bx = a, где B = A^TA, a = A^Tb
\end{equation}
и $A=M_{m \times n}$  -- матрица плана. решение системы \ref{eq:6.1} называется \textbf{квазирешением} переопределенной и не имеющей решений (при $m\gg n$) системы уравнений 
\begin{equation}
	Ax = b
\end{equation}