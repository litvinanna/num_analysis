\section{Метод наименьших квадратов (2)}
\label{lecture:6}

В лекции \ref{lecture5} было показано, что реализация МНК сводится к решению системы \textbf{нормальных} уравнений 
\begin{equation} \label{eq:6.1}
	Bx = a, где B = A^TA, a = A^Tb
\end{equation}
и $A=M_{m \times n}$  -- матрица плана. решение системы \ref{eq:6.1} называется \textbf{квазирешением} переопределенной и не имеющей решений (при $m\gg n$) системы уравнений 
\begin{equation}
	Ax = b
\end{equation}
Рассмотрим методы поиска квазирешений, то есть рещений системы \ref{eq:6.1}. 

В пакетах прикладных программ содержится большое количество различных методов решений систем вида \ref{eq:6.1}. Ниже будут приведены некоторые из этих методов, чаще всего используемые в контексте МНК. При этом мы ограничимся методами дающими точное значение квазирешения. В соответствии с этим, методы последовательных приближений (итерационные) рассматриваться не будут.

Предполагается, что ранг матрицы $B$ равен $n$
\begin{equation} 
	rank B = n
\end{equation}
Следовательно
\begin{equation}
	det B \neq 0
\end{equation}
и следовательно решение $x$ сущетсвует и единственно.

Все описанные ниже методы решения системы \ref{eq:6.1} основаны на представлении матрицы $B$ в виде
\begin{equation} 
	B = B_1 B_2 \text{ или } B = B_1 B_2 B_3
\end{equation}
где матрицы $B_i$ \textbf{легко обращаются}.
Так как
\begin{equation} 
	(B_1B_2)^{-1} = B_2^{-1} B_1^{-1}, (B_1B_2B_3)^{-1} = B_3^{-1}B_2^{-1} B_1^{-1},
\end{equation}
то решение системы \ref{eq:6.1} имеет вид
\begin{equation}
	x = B^{-1} a = B_2^{-1} B_1^{-1} a \text{ или } x = B_3^{-1}B_2^{-1} B_1^{-1}.
\end{equation}

Легко обращаются следующие типы квадратных матриц $M_{n \times n}$. Это верхние и нижние треугольные матрицы и ортогональные матрицы.

\textbf{Верхнетреугольные матрицы} имеют вид
\begin{equation} 
	L = \begin{pmatrix}
		e_{11} & e_{12} & \dots &e_{1(n-1)}&e_{1n} \\
		0 & e_{22} & \dots &e_{2(n-1)}&e_{2n} \\
		%		0 & 0 & \dots &e_{3(n-1)}&e_{3n} \\
		\vdots & \vdots & \ddots &\vdots &  \vdots\\
		0 & 0 & \dots & e_{(n-1)(n-1)} &e_{(n-1)n} \\
		0 & 0 & \dots & 0 &e_{nn} \\
	\end{pmatrix}
\end{equation}
и соответственно \textbf{нижнетреугольные} матрицы вида 
\begin{equation} 
	L = \begin{pmatrix}
		e_{11}	&		&		&	&        \\
		&e_{22}	&   		& \text{\huge0} &\\
		&		&e_{33} &	&            \\
		&\dots & & \ddots &          \\
		& 		&   		&   & e_{nn} 
	\end{pmatrix},
\end{equation}
где заштрихованная область содердит ненулевые элементы. Уравнения $Lx = y$ легко решаются, а решение $x = L^{-1}y$ задает обратную матрицу $L^{-1}$.


Матрица $P \in M_{n \times n}$ называется \textbf{ортогональной}, если
\begin{equation} \label{eq:6.11}
	(Px, Py) = (x, y), \forall x, y \in \mathbb{R}^n = M_{n \times 1}
\end{equation}
откуда сразу следует, что
\begin{equation} \label{eq:6.12}
	P^{-1} = P^T .
\end{equation}

Произвольные матрицы $B, B'$ называются \textbf{подобными}, если существуют \textbf{обратимые} (имеющие обратную) матрицы $Q_1, Q_2$ такие что
\begin{equation}
	B' = Q_1BQ_2
\end{equation}
и соответственно
\begin{equation}
	B = Q_2^{-1}B'Q_1^{-1}.
\end{equation}

Оказывается, что, если матрица $B$ -- \textbf{симметрична}, то есть если
\begin{equation} \label{eq:6.14}
	B = B^T
\end{equation}
то существует ортогональная матрица $P$ такая, что
\begin{equation}
	B' = P^{-1}BP
\end{equation}
и матрица $B'$ -- \textbf{диагональна}, то есть $B$ подобна диагональной, то есть приводится к диагональному виду
\begin{equation}
	B' = \begin{pmatrix}
		\lambda_1 & \dots & 0  \\
		\vdots & \ddots & \vdots \\
		0 & \dots & \lambda_n 
	\end{pmatrix}
\end{equation}
и тогда, если $\lambda_i \neq 0 (detB \neq 0)$, то
\begin{equation}
	B^{-1} = P(B')^{-1}P^{-1}, \quad
	(B')^{-1} = \begin{pmatrix}
		\lambda_1^{-1} & \dots & 0  \\
		\vdots & \ddots & \vdots \\
		0 & \dots & \lambda_n^{-1} 
	\end{pmatrix}.
\end{equation}

Такой метод решения системы \ref{eq:6.1} (при $B^T = B$) называется \textbf{приведением к диагональному виду}, а величины $\lambda_1 \dots \lambda_n$ -- собственные значения матрицы $B$. Если в пакете прикладных программ есть программа нахождения собственных значений симметричных матриц, то, как правило, эта программа выдает и матрицу $P$ и следовательно решение системы нормальных уравнений, так как в этом случае выполняется условие \ref{eq:6.14}.

Кроме этого рассмотрим еще \textbf{метод Гаусса} и \textbf{метод Холесского}.

\textbf{Метод Гаусса} (метод последовательного исключения неизвестных) основан на том, что любая \textbf{невырожденная} (не обязательно симметричная) матрица ($detB \neq 0$) может быть представлена в виде
\begin{equation}
	B = L_1L_2
\end{equation}
где $L_1$ -- верхнетреугольная, $L_2$ -- нижнетреугольная с единицами на диагонали.

\textbf{Метод Холесского} применим, если 
\begin{equation} \label{eq:6.19}
	B^T = B \textrm{ и } (Bx, x) > 0, \forall x>0.
\end{equation}

Оба эти условия выполняются для $B = A^TA$. Если имеют место условия \ref{eq:6.18}, то матрицу $B$ можно представить в виде
\begin{equation}
	B = LL^T
\end{equation}
где $L$ -- нижнетреугольная матрица с положительными числами $\mu_i, i=1 \dots n$ на диагонали. Тогда
\begin{equation}
	\lambda_i = \mu_i^2
\end{equation}
собственные значения матрицы $B$.

Кроме указанных трех методов при решении нормальных уравнений используются \textbf{методы QR} и \textbf{SVD} (singular value decomposition) \textbf{разложений}. Подчеркнем, что эти методы не требуют знания нормальных уравнений и для поиска квазирешений используется только матрица плана $A$.

В \textbf{методе QR} разложения ищется верхнетругольная матрица $R \in M_{n \times n}$ и \textbf{квазиортогональная} матрица $Q \in M_{m \times n}$, такая, что 
\begin{equation}
	Q \in M_{m \times n}, Q^TQ = E \quad 
	E \equiv \begin{pmatrix}
		1 & \dots & 0  \\
		\vdots & \ddots & \vdots \\
		0 & \dots & 1 
	\end{pmatrix} \in M_{n \times n}.
\end{equation}

Утверждается, что существуют $R, Q$ такие, что при условии \ref{eq:6.3}
\begin{equation} \label{eq:6.23}
	A = QR.
\end{equation}

Определим $x$ равенством
\begin{equation} \label{eq:6.24}
	x = R^{-1}Q^Tb.
\end{equation}

Из \ref{eq:6.23} и \ref{eq:6.24} следует, что $A^TAx = A^Tb$ и следовательно равенство \ref{eq:6.24} задает квазирешение.

\textbf{Метод SVD} основан на том, что любую матрицу $A \in M_{m \times n}$ можно представить в виде 
\begin{equation}
	A = U\Sigma V^T
\end{equation}
где $U \in M_{m \times m}$ -- ортогональная матрица, $V \subset M_{n \times n}$ -- ортогональная матрица и $\Sigma \in M_{m \times n}$ -- квазидиагональная матрица вида
\begin{equation}
	\Sigma \in M_{m \times n} \quad 
	\Sigma = \begin{pmatrix}
		\begin{matrix}
			\sigma_1 &  &  & \text{\huge0} \\
			 & \sigma_2 & & \\
			&  & \ddots & \\
			\text{\huge0}& & & \sigma_n 
		\end{matrix} \\
		\hline 
		\bigzero
	\end{pmatrix}, \sigma_1 \geq \sigma_2 \geq \dots \geq \sigma_n.
\end{equation}

Рассмотрим матрицу $\Sigma' \in M_{n \times m}$ -- квазиобратную к $\Sigma$
\begin{equation} 
	\Sigma' = \begin{pmatrix}
		\begin{matrix}
		\sigma_1^{-1} & \dots & 0  \\
		\vdots & \ddots & \vdots \\
		0 & \dots & \sigma_n^{-1}
		\end{matrix} 
		& \rvline & & \bigzero &
	\end{pmatrix}.
\end{equation}

Тогда 
\begin{equation}
	\Sigma'\Sigma = E \in M_{n \times n}.
\end{equation}

Утверждается, что 
\begin{equation}
	x = V \Sigma'U^Tb \textrm{ -- квазирешение}.
\end{equation}

Величины $\sigma_1 \dots \sigma_n$ называются \textbf{сингулярными числами} и это собственные значения матрицы $B = A^TA$.

Пусть найдено квазирешение $x$. Рассмотрим вопрос о его \textbf{устойчивости}. В уравнении \ref{eq:6.1} $a = A^Tb$ и вектор $b$ задан с ошибкой порядка $\varepsilon$. Решение $x$ имеет смысл только если при "малом"\ изменении $a x$ также изменится "мало".

Пусть $x$ -- решение уравнения $Bx = a$ и соответственно $x + \delta x$ -- решение уравнения $B(x+\delta x) = a+\delta a$. Тогда устойчивость решения $x$ характеризуется числом \parbox{16mm}{$\kappa = \kappa(b)$} -- \textbf{числом обсуловленности матрицы $B$}. Величина $\kappa(B)$ определяется из условия 
\begin{equation} \label{eq:6.30}
	\frac{\|\delta x\|}{\|x\|} \leq \kappa(B) \frac{\|\delta a\|}{\|a\|}.
\end{equation}

Ниже рассматривается представляющий наибольший интерес случай Эвклидовой нормы ($\|x\| = \sqrt{(x, x)}$). Чтобы найти $\kappa(B)$, введем понятие \textbf{нормы $\|B\|$ матрицы $B$}. По определению
\begin{equation} \label{eq:6.31}
	\|B\| = \max_{x \neq 0} \frac{\|Bx\|}{\|x\|} = \max_{\|x\| = 1} \|Bx\|.
\end{equation}

Указанная норма называется \textbf{спектральной}. Возможны и другие определения нормы матрицы. Напрмимер, \textbf{Эвклидова норма $\|B\|_2$} определяется равенством
\begin{equation}
	\|B\|_2 = [\sum^n_{i, j=1}{(a_{ij})^2}]^{\frac{1}{2}}.
\end{equation}

Можно показать, что
\begin{align}
	\begin{split}
		\|B\| &= \max |\lambda_i| \\
		\|B\|_2 &= [\sum^n_{i=1}{(x_{i})^2}]^{\frac{1}{2}},
	\end{split}
\end{align}
где $\lambda_i, i=1 \dots n$ -- собственные значения матрицы $B = B^T$. Из \ref{eq:6.31} следует, что 
\begin{equation}
	 \|Bx\| \leq \|B\|\|x\|.
\end{equation}

Покажем, что 
\begin{equation}
	\kappa(B) = \|B\|\|B^{-1}\|.
\end{equation}

По определению величины $\delta x$ в неравенстве \ref{eq:6.30} $\delta x = B^{-1} \delta a$ и следовательно
\begin{equation} \label{eq:6.36}
	\|\delta x\| \leq \|B^{-1}\|\|\delta a\|.
\end{equation} 

С другой стороны, $\|Bx\| = \|a\|$ и следовательно
\begin{equation} \label{eq:6.37}
	\|x\| \geq \|a\|\|B^{-1}\|.
\end{equation}

Из \ref{eq:6.36} и \ref{eq:6.37} следует, что
\begin{equation}
	\frac{\|\delta x\|}{\|x\|} \leq \|B\|\|B^{-1}\| \frac{\delta a}{a}.
\end{equation}

Заметим, что SVD разложение автоматически дает число обсловленности
\begin{equation}
	\kappa(B) = \frac{\sigma_1}{\sigma_n}, \kappa(B) = \frac{\lambda_\max}{\lambda_\min}, B=A^TA.
\end{equation}

Если необходимо решить систему нормальных уравнений \ref{eq:6.1}, то прежде чем это делать надо определить $\kappa(B)$. Пакетом прикладных программ, который не дает числа обсуловленности пользоваться не следует.

Если полученная с помощью \ref{eq:6.30} относительная ошибка $\|\delta x\|\|x\|^{-1}$ нас устраивает, то остается только проверить, что 
\begin{equation} \label{eq:6.40}
	|\overline{f}(t_i) - b_i| < \varepsilon.
\end{equation}

Если величина $\kappa(B)$ слишком великак или не выполняется условие \ref{eq:6.40}, то необходимо \textbf{пересматривать матрицу плана}.