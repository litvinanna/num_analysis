\section{Лекция 3. Интерполяционный полином в форме Ньютона. Численное дифференцирование и интегрирование.}
На прошлой лекции был определен интерполяционный полином в форме Лагранжа (\ref*{2.19}) (\ref*{2.21}). Этот же полином (он единственный) удобнее записывать в форме Ньютона:
\begin{dmath}
	\pi(f(x), x) = f(x_1) + f(x_1, x_2)(x-x_1) + f(x_1, x_2, x_3)(x-x_1)(x-x_2) + \dots + f(x_1, x_2, \dots, x_n)(x-x_1)\dots(x-x_{n-1})
\end{dmath}
Величины $f(x_1, x_2, \dots, x_k)$ называются разделенными разностями и определяются из рекуррентных соотношений:
\begin{dmath}
\begin{cases}
	f(x_1, x_2) = \frac{f(x_2) - f(x_1)}{x_2 - x_1} \\ 
	f(x_1, x_2, x_3) = \frac{f(x_2, x_3) - f(x_1, x_2)}{x_3 - x_1} \\ 
	f(x_1, x_2, x_3, x_4) = \frac{f(x_2, x_3, x_4) - f(x_1, x_2, x_3)}{x_4-x_1} \\
	\dots
\end{cases}
\end{dmath}
Так как формула (\ref{eq:3.1}) не зависит от нумерации точек разбиения, то для её доказательства достаточно перейти к нумерации, в которой $x_1$ заменяется на $x_j$ (???). Эта формула удобна тем, что при добавлении точки $x_{n+1}$ к формуле (\ref{eq:1}) надо просто прибавить $f(x_1, x_2, \dots, x_n, x_{n+1})(x-x_1)\dots(x-x_{n-1})(x-x_n)$. Кроме того, эта формула указывает на связь интерполяционного полинома с разложением в ряд Тейлора.\\
Чтобы это увидеть, рассмотрим равномерное разбиение (\ref{2.24}), в котором $x_{i+1}-x_i=\delta$. Тогда 
\begin{dmath}
\begin{aligned}
	f(x_1, x_2) = \frac{f(x_1+\delta) - f(x_1)}{\delta} \simeq f^{(1)}(x_1) \\
	f(x_1, x_2, x_3) = \frac{1}{2\delta}f^{(1)}(x_2)-f^{(1)}(x_1) \simeq  \frac{1}{2}f^{(2)}(x_1) \\
	\dots
\end{aligned}
\end{dmath}
Имеется и следующий аналог формулы для остаточного члена в ряду Тейлора:
\begin{equation}
	|f(x) - \pi_k(f|X)(x)| = \frac{f^{n}(\xi)}{n!} \omega_n (x,X)
\end{equation}
В этой формуле:
\begin{equation}
	\xi \in [y_1, y_2] ;\quad y_1=min_{i}(x-x_i) ;\quad y_2=max_{i}(x-x_i) ;\quad \omega_n(x, X) = (x-x_1)\dots(x-x_n)
\end{equation}
Формула \ref{eq:4} даёт ещё один подход к оценке точности интерполяции и показывает, какие трудности возникают при попытке использовать интеполяционный полином вне интервала интерполяции, то есть для \textit{экстраполяции}.

\bigskip
Переходим к рассмотрению задач численного дифференцирования и интегрирования. 

Будем использовать обозначение:
\begin{equation}
	\pi_n(f|X)(x) \equiv p_n(x)\\
\end{equation}
т. о. $\pi_n(x)$ --- полином степени $n-1$.

Общая идея состоит в использовании формул вида 
\begin{equation}
	f^{(1)}(x) \backsimeq {p_n}^{(1)}(x)
\end{equation}
\begin{equation}
	\int_{a}^{b}f(x) \backsimeq \int_{a}^{b}{p_n}^{(1)}(x)
\end{equation}
Рассмотрим задачу численного дифференцирования. Пусть известны величины $f(x_1), \quad f(x_2), \quad x_2 = x_1 + \delta$, и величина $\delta$ "достаточно мала". Задача состоит в построении приблизительной формулы для величины $f^{(1)}(x_1)$

Используем линейную интерполяцию, т.е. $p_2(x)$:
%\begin{equation}
%p = f(x_1) + \fraq{f(x_2) - f(x_1)}{x_2-x_1}(x-x_1)
%	p^{(1)} (x_1) = \fraq{f(x_1+\delta) - f(x_1)}{\delta}
%\end{equation}
Тогда получим следующую простейшую формулу численного дифференцирования: