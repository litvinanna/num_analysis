\section{Интерполяционный полином в форме Ньютона \\ Численное дифференцирование и интегрирование}
На прошлой лекции был определен интерполяционный полином в форме Лагранжа (\ref{eq:2.19}, \ref{eq:2.21}). Этот же полином (он единственный) удобнее записывать в форме Ньютона:
\begin{dmath}\label{eq:3.1}
	\pi(f(x), x) = f(x_1) + f(x_1, x_2)(x-x_1) + f(x_1, x_2, x_3)(x-x_1)(x-x_2) + \dots + f(x_1, x_2, \dots, x_n)(x-x_1)\dots(x-x_{n-1})
\end{dmath}

Величины $f(x_1, x_2, \dots, x_k)$ называются разделенными разностями и определяются из рекуррентных соотношений:

\begin{equation}
\begin{cases}
	f(x_1, x_2) = \frac{f(x_2) - f(x_1)}{x_2 - x_1} \\ 
	f(x_1, x_2, x_3) = \frac{f(x_2, x_3) - f(x_1, x_2)}{x_3 - x_1} \\ 
	f(x_1, x_2, x_3, x_4) = \frac{f(x_2, x_3, x_4) - f(x_1, x_2, x_3)}{x_4-x_1} \\
	\dots
\end{cases}
\end{equation}

Так как формула (\ref{eq:3.1}) не зависит от нумерации точек разбиения, то для её доказательства достаточно перейти к нумерации, в которой $x_1$ заменяется на $x_j$ (???). Эта формула удобна тем, что при добавлении точки $x_{n+1}$ к формуле (\ref{eq:3.1}) надо просто прибавить $f(x_1, x_2, \dots, x_n, x_{n+1})(x-x_1)\dots(x-x_{n-1})(x-x_n)$. Кроме того, эта формула указывает на связь интерполяционного полинома с разложением в ряд Тейлора.

Чтобы это увидеть, рассмотрим равномерное разбиение (\ref{eq:2.24}), в котором $x_{i+1}-x_i=\delta$. Тогда 


\begin{dmath}
\begin{aligned}
	f(x_1, x_2) &= \frac{f(x_1+\delta) - f(x_1)}{\delta} \simeq f^{(1)}(x_1) \\
	f(x_1, x_2, x_3) &= \frac{1}{2\delta}f^{(1)}(x_2)-f^{(1)}(x_1) \simeq  \frac{1}{2}f^{(2)}(x_1) \\
	\dots
\end{aligned}
\end{dmath}

Имеется и следующий аналог формулы для остаточного члена в ряду Тейлора:
\begin{equation} \label{eq:3.4}
	|f(x) - \pi_k(f|X)(x)| = \frac{f^{n}(\xi)}{n!} \omega_n (x,X)
\end{equation}
В этой формуле:
\begin{equation}
\begin{aligned}
	&\xi \in [y_1, y_2]; \quad y_1=min_{i}(x-x_i); \quad y_2=max_{i}(x-x_i) ;\\ 
	&\omega_n(x, X) = (x-x_1)\dots(x-x_n)
\end{aligned}
\end{equation}
Формула \ref{eq:3.4} даёт ещё один подход к оценке точности интерполяции и показывает, какие трудности возникают при попытке использовать интеполяционный полином вне интервала интерполяции, то есть для \textbf{экстраполяции}.

\bigskip
Переходим к рассмотрению задач численного дифференцирования и интегрирования. 

Будем использовать обозначение:
\begin{equation}
	\pi_n(f|X)(x) \equiv p_n(x)\\
\end{equation}
таким образом $\pi_n(x)$ --- полином степени $n-1$.

Общая идея состоит в использовании формул вида: 
\begin{equation} \label{eq:2.8}
	f^{(1)}(x) \backsimeq {p_n}^{(1)}(x)
\end{equation}
\begin{equation}
	\int_{a}^{b}f(x) \backsimeq \int_{a}^{b}{p_n}^{(1)}(x)
\end{equation}
Рассмотрим задачу численного дифференцирования. Пусть известны величины $f(x_1), f(x_2), x_2 = x_1 + \delta$, и величина $\delta$ "достаточно мала". Задача состоит в построении приблизительной формулы для величины $f^{(1)}(x_1)$

Используем линейную интерполяцию, т.е. $p_2(x)$:
\begin{equation}
	\begin{aligned}
	p &= f(x_1) + \frac{f(x_2) - f(x_1)}{x_2-x_1}(x-x_1) \\
	{p_2}^{(1)}(x_1) &= \frac{f(x_1+\delta) - f(x_1)}{\delta}
	\end{aligned}
\end{equation}
Тогда получим следующую простейшую формулу численного дифференцирования:
\begin{dmath} \label{eq:3.10}
	f^{(1)}(x_1) \backsimeq \frac{f(x_1 + \delta) - f(x_1)}{\delta} = {p_2}^{(1)}(x_1)
\end{dmath}

Пусть $f \in W^2(M_2, I)$, оценим точность формулы \ref{eq:3.10}.

Разложим правую часть этой формулы в ряд Тейлора. Получим, что
\begin{dmath} 
	\begin{aligned}
f^{(1)}(x_1) &= \frac{1}{\delta} [f(x_1) + \frac{f^{(1)}(x_1)}{1!}\delta + O(M_2\delta^2)-f(x_1)] \\
&= f^{(1)}(x_1) + O(M_2 \delta)
	\end{aligned}
\end{dmath}
Символ \textbf{$O(M_2\delta)$} обозначает величину меньшую, чем $CM_2\delta$ ($C\backsimeq$1). Таким образом
\begin{equation}
	|f^{(1)}(x_1) - {p_2}^{(1)}(x_1)| = O(M_2\delta)
\end{equation}

Теперь предположим, что известны величины $f(x_1), f(x_2), f(x_3)$ и $x_2=x_1+\delta, x_3 = x_2 + \delta$ и вычислим  ${p_3}^{(1)}(x_2)$. Получим, что  
\begin{equation}
	{p_3}^{(1)}(x_2) = \frac{f(x_1+2\delta) - f(x_1)}{2\delta} = \frac{f(x_2+\delta) - f(x_2 - \delta)}{2\delta}
\end{equation}

Пусть $f \in W^3(M_3, I)$, оценим точность формулы
\begin{equation} \label{eq:3.14}
	f^{(1)}(x_2) \backsimeq \frac{f(x_2+\delta) - f(x_2 - \delta)}{2\delta} = {p_3}^{(1)}(x_2)
\end{equation}
Разложим в ряд Тейлора правую часть этого "'равенства"'. Получим, что
\begin{dmath} 
	\begin{aligned}
		{p_3}^{(1)}(x_2) &= \frac{1}{2\delta} [f(x_2) + {f^{(1)}(x_2)}\delta + \frac{f^{(2)}(x_2)}{2} \delta^2 \\ &- f(x_2) + f^{(1)}(x_2)\delta - \frac{f^{(2)}(x_2)}{x_2} + O(M_3\delta^3)]
	\end{aligned}
\end{dmath}
И следовательно
\begin{equation}
	f^{(1)}(x_2) =  {p_3}^{(1)}(x_2) + O(M_3\delta^2)
\end{equation}
При "'малых $\delta$"' формула \ref{eq:3.14} гораздо точнее, чем \ref{eq:3.10}, и именно ее рекомендуется использовать при обработке ЭД.

Численное дифференцирование --- неустойчивая операция, так как в ней присутствует деление на "'малую"' величину $\delta$. Рассмотрим, к чему приводит учёт ошибок в задаче численного дифференцирования.

Если известны только $y_i = f(x_i) + \varepsilon_i$, то формула \ref{eq:3.10} приобретает вид:
\begin{dmath}
	f^{(1)}(x_1) = \frac{y_2 - y_1}{\delta} \backsimeq \frac{f(x_1 + \delta) + \varepsilon_2 - f(x_1) - \varepsilon_1}{\delta} = {p_2}^{(1)}(x_1)
\end{dmath}
И при $f \in W^2(M_2, I)$, поступая так же, как и выше и учитывая, что $\varepsilon_i$ --- случайные величины, получим, что
\begin{dmath}
	\begin{cases}
		\frac{y_2 - y_1}{\delta} = f^{(1)}(x_1) + \Delta_1(\varepsilon, \delta) \\
		\Delta_1(\varepsilon, \delta) \backsimeq M_2\delta + \frac{\varepsilon}{\delta}
	\end{cases}
\end{dmath}
Величина $\Delta_1(\varepsilon, \delta)$ минимальна при
\begin{equation}
	\delta \backsimeq \sqrt{\frac{\varepsilon}{M_2}} = \delta_1; \quad \Delta_1(\varepsilon, \delta_1) = \sqrt{\varepsilon M_2}
\end{equation}
Таким образом, если при наличии ошибок мы выбираем $\delta<\delta_1$, то ошибка численного дифференцирования не уменьшится, а возрастет. В эксперименте $\delta$ --- это интервал между измерениями, и т.о. при наличии шума (случайных ошибок) не следует мерить слишком часто.

В случае формулы \ref{eq:3.14} аналогично предыдущему получим, что
\begin{dmath}
	\begin{cases}
		\frac{y_3 - y_1}{\delta} = f^{(1)}(x_2) + \Delta_2(\varepsilon, \delta) \\ 
		\Delta_2(\varepsilon, \delta) \backsimeq M_3\delta^2 + \frac{\varepsilon}{\delta} \quad (|\varepsilon_i|<\varepsilon)
	\end{cases}
\end{dmath}
Оптимальной является величина
\begin{equation}
	\delta \backsimeq {(\frac{\varepsilon}{M_2})}^{\frac{1}{3}} = \delta_2; \quad \Delta_2(\varepsilon, \delta_1) = \varepsilon^{\frac{2}{3}} \delta^{\frac{1}{3}}
\end{equation}
Так как $\varepsilon$ --- "'малая величина"' , то 
\begin{equation}
	\delta_2 >> \delta_1 \quad но \quad \Delta_2(\varepsilon, \delta_2) \backsimeq \Delta_1(\varepsilon, \delta_1)
\end{equation}

Переходим к формулам численного интегрирования. В пакетах прикладных программ эти формулы называются \textbf{квадратурными}.

В отличие от численного дифференцирования, численное интегрирование --- это устойчивая операция, и учет ошибок $\varepsilon_i$ не играет роли при оценке точности квадратурных формул.

Будем предполагать, что задано разбиение $X=(x_0=a<x_1<x_2\dots<x_n=b)$ и известны величины $f(x_i)$. Если число $n$ велико, то использование формулы \ref{eq:2.8} для построения квадратурных формул требует использования полиномов высокого порядка. Чтобы этого избежать, на практике используются \textbf{составные квадратурные формулы}. Эти формулы основаны на том, что
\begin{equation}
	\int_{a}^{b} f(x)dx = \int_{a}^{c}f(x)dx + \int_{b}^{c}f(x)dx
\end{equation}
Для получения составных квадратурных формул  отрезок $[ab]$ делится на отрезки, содержащие небольшое количество точек разбиения, и в каждом таком интервале используется интерполяционный полином небольшого порядка.

Рассмотрим, например, формулу
\begin{equation}
	\int_{a}^{b} f(x)dx = \sum_{i=0}^{n-1}\int_{x_i}^{x_{i+1}}f(x)dx
\end{equation}
Простейшая квадратурная формула, \textbf{формула трапеции}, получится, если в каждом из интегралов по отрезкам $[x_ix_{i+1}]$ $f(x)$ заменить на $p_2(x)$. В случае равномерного разбиения
\begin{equation}
x_i = a + ih; \quad h = \frac{a-b}{N} \quad (n=N) \quad i=0 \dots N
\end{equation}
формула трапеций имеет вид
\begin{dmath}
	\begin{cases}
		J = \int_{a}{b}fdx \backsimeq J^T_N \\ 
		J^T_N = [\frac{1}{2}(f(a)+f(b)) + \sum_{k=1}^{N-1}f(x_k)]\frac{b-a}{N}
	\end{cases}
\end{dmath}
и, если $f \in W^2(M_2,I)$, то
\begin{equation}
	|J-J^T_N| <= \frac{(b-a)^3}{N^2}M_2
\end{equation}

Кроме формулы трапеции на практике часто применяется \textbf{формула Симпсона}. Рассмотрим эту формулу. Предполагается, что $n=2N$, т.е.
\begin{equation}
	X = (x_0 x_1 x_2, x_2 x_3 x_4, \dots x_{2N-2}x_{2N-1}x_{2N})
\end{equation}
и интеграл $J$ записывается в виде
\begin{equation}
	J=\int_{x_0}^{x_2}f(x)dx + \int_{x_2}^{x_4}f(x)dx + \dots \int_{2N-2}^{2N}f(x)dx
\end{equation}
В каждом из интегралов в правой части этой формулы $f(x)$ заменяется на $p_3(x)$. Если разбиение равномерно
\begin{equation}
	x_k = a + \frac{h}{2}k, \quad h=\frac{b-a}{N}
\end{equation}
то в формуле Симпсона
\begin{equation}
	J \backsimeq J_N^S
\end{equation}
величина $J_N^S$ имеет вид
\begin{equation}
	J_N^S = \frac{b-a}{6N}[f(a) + f(b) + 4\sum_{k=1}^{N}f(x_{2k-1}) + 2\sum_{k=1}^{N}f(x_2k)]
\end{equation}
При $f \in W^4(M_4,I)$
\begin{equation}
	|J-J^S_N| <= \frac{M_4}{12}\frac{(b-a)^5}{N^4}
\end{equation}
При больших $N$ эта формула существенно точнее формулы трапеции.

В пакетах прикладных программ содержится много других квадратурных формул. Однако они практически не используются при обработке ЭД. Исключением являются только квадратурные формулы на основе сплайнов (см. Лекцию 4).